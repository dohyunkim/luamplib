% \iffalse meta-comment -- by the way, this file contains UTF-8
%
% Copyright (C) 2008-2025 by Hans Hagen, Taco Hoekwater, Elie Roux,
% Manuel Pégourié-Gonnard, Philipp Gesang and Kim Dohyun.
% Currently maintained by Kim Dohyun.
% Support: <https://github.com/lualatex/luamplib>
%
% This work is under the GPL v2.0 license.
%
% This work consists of the main source file luamplib.dtx
% and the derived files
%    luamplib.sty, luamplib.lua and luamplib.pdf.
%
% Unpacking:
%    tex luamplib.dtx
%
% Documentation:
%    lualatex luamplib.dtx
%
%<*ignore>
\begingroup
  \def\x{LaTeX2e}%
\expandafter\endgroup
\ifcase 0\ifx\install y1\fi\expandafter
         \ifx\csname processbatchFile\endcsname\relax\else1\fi
         \ifx\fmtname\x\else 1\fi\relax
\else\csname fi\endcsname
%</ignore>
%<*install>
\input docstrip.tex
\Msg{************************************************************************}
\Msg{* Installation}
\Msg{* Package: luamplib - metapost package for LuaTeX.}
\Msg{************************************************************************}

\keepsilent
\askforoverwritefalse

\let\MetaPrefix\relax

\preamble

See source file '\inFileName' for licencing and contact information.

\endpreamble

\let\MetaPrefix\DoubleperCent

\generate{%
  \usedir{tex/luatex/luamplib}%
  \file{luamplib.sty}{\from{luamplib.dtx}{package}}%
}

\def\MetaPrefix{-- }

\def\luapostamble{%
  \MetaPrefix^^J%
  \MetaPrefix\space End of File `\outFileName'.%
}

\def\currentpostamble{\luapostamble}%

\generate{%
  \usedir{tex/luatex/luamplib}%
  \file{luamplib.lua}{\from{luamplib.dtx}{lua}}%
}

\obeyspaces
\Msg{************************************************************************}
\Msg{*}
\Msg{* To finish the installation you have to move the following}
\Msg{* files into a directory searched by TeX:}
\Msg{*}
\Msg{*     luamplib.sty luamplib.lua}
\Msg{*}
\Msg{* Happy TeXing!}
\Msg{*}
\Msg{************************************************************************}

\endbatchfile
%</install>
%<*ignore>
\fi
%</ignore>
%<*driver>
\NeedsTeXFormat{LaTeX2e}
\ProvidesFile{luamplib.drv}%
  [2025/03/20 v2.37.2 Interface for using the mplib library]%
\documentclass{ltxdoc}
\usepackage{metalogo,multicol,xspace}
\usepackage[x11names]{xcolor}
%
\def\primarycolor{DodgerBlue4}  %%-> rgb  16  78 139 | #104e8b
\def\secondarycolor{Goldenrod4} %%-> rgb 139 105 200 | #8b6914
%
\usepackage[
    bookmarks=true,
   colorlinks=true,
    linkcolor=\primarycolor,
     urlcolor=\secondarycolor,
    citecolor=\primarycolor,
     pdftitle={The luamplib package},
   pdfsubject={Interface for using the mplib library},
    pdfauthor={Hans Hagen, Taco Hoekwater, Elie Roux, Philipp Gesang \& Kim Dohyun},
  pdfkeywords={luatex, lualatex, mplib, metapost}
]{hyperref}
\usepackage{fontspec}
\setmainfont[
  Numbers     = OldStyle,
  Ligatures   = TeX,
  BoldFont    = {Linux Libertine O Bold},
  ItalicFont  = {Linux Libertine O Italic},
  SlantedFont = {Linux Libertine O Italic},
]{Linux Libertine O}
\setmonofont[Scale=MatchLowercase]{InconsolataN}
%setsansfont[Ligatures=TeX]{Linux Biolinum O}
\setsansfont[UprightFont=*Medium,BoldFont=*Heavy,Ligatures=TeX,Scale=MatchLowercase]{Iwona}
%setmathfont{XITS Math}

\usepackage{hologo}

\newcommand\ConTeXt {Con\TeX t\xspace}

\newcommand*\email [1] {<\href{mailto:#1}{#1}>}
\newcommand \file       {\nolinkurl}

\newcommand*\metapost {\textsc{metapost}\xspace}
\newcommand*\mplib {\pkg{mplib}\xspace}

\begin{document}
  \DocInput{luamplib.dtx}%
\end{document}
%</driver>
% \fi
%
% \CheckSum{0}
%
% \CharacterTable
%  {Upper-case    \A\B\C\D\E\F\G\H\I\J\K\L\M\N\O\P\Q\R\S\T\U\V\W\X\Y\Z
%   Lower-case    \a\b\c\d\e\f\g\h\i\j\k\l\m\n\o\p\q\r\s\t\u\v\w\x\y\z
%   Digits        \0\1\2\3\4\5\6\7\8\9
%   Exclamation   \!     Double quote  \"     Hash (number) \#
%   Dollar        \$     Percent       \%     Ampersand     \&
%   Acute accent  \'     Left paren    \(     Right paren   \)
%   Asterisk      \*     Plus          \+     Comma         \,
%   Minus         \-     Point         \.     Solidus       \/
%   Colon         \:     Semicolon     \;     Less than     \<
%   Equals        \=     Greater than  \>     Question mark \?
%   Commercial at \@     Left bracket  \[     Backslash     \\
%   Right bracket \]     Circumflex    \^     Underscore    \_
%   Grave accent  \`     Left brace    \{     Vertical bar  \|
%   Right brace   \}     Tilde         \~}
%
% \title{The \pkg{luamplib} package}
% \author{Hans Hagen, Taco Hoekwater, Elie Roux, Philipp Gesang and Kim Dohyun\\
% Current Maintainer: Kim Dohyun\\
% Support: \url{https://github.com/lualatex/luamplib}}
% \date{2025/03/20 v2.37.2}
%
% \maketitle
%
% \begin{abstract}
% Package to have \metapost code typeset directly in a document with \LuaTeX.
% \end{abstract}
%
% \section{Documentation}
%
% This package aims at providing a simple way to typeset directly \metapost
% code in a document with \LuaTeX. \LuaTeX\ is built with the Lua
% \mplib library, that runs \metapost code. This package is basically a
% wrapper for the Lua \mplib functions and some \TeX\
% functions to have the output of the \mplib functions in the pdf.
%
% Using this package is easy: in Plain, type your \metapost code between the
% macros \cs{mplibcode} and \cs{endmplibcode}, and in \LaTeX\ in the
% |mplibcode| environment.
%
% The resulting \metapost figures are put in a \TeX\ |hbox| with dimensions
% adjusted to the \metapost code.
%
% The code of \pkg{luamplib} is basically from the |luatex-mplib.lua| and |luatex-mplib.tex| files
% from \ConTeXt. They have been adapted to \LaTeX\ and Plain by Elie Roux and
% Philipp Gesang and new functionalities have been added by Kim Dohyun.
% The most notable changes are:
%
% \begin{itemize}
% \item possibility to use |btex ... etex| to typeset \TeX\ code.
%   |textext|\,\meta{string} is a more versatile macro equivalent to |TEX|\,\meta{string} from \pkg{TEX.mp}.
%   |TEX| is also allowed and is a synonym of |textext|.
%   The argument of
%   \mplib's primitive |maketext| will also be processed by the same routine.
%
% \item possibility to use |verbatimtex ... etex|, though it's behavior cannot
%   be the same as the stand-alone \pkg{mpost}.
%   Of course you cannot include \cs{documentclass}, \cs{usepackage} etc.
%   When these \TeX\ commands are found in |verbatimtex ... etex|, the entire code will be
%   ignored.
%   The treatment of |verbatimtex| command has changed a lot since v2.20:
%   see \hyperlink{mpliblegacybehavior}{below} \S\,1.1.
%
% \item in the past,
%   the package required PDF mode in order to have some output.
%   Starting with version 2.7 it works in DVI mode as well, though
%   DVIPDFMx is the only DVI tool currently supported.
% \end{itemize}
%
% It seems to be convenient to divide the explanations of some more changes and cautions
% into three parts: \TeX, \textsc{MetaPost}, and Lua interfaces.
%
% \subsection{\TeX}
%
% \subsubsection{\cs{mplibforcehmode}}
%   \hypertarget{mplibforcehmode}{}\relax
%   When this macro is declared, every \metapost figure box will be
%   typeset in horizontal mode, so \cs{centering}, \cs{raggedleft} etc
%   will have effects. \cs{mplibnoforcehmode}, being default, reverts this
%   setting.\footnote{Actually these commands redefine \cs{prependtomplibbox}. So you
%     can redefine this command with anything suitable before a box.
%     But see \hyperlink{taggedPDF}{below} on Tagged PDF. }
%
% \subsubsection{\cs{everymplib\{...\}}, \cs{everyendmplib\{...\}}}
%   \cs{everymplib} and \cs{everyendmplib} redefine
%   the lua table containing \metapost code
%   which will
%   be automatically inserted at the beginning and ending of each \metapost code chunk.
%\begin{verbatim}
%     \everymplib{ beginfig(0); }
%     \everyendmplib{ endfig; }
%     \begin{mplibcode}
%       % beginfig/endfig not needed
%       draw fullcircle scaled 1cm;
%     \end{mplibcode}
%\end{verbatim}
%
% \subsubsection{\cs{mplibsetformat\{plain\textbar metafun\}}}
%   There are (basically) two formats for \metapost: \emph{plain} and
%   \emph{metafun}. By default, the \emph{plain} format is used, but you can set
%   the format to be used by future figures at any time using
%   \cs{mplibsetformat}\marg{format name}.
%
%   \textsc{n.b.} As \emph{metafun} is such a complicated format,
%   we cannot support all the functionalities producing special effects provided by \emph{metafun}.
%   At least, however, transparency (actually opacity), shading (gradient colors) and transparency group
%   are fully supported,
%   and outlinetext is supported by our own alternative |mpliboutlinetext|
%   (see \hyperlink{mpliboutlinetext}{below} \S\,1.2).
%   You can try other effects as well, though we did not fully tested their proper functioning.
%
% \begin{description}
% \item[transparency]
%   (|texdoc metafun| \S\,8.2) Transparency is so simple that you can apply it to an object,
%   with \emph{plain} format as well as \emph{metafun},
%   just by appending |withprescript| |"tr_transparency=|\meta{number}|"| to the sentence.
%   ($0 \le \meta{number} \le 1$)
%
%   From v2.36, |withtransparency| is available with \emph{plain} as well.
%   See \hyperlink{luamplibtransparency}{below} \S\,1.2.
%
% \item[shading]
%   (|texdoc metafun| \S\,8.3) One thing worth mentioning about shading is:
%   when a color expression is given in string type,
%   it is regarded by \pkg{luamplib} as
%   a color expression of \TeX\ side.
%   For instance, when |withshadecolors("orange", 2/3red)| is given, the first color |"orange"| will be
%   interpreted as a \pkg{color}, \pkg{xcolor} or \pkg{l3color}'s expression.
%
%   From v2.36, shading is available with \emph{plain} format as well with extended functionality.
%   See \hyperlink{luamplibshading}{below} \S\,1.2.
%
% \item[transparency group]
%   \hypertarget{metafunformat}{}\relax
%   (|texdoc metafun| \S\,8.8)
%   As for transparency group, the current \emph{metafun} document is not correct.
%   The true syntax is:
%\begin{verbatim}
%     draw <picture>|<path> asgroup <string>
%\end{verbatim}
%   where \meta{string} should be |""| (empty), |"isolated"|, |"knockout"|, or |"isolated,knockout"|.
%   Beware that currently many of the PDF rendering applications, except Adobe Acrobat Reader,
%   cannot properly render the isolated or knockout effect.
%
%   Transparency group is available with \emph{plain} format as well, with extended functionality.
%   See \hyperlink{transparencygroup}{below} \S\,1.2.
% \end{description}
%
% \subsubsection{\cs{mplibnumbersystem\{scaled\textbar double\textbar decimal\}}}
%   Users can choose |numbersystem| option.
%   The default value is |scaled|, which can be changed
%   by declaring \cs{mplibnumbersystem\{double\}} or
%   \cs{mplibnumbersystem\{decimal\}}.
%
% \subsubsection{\cs{mplibshowlog\{enable\textbar disable\}}}
%   Default: |disable|.
%   When \cs{mplibshowlog\{enable\}}\footnote{As for user's setting,
%   |enable|, |true| and |yes| are identical;
%   |disable|, |false| and |no| are identical.} is declared, log messages returned by
%   the \metapost process will be printed to the |.log| file.
%   This is the \TeX{} side interface for |luamplib.showlog|.
%
% \subsubsection{\cs{mpliblegacybehavior\{enable\textbar disable\}}}
%   \hypertarget{mpliblegacybehavior}{}\relax
%   By default, \cs{mpliblegacybehavior\{enable\}}
%   is already declared for backward compatibility,
%   in which case \TeX\ code in
%   |verbatimtex| |...| |etex| that comes just before |beginfig()|
%   will be inserted before the
%   following \metapost figure box.  In this way,
%   each figure box can be freely moved horizontally or vertically.
%   Also, a box number can be assigned to a figure box, allowing it to be
%   reused later.\footnote{But
%     the recommended way to reuse a figure is using \cs{mplibgroup} command.
%     See \hyperlink{mplibgroupendmplibgroup}{below} \S\,1.2.
%   }
%\begin{verbatim}
%     \mplibcode
%     verbatimtex \moveright 3cm etex; beginfig(0); ... endfig;
%     verbatimtex \leavevmode etex; beginfig(1); ... endfig;
%     verbatimtex \leavevmode\lower 1ex etex; beginfig(2); ... endfig;
%     verbatimtex \endgraf\moveright 1cm etex; beginfig(3); ... endfig;
%     \endmplibcode
%\end{verbatim}
%   \textsc{n.b.} \cs{endgraf} should be used instead of \cs{par} inside
%   |verbatimtex ... etex|.
%
%   On the other hand, \TeX\ code in |verbatimtex ... etex|
%   between |beginfig()| and |endfig| will be inserted
%   after flushing out the \metapost figure.
%   As shown in the example below, |VerbatimTeX| \meta{string} is a synonym of |verbatimtex| |...| |etex|.
%\begin{verbatim}
%     \mplibcode
%       D := sqrt(2)**7;
%       beginfig(0);
%       draw fullcircle scaled D;
%       VerbatimTeX("\gdef\Dia{" & decimal D & "}");
%       endfig;
%     \endmplibcode
%     diameter: \Dia bp.
%\end{verbatim}
%
%   By contrast,
%   when \cs{mpliblegacybehavior\{disable\}} is declared, any
%   |verbatimtex| |...| |etex| will be executed, along with |btex| |...| |etex|,
%   sequentially one by one.
%   So, some \TeX\ code in |verbatimtex ... etex| will have effects on
%   following |btex| |...| |etex| codes.
%\begin{verbatim}
%     \begin{mplibcode}
%       beginfig(0);
%       draw btex ABC etex;
%       verbatimtex \bfseries etex;
%       draw btex DEF etex shifted (1cm,0); % bold face
%       draw btex GHI etex shifted (2cm,0); % bold face
%       endfig;
%     \end{mplibcode}
%\end{verbatim}
%
% \subsubsection{\cs{mplibtextextlabel\{enable\textbar disable\}}}
%   Default: |disable|.
%   \cs{mplibtextextlabel\{enable\}} enables
%   the labels typeset via |textext| instead of |infont| operator.
%   So, |label("my text",origin)| thereafter is exactly the same as
%   |label(textext "my text", origin)|.
%
%   \textsc{n.b.} In the background,
%   \pkg{luamplib} redefines |infont| operator so that the right side
%   argument (the font part) is totally ignored. Therefore the left side arguemnt (the text part)
%   will be typeset with the current \TeX\ font.
%
%   From v2.35, however, the redefinition of |infont| operator has been revised:
%   when the character code of the text argument is less than $32$~(control characters),
%   or is equal to $35$~(|#|), $36$~(|$|), \iffalse |$|\fi
%   $37$~(|%|), $38$~(|&|), $92$~(|\|), $94$~(|^|),
%   $95$~(|_|), $123$~(|{|), $125$~(|}|), $126$~(|~|) or $127$~(DEL),
%   the original |infont| operator will be used instead of |textext| operator
%   so that the font part will be honored.
%   Despite the revision, please take care of |char| operator in the text argument,
%   as this might bring unpermitted characters into \TeX.
%
% \subsubsection{\cs{mplibcodeinherit\{enable\textbar disable\}}}
%   Default: |disable|.
%   \cs{mplibcodeinherit\{enable\}} enables the inheritance
%   of variables, constants, and macros defined by previous \metapost code chunks.
%   On the contrary, \cs{mplibcodeinherit\{disable\}} will make
%   each code chunk being treated as an independent instance, never
%   affected by previous code chunks.
%
% \subsubsection{Separate \metapost instances}
%   \pkg{luamplib} v2.22 has added the support for several named \metapost instances
%   in \LaTeX{} |mplibcode| environment.
%   Plain \TeX\ users also can use this functionality.
%   The syntax for \LaTeX\ is:
%\begin{verbatim}
%   \begin{mplibcode}[instanceName]
%     % some mp code
%   \end{mplibcode}
%\end{verbatim}
%   The behavior is as follows.
%   \begin{itemize}
%   \item  All the variables and functions are shared
%     only among all the environments belonging to the same instance.
%   \item  \cs{mplibcodeinherit} only affects environments
%     with no instance name set (since if a name is set,
%     the code is intended to be reused at some point).
%   \item |btex ... etex| boxes are also shared and do not
%     require \cs{mplibglobaltextext}.
%   \item  When an instance names is set,
%     respective \cs{currentmpinstancename} is set as well.
%   \end{itemize}
%   In parellel with this functionality, we support
%   optional argument of instance name for \cs{everymplib} and
%   \cs{everyendmplib}, affecting only those |mplibcode| environments
%   of the same name.
%   Unnamed \cs{everymplib} affects not only those instances with no name,
%   but also those with name but with no corresponding \cs{everymplib}.
%   The syntax is:
%\begin{verbatim}
%     \everymplib[instanceName]{...}
%     \everyendmplib[instanceName]{...}
%\end{verbatim}
%
% \subsubsection{\cs{mplibglobaltextext\{enable\textbar disable\}}}
%   Default: |disable|.
%   Formerly, to inherit |btex| |...| |etex| boxes as well as other \metapost macros, variables and constants,
%   it was necessary to declare \cs{mplibglobaltextext\{enable\}} in advance.
%   But from v2.27, this is implicitly enabled when \cs{mplibcodeinherit}
%   is enabled.
%   This optional command still remains mostly for backward compatibility.
%\begin{verbatim}
%   \mplibcodeinherit{enable}
%   %\mplibglobaltextext{enable}
%   \everymplib{ beginfig(0);} \everyendmplib{ endfig;}
%   \mplibcode
%     label(btex $\sqrt{2}$ etex, origin);
%     draw fullcircle scaled 20;
%     picture pic; pic := currentpicture;
%   \endmplibcode
%   \mplibcode
%     currentpicture := pic scaled 2;
%   \endmplibcode
%\end{verbatim}
%
% \subsubsection{\cs{mplibverbatim\{enable\textbar disable\}}}
%   Default: |disable|.
%   Users can issue \cs{mplibverbatim\{enable\}}, after which
%   the contents of mplibcode environment will be read verbatim. As a result,
%   except for \cs{mpdim} and \cs{mpcolor} (see \hyperlink{mpdim}{below}),
%   all other \TeX\ commands outside of the
%   |btex| or |verbatimtex| |...| |etex| are not expanded and will be fed
%   literally to the \mplib library.
%
% \subsubsection{\cs{mpdim\{...\}}}
%   \hypertarget{mpdim}{}\relax
%   Besides other \TeX\ commands, \cs{mpdim} is specially allowed
%   in the mplibcode environment. This feature is inpired by \pkg{gmp} package authored by
%   Enrico Gregorio. Please refer to the manual of \pkg{gmp} package for details.
%\begin{verbatim}
%     \begin{mplibcode}
%       beginfig(1)
%       draw origin--(.6\mpdim{\linewidth},0) withpen pencircle scaled 4
%       dashed evenly scaled 4 withcolor \mpcolor{orange};
%       endfig;
%     \end{mplibcode}
%\end{verbatim}
%
% \subsubsection{\cs{mpcolor[...]\{...\}}}
%   With \cs{mpcolor} command, color names or expressions of
%   \pkg{color}, \pkg{xcolor} and \pkg{l3color} module/packages can be used in the mplibcode
%   environment (after |withcolor| operator).
%   See the example \hyperlink{mpdim}{above}.
%   The optional |[...]| denotes the option of \pkg{xcolor}'s \cs{color} command.
%   For spot colors, \pkg{l3color} (in PDF/DVI mode),
%   \pkg{colorspace}, \pkg{spotcolor}
%   (in PDF mode) and \pkg{xespotcolor} (in DVI mode) packages are supported
%   as well.
%
% \subsubsection{\cs{mpfig} \ldots\ \cs{endmpfig}}
%   Besides the |mplibcode| environment (for \LaTeX) and
%   \cs{mplibcode ...} \cs{endmplibcode} (for Plain),
%   we also provide unexpandable \TeX\ macros \cs{mpfig} |...| \cs{endmpfig} and its starred version
%   \cs{mpfig*} |...| \cs{endmpfig} to save typing toil.
%   The former is roughly the same as follows:
%\begin{verbatim}
%     \begin{mplibcode}[@mpfig]
%     beginfig(0)
%     token list declared by \everymplib[@mpfig]
%     ...
%     token list declared by \everyendmplib[@mpfig]
%     endfig;
%     \end{mplibcode}
%\end{verbatim}
%   and the starred version is roughly the same as follows:
%\begin{verbatim}
%     \begin{mplibcode}[@mpfig]
%     ...
%     \end{mplibcode}
%\end{verbatim}
%   In these macros \cs{mpliblegacybehavior\{disable\}}
%   is forcibly declared.
%   Again, as both share the same instance name, \metapost codes are inherited among them.
%   A simple example:
%\begin{verbatim}
%     \everymplib[@mpfig]{ drawoptions(withcolor .5[red,white]); }
%     \mpfig* input boxes \endmpfig
%     \mpfig
%       circleit.a(btex Box 1 etex); drawboxed(a);
%     \endmpfig
%\end{verbatim}
%   The instance name (default: |@mpfig|) can be changed by redefining
%   \cs{mpfiginstancename}, after which a new \mplib instance will start and
%   code inheritance too will begin anew.  \cs{let}\cs{mpfiginstancename}\cs{empty} will
%   prevent code inheritance if \cs{mplibcodeinherit\{true\}} is not declared.
%
% \subsubsection{About cache files}
%   To support |btex ... etex| in external |.mp| files, \pkg{luamplib}
%   inspects the content of each and every |.mp| file and makes caches
%   if nececcsary, before returning their paths to \LuaTeX's \mplib library.
%   This could waste the compilation time, as most |.mp| files
%   do not contain |btex ... etex| commands.  So \pkg{luamplib} provides
%   macros as follows, so that users can give instructions about files
%   that do not require this functionality.
%   \begin{itemize}
%   \item \cs{mplibmakenocache}|{|\meta{filename}|[,|\meta{filename}|,...]}|
%   \item \cs{mplibcancelnocache}|{|\meta{filename}|[,|\meta{filename}|,...]}|
%   \end{itemize}
%   where \meta{filename} is a filename excluding |.mp| extension.
%   Note that |.mp| files under |$TEXMFMAIN/metapost/base| and
%   |$TEXMFMAIN/metapost/context/base| are already registered by default.
%
%   By default, cache files will be stored in |$TEXMFVAR/luamplib_cache| or,
%   if it's not available (mostly not writable),
%   in the directory where output files are saved:
%   to be specific, |$TEXMF_OUTPUT_DIRECTORY/luamplib_cache|,
%   |./luamplib_cache|, |$TEXMFOUTPUT/luamplib_cache|, and |.|, in this order.
%   |$TEXMF_OUTPUT_DIRECTORY| is normally the value of |--output-directory|
%   command-line option.
%
%   Users can change this behavior by the command
%   \cs{mplibcachedir}\marg{directory path}, where tilde (|~|) is interpreted
%   as the user's home directory (on a windows machine as well).
%   As backslashes (\cs{}) should be escaped by users, it would be easier to use
%   slashes (|/|) instead.
%
% \subsubsection{About figure box metric}
%   Notice that, after each figure is processed, the macro \cs{MPwidth} stores
%   the width value of the latest figure; \cs{MPheight}, the height value.
%   Incidentally, also note that \cs{MPllx}, \cs{MPlly}, \cs{MPurx}, and
%   \cs{MPury} store the bounding box information of the latest figure
%   without the unit |bp|.
%
% \subsubsection{luamplib.cfg}
%   At the end of package loading, \pkg{luamplib} searches
%   |luamplib.cfg| and, if found, reads the file in automatically.
%   Frequently used settings such as \cs{everymplib}, \cs{mplibforcehmode}
%   or \cs{mplibcodeinherit} are suitable for going into this file.
%
% \subsubsection{Tagged PDF}
%   \hypertarget{taggedPDF}{}\relax
% When \pkg{tagpdf} package is loaded and activated, |mplibcode| environment accepts additional options for tagged PDF\@.
% The code related to this functionality is currently in experimental stage, not guaranteeing backward compatibility.
% Available optional keys are similar to those of the \LaTeX's |picture| environment
% (|texdoc| |latex-lab-graphic|).
% \begin{description}
% \item[\texttt{alt=}\meta{text}] starts a |Figure| tag by default and
%   sets an alternate text of the figure from the \meta{text}.
%   BBox info will be added automatically to the PDF\@.
%   This key is needed for ordinary \metapost figures, for which,
%   if no |alt| text is given, a default text will be used with a warning issued.
%   You can change the alternate text within \metapost code as well:
%   |VerbatimTeX| |"\mplibalttext|\marg{text}|";|
% \item[\texttt{actualtext=}\meta{text}] starts a |Span| tag implicitly and sets a
%   replacement text (a.k.a. actual text) from the \meta{text}.\footnote{\relax
%     It is not recommended to personally redefine \cs{prependtomplibbox}.
%     Apart from using \cs{mplibforcehmode} or \cs{mplibnoforcehmode},
%     the redefinition might be incompatible with |actualtext| key.
%     See \hyperlink{mplibforcehmode}{above} on these commands. }
%   BBox info will not be added.
%   This key is intended for figures which can be represented by a character or
%   a small sequence of characters.
%   You can change the actual text within \metapost code as well:
%   |VerbatimTeX| |"\mplibactualtext|\marg{text}|";|
% \item[\texttt{artifact}] starts an |Artifact| MC (marked content).
%   BBox info will not be added.
%   This key is intended for decorative figures which have no semantic meaning.
% \item[\texttt{text}] starts an |Artifact| MC but enables tagging on tex-text boxes
%   (such as |btex| |...| |etex|, excluding pictures made by |infont| operator).\footnote{\relax
%     The key |text| also shares the limitation mentioned in the previous footnote.}
%   BBox info will not be added.
%   This key is intended for figures whose meaning is the sequence of texts in the
%   tex-text boxes in the order they are drawn in the figure.
%   Inside |text|-mode figures, reusing tex-text boxes is strongly discouraged.
% \item[\texttt{off}] Given this key, nothing will be tagged by \pkg{luamplib}.
% \item[\texttt{tag=}\meta{name}] You can choose a tag name, default value being |Figure|.\relax
%   \footnote{The option |tag=false|, however, is a synonym of the |off| key.}
%   For instance, you can set `|tag=Formula,| |alt=|\meta{text}' to get
%   a |Formula| element with its alternate text.\footnote{Beware that this bypasses
%     \LaTeX's regular math formula tagging, for which the |text| key is needed.}
% \item[\texttt{adjust-BBox=}\meta{dimens}] You can correct the BBox attribute of the figure
%   by space-separated four dimensional values, which will be added to the
%   automatically calculated BBox values.
%   To draw the bounding box for checking with half-transparent red color, you can add
%   |debug=|\allowbreak|BBox| to the argument of \cs{DocumentMetadata} command.
% \item[\texttt{tagging-setup=}\meta{key-val list}]
%   This key accepts as its value the list of key-value options mentioned so far.
% \end{description}
%
% The above options are provided also for \cs{mpfig} and \cs{usemplibgroup} (see \hyperlink{usemplibgroup}{below} \S\,1.2) commands.
%\begin{verbatim}
%     \begin{mplibcode}[myInstanceName, alt=drawing of a circle]
%     ...
%     \end{mplibcode}
%
%     \mpfig[alt=drawing of a square box]
%     ...
%     \endmpfig
%
%     \usemplibgroup[alt=drawing of a triangle]{...}
%
%     \mppattern{...}           % see below
%       \mpfig[off]             % do not tag this figure
%       ...
%       \endmpfig
%     \endmppattern
%\end{verbatim}
% As for the instance name of |mplibcode| environment,
% |instance=|\meta{name} or |instancename=|\allowbreak\meta{name} is also allowed
% in addition to the raw instance name as shown above.
%
%
% \subsection{\textsc{MetaPost}}
%
% \subsubsection{\texttt{mplibdimen ...}, \texttt{mplibcolor ...}}
%   These are \metapost interfaces for the \TeX\ commands
%   \cs{mpdim} and \cs{mpcolor} (see \hyperlink{mpdim}{above} \S\,1.1). For example,
%   |mplibdimen|\,|"\linewidth"| is basically the same as \cs{mpdim\{\cs{linewidth}\}}, and
%   |mplibcolor|\,|"red!50"| is basically the same as \cs{mpcolor\{red!50\}}.
%   The difference is that these \metapost operators can also be used in external |.mp| files,
%   which cannot have \TeX\ commands outside of the |btex| or |verbatimtex| |...| |etex|.
%
% \subsubsection{\texttt{mplibtexcolor ...}, \texttt{mplibrgbtexcolor ...}}
%   |mplibtexcolor|, which accepts a string argument, is a \metapost operator that converts a \TeX\ color expression
%   to a \metapost color expression, that can be used anywhere color expression is expected
%   as well as after the |withcolor| operator.
%   For instance:
%\begin{verbatim}
%     color col;
%     col := mplibtexcolor "olive!50";
%\end{verbatim}
%   But the result may vary in its color model (gray/rgb/cmyk)
%   according to the given \TeX\ color. (Spot colors are forced to
%   cmyk model, so this operator is not recommended for spot colors.)
%   Therefore the example shown above would raise a \metapost error:
%   |cmykcolor col;| should have been declared.
%   By contrast, |mplibrgbtexcolor| \meta{string} always returns rgb model expressions.
%
% \subsubsection{\texttt{mplibgraphictext ...}}
%   |mplibgraphictext| is a \metapost operator, the effect of which is similar to that of
%   \ConTeXt's |graphictext| or our own |mpliboutlinetext|
%   (see \hyperlink{mpliboutlinetext}{below}).
%   However the syntax is somewhat different.
%\begin{verbatim}
%     mplibgraphictext "Funny"
%       fakebold 2.3                        % fontspec option
%       drawcolor .7blue fillcolor "red!50" % color expressions
%\end{verbatim}
%   |fakebold|, |drawcolor| and |fillcolor| are optional;
%   default values are |2|, |"black"| and |"white"| respectively.
%   When the color expressions are given in string type, they are regarded as
%   \pkg{color}, \pkg{xcolor} or \pkg{l3color}'s expressions.
%   All from |mplibgraphictext| to the end of sentence will compose an
%   anonymous |picture|, which can be drawn or assigned to a variable.
%   Incidentally, |withdrawcolor| and |withfillcolor| are synonyms of
%   |drawcolor| and |fillcolor|, hopefully to be compatible with |graphictext|.
%
%   \textsc{n.b.}
%   In some cases, |mplibgraphictext| will produce better results than \ConTeXt\
%   or even than our own |mpliboutlinetext|,
%   especially when processing complicated \TeX\ code
%   such as the vertical writing in Chinese or Japanese.
%   However, because the implementation is quite different from others,
%   there are some limitations such that you can't
%   apply shading (gradient colors) to the text with |withshademethod| from
%   \emph{metafun}.\footnote{But
%   this limitation is now lifted by the introduction of |withshadingmethod|.
%   See \hyperlink{luamplibshading}{below}.}
%   Again,
%   in DVI mode, \pkg{unicode-math} package is needed for math formula,
%   as we cannot embolden type1 fonts in DVI mode.
%
%   \subsubsection{\texttt{mplibglyph ... of ...}}
%   From v2.30, we provide a new \metapost operator |mplibglyph|, which returns a \metapost picture
%   containing outline paths of a glyph in opentype, truetype or type1 fonts.
%   When a type1 font is specified, \metapost primitive |glyph| will be called.
%\begin{verbatim}
%      mplibglyph 50  of \fontid\font          % slot 50 of current font
%      mplibglyph "Q" of "TU/TeXGyrePagella(0)/m/n/10"    % font csname
%      mplibglyph "Q" of "texgyrepagella-regular.otf"     % raw filename
%      mplibglyph "Q" of "Times.ttc(2)"                   % subfont number
%      mplibglyph "Q" of "SourceHanSansK-VF.otf[Regular]" % instance name
%\end{verbatim}
%   Both arguments before and after of ``|of|'' can be either a number or a string.
%   Number arguments are regarded as a glyph slot (GID) and a font id number, repectively.
%   String argument at the left side is regarded as a glyph name in the font or a unicode character.
%   String argument at the right side is regarded as a \TeX\ font csname (without backslash) or
%   the raw filename of a font. When it is a font filename, a number within parentheses
%   after the filename denotes a
%   subfont number (starting from zero) of a TTC font; a string within brackets denotes
%   an instance name of a variable font.
%
%   \subsubsection{\texttt{mplibdrawglyph ...}}
%   The picture returned by |mplibglyph| will be quite similar to the result of |glyph| primitive in its structure.
%   So, \metapost's |draw| command will fill the inner path of the picture with the background color.
%   In contrast, |mplibdrawglyph| \meta{picture} command fills the paths according to the nonzero winding
%   number rule. As a result, for instance, the area surrounded by inner path of ``O''
%   will remain transparent.
%
%   \leavevmode\llap{\textcolor{red}{☞}\kern1.2\parindent}\relax
%   To apply the nonzero winding number rule to a picture containing paths,
%   \pkg{luamplib} appends |withpostscript| |"collect"|
%   to the paths except the last one in the picture.
%   If you want the even-odd rule instead, you can, with \emph{plain} format as well,
%   additionally declare |withpostscript| |"evenodd"| to the last path in the picture.
%
% \subsubsection{\texttt{mpliboutlinetext (...)}}
%   \hypertarget{mpliboutlinetext}{}\relax
%   From v2.31, a new \metapost operator |mpliboutlinetext| is available, which mimicks
%   \emph{metafun}'s |outlinetext|. So the syntax is the same: see the \emph{metafun}
%   manual \S\,8.7 (|texdoc| |metafun|). A simple example:
%\begin{verbatim}
%     draw mpliboutlinetext.b ("$\sqrt{2+\alpha}$")
%         (withcolor \mpcolor{red!50})
%         (withpen pencircle scaled .2 withcolor red)
%         scaled 2 ;
%\end{verbatim}
%   After the process, |mpliboutlinepic[]|
%   and |mpliboutlinenum| will be preserved as global variables;
%   |mpliboutlinepic[1]| |...| |mpliboutlinepic[mpliboutlinenum]|
%   will be an array of images each of which containing a glyph or a rule.
%
%   \textsc{n.b.} As Unicode grapheme cluster is not considered in the array, a unit that must be
%   a single cluster might be separated apart.
%
% \subsubsection{\cs{mppattern\{...\} ...} \cs{endmppattern}, \texttt{... withpattern ...},
%     \texttt{withmppattern ...}}
%   \hypertarget{mppattern}{}\relax
%   \TeX\ macros
%   \cs{mppattern}\marg{name} |...| \cs{endmppattern} define a tiling pattern
%   associated with the \meta{name}.
%   \metapost operator |withpattern|, the syntax being
%   \meta{path}\,\textbar\,\meta{textual picture} |withpattern| \meta{string},
%   will return a \metapost picture which fills
%   the given path or text with a tiling pattern of the \meta{name}
%   by replicating it horizontally and vertically.
%   The \emph{textual picture} here means any text typeset by \TeX, mostly the result
%   of the |btex| command (though technically this is not a true textual picture)
%   or the |infont| operator.
%
%   |withmppattern| \meta{string} is a command virtually the same as |withpattern|,
%   but the former does not force the result of \metapost picture. So users can
%   use any drawing command suitable, such as |fill| or |filldraw| as well as |draw|.
%
%   An example:
%\begin{verbatim}
%    \mppattern{mypatt}               % or \begin{mppattern}{mypatt}
%      [                              % options: see below
%        xstep = 10,
%        ystep = 12,
%        matrix = {0, 1, -1, 0},      % or "0 1 -1 0"
%      ]
%      \mpfig                         % or any other TeX code,
%        draw (origin--(1,1))
%          scaled 10
%          withcolor 1/3[blue,white]
%          ;
%        draw (up--right)
%          scaled 10
%          withcolor 1/3[red,white]
%          ;
%      \endmpfig
%    \endmppattern                    % or \end{mppattern}
%
%    \mpfig
%      draw fullcircle scaled 90
%        withpostscript "collect"
%        ;
%      filldraw fullcircle scaled 200
%        withmppattern "mypatt"
%        withpen pencircle scaled 1
%        withcolor \mpcolor{red!50!blue!50}
%        withpostscript "evenodd"
%        ;
%    \endmpfig
%\end{verbatim}
%
%   The available options are listed in Table~\ref{tab:mppatternoptions}.
%   \begin{table}
%   \centering
%   \caption{options for \cs{mppattern}}\label{tab:mppatternoptions}
%   \begin{tabular}{lll}\hline
%     Key   & Value Type & Explanation\\\hline
%     |xstep|    &\emph{number} & horizontal spacing between pattern cells\\
%     |ystep|    &\emph{number} & vertical spacing between pattern cells\\
%     |xshift|   &\emph{number} & horizontal shifting of pattern cells\\
%     |yshift|   &\emph{number} & vertical shifting of pattern cells\\
%     |bbox|     &\emph{table} or \emph{string} & |llx|, |lly|, |urx|, |ury| values\kern1pt*\\
%     |matrix|   &\emph{table} or \emph{string} & |xx|, |yx|, |xy|, |yy| values\kern1pt* or MP transform code\\
%     |resources|&\emph{string} & PDF resources if needed\\
%     |colored| or |coloured| &\emph{boolean}& |false| for uncolored pattern. default: |true|\\\hline
%                &                & \small *\,in string type, numbers are separated by spaces\\
%   \end{tabular}
%   \end{table}
%
%   For the sake of convenience, the width and height values of tiling patterns will be written down
%   into the log file. (depth is always zero.) Users can refer to them for option setting.
%
%   As for |matrix| option, \metapost code such as `|rotated 30 slanted .2|' is allowed as well
%   as string or table of four numbers. You can also set |xshift| and |yshift|
%   values by using `|shifted|' operator. But when |xshift| or |yshift| option is explicitly
%   given, they have precedence over the effect of `|shifted|' operator.
%
%   When you use special effects such as transparency in a pattern,
%   |resources| option is needed: for instance, |resources="/ExtGState 1 0 R"|.
%   However, as \pkg{luamplib} automatically includes the resources of the current page, this option
%   is not needed in most cases.
%
%   Option |colored=false| (|coloured| is a synonym of |colored|) will generate an uncolored pattern which shall have no color at all.
%   Uncolored pattern will be painted later by the color of a \metapost object.
%   An example:
%\begin{verbatim}
%      \begin{mppattern}{pattnocolor}
%        [
%          colored = false,
%          matrix = "slanted .3 rotated 30",
%        ]
%        \tiny\TeX
%      \end{mppattern}
%
%      \begin{mplibcode}
%        beginfig(1)
%        picture tex;
%        tex = mpliboutlinetext.p ("\bfseries \TeX");
%        for i=1 upto mpliboutlinenum:
%          j:=0;
%          for item within mpliboutlinepic[i]:
%            j:=j+1;
%            filldraw pathpart item scaled 10
%            if j < length mpliboutlinepic[i]:
%                withpostscript "collect"
%            else:
%                withmppattern "pattnocolor"
%                withpen pencircle scaled 1/2
%                withcolor (i/4)[red,blue]          % paints the pattern
%            fi;
%          endfor
%        endfor
%        endfig;
%      \end{mplibcode}
%\end{verbatim}
% A much simpler and efficient way to obtain a similar result
% (without colorful characters in this example)
% is to give a \emph{textual picture} as the operand of |withpattern| or |withmppattern|:
%\begin{verbatim}
%      \begin{mplibcode}
%        beginfig(2)
%        draw mplibgraphictext "\bfseries\TeX"
%               fakebold 1
%               fillcolor 1/3[red,blue]            % paints the pattern
%               drawcolor 2/3[red,blue]
%               scaled 10
%               withmppattern "pattnocolor" ;
%        endfig;
%      \end{mplibcode}
%\end{verbatim}
%
% \subsubsection{\texttt{... withfademethod ...}}
%   This is a \metapost operator which makes the color of an object gradiently transparent.
%   The syntax is \meta{path}\,\textbar\,\meta{picture} |withfademethod| \meta{string},
%   the latter being either |"linear"| or |"circular"|.
%   Though it is similar to the |withshademethod| from \emph{metafun},
%   the differences are: (1) the operand of |withfademethod| can be a picture as well as a path;
%   (2) you cannot make gradient colors, but can only make gradient opacity.
%
%   Related macros to control optional values are:
%   \begin{description}
%     \let\bfseries\relax
%   \item [|withfadeopacity (|\emph{number, number}|)|]
%     sets the starting opacity and the ending opacity, default value being |(1,0)|.
%     `|1|' denotes full color; `|0|' full transparency.
%   \item [|withfadevector (|\emph{pair, pair}|)|]
%     sets the starting and ending points. Default value in the linear mode is
%     |(llcorner p, lrcorner p)|, where |p| is the operand,
%     meaning that fading starts from the left edge and ends at the right edge.
%     Default value in the circular mode is |(center p, center p)|, which means
%     centers of both starting and ending circles are the center of the bounding box.
%   \item [|withfadecenter|] is a synonym of |withfadevector|.
%   \item [|withfaderadius (|\emph{number, number}|)|]
%     sets the radii of starting and ending circles. This is no-op in the linear mode.
%     Default value is |(0, abs(center p - urcorner p))|, meaning that fading starts from the
%     center and ends at the four corners of the bounding box.
%   \item [|withfadebbox (|\emph{pair, pair}|)|]
%     sets the bounding box of the fading area, default value being |(llcorner p, urcorner p)|.
%     Though this option is not needed in most cases, there could be cases when users want to
%     explicitly control the bounding box.
%     Particularly, see the description \hyperlink{withgroupbbox}{below}
%     on the analogous macro |withgroupbbox|.
%   \end{description}
%   An example:
%\begin{verbatim}
%     \mpfig
%       picture mill;
%       mill = btex \includegraphics[width=100bp]{mill} etex;
%       draw mill
%         withfademethod  "circular"
%         withfadecenter  (center mill, center mill)
%         withfaderadius  (20, 50)
%         withfadeopacity (1, 0)
%         ;
%     \endmpfig
%\end{verbatim}
%
% \subsubsection{\texttt{... asgroup ...}}
% \hypertarget{transparencygroup}{}\relax
% As said \hyperlink{metafunformat}{before},
% transparency group is available with \emph{plain} as well as \emph{metafun} format.
% The syntax is exactly the same:
% \meta{picture}\,\textbar\,\meta{path} |asgroup|
% |""|\,\textbar\,|"isolated"|\,\textbar\,|"knockout"|\,\textbar\ |"isolated,knockout"|,
% which will return a \metapost picture.
% It is called \emph{Transparency Group} because the objects contained in the group are composited
% to produce a single object, so that outer transparency effect, if any,
% will be applied to the group as a whole, not to the individual objects cumulatively.
%
% The additional feature provided by \pkg{luamplib} is that
% you can reuse the group as many times as you want
% in the \TeX{} code or in other \metapost code chunks,
% with infinitesimal increase in the size of PDF file.
% For this functionality we provide \TeX{} and \metapost macros as follows:
% \begin{description}
%   \let\bfseries\relax
% \item[|withgroupname| \meta{string}] associates a transparency group with the given name.
%   When this is not appended to the sentence with |asgroup| operator,
%   the default group name `|lastmplibgroup|' will be used.
%
% \item[\cs{usemplibgroup}\marg{name}]
%   \hypertarget{usemplibgroup}{}\relax
%   is a \TeX\ command to reuse a transparency group of the name
%   once used.
%   Note that the position of the group will be origin-based:
%   in other words, lower-left corner of the group will be shifted to the origin.
%
% \item[|usemplibgroup| \meta{string}] is a \metapost command which will add
%   a transparency group of the name to the |currentpicture|.
%   Contrary to the \TeX\ command just mentioned,
%   the position of the group is the same as the original transparency group.
%
% \item[|withgroupbbox (|\emph{pair}|,|\emph{pair}|)|]
%   \hypertarget{withgroupbbox}{}\relax
%   sets the bounding box of the transparency group,
%   default value being |(llcorner p, urcorner p)|.
%   This option might be needed especially when you draw with a thick pen
%   a path that touches the boundary;
%   you would probably want to append to the sentence
%   `|withgroupbbox| |(bot| |lft| |llcorner| |p,| |top| |rt| |urcorner| |p)|',
%   supposing that the pen was selected by the |pickup| command.
%
% \end{description}
% An example showing the difference between the \TeX\ and \metapost commands:
%\begin{verbatim}
%   \mpfig
%     draw image(
%       fill fullcircle scaled 100 shifted 25right withcolor blue;
%       fill fullcircle scaled 100 withcolor red ;
%     ) asgroup ""
%       withgroupname "mygroup";
%     draw (left--right) scaled 10;
%     draw (up--down) scaled 10;
%   \endmpfig
%
%   \noindent
%   \clap{\vrule width 20pt height .25pt depth .25pt}%
%   \clap{\vrule width .5pt height 10pt depth 10pt}%
%   \usemplibgroup{mygroup}
%
%   \mpfig
%     usemplibgroup "mygroup" rotated 15
%       withtransparency (1, 0.5) ;
%     draw (left--right) scaled 10;
%     draw (up--down) scaled 10;
%   \endmpfig
%\end{verbatim}
%
% Also note that normally the reused transparency groups are not affected by outer color commands.
% However, if you have made the original transparency group using |withoutcolor| command,
% colors will have effects on the uncolored objects in the group.
%
% \subsubsection{\cs{mplibgroup\{...\} ...} \cs{endmplibgroup}}
%   \hypertarget{mplibgroupendmplibgroup}{}\relax
% These \TeX\ macros are described here in this subsection, as they are
% deeply related to the |asgroup| operator.
% Users can define a transparency group or a normal \emph{form XObject}
% with these macros from \TeX\ side.
% The syntax is similar to the \cs{mppattern} command (see \hyperlink{mppattern}{above}).
% An example:
%\begin{verbatim}
%   \mplibgroup{mygrx}                  % or \begin{mplibgroup}{mygrx}
%     [                                 % options: see below
%       asgroup="",
%     ]
%     \mpfig                            % or any other TeX code
%       pickup pencircle scaled 10;
%       draw (left--right) scaled 30 rotated 45 ;
%       draw (left--right) scaled 30 rotated -45 ;
%     \endmpfig
%   \endmplibgroup                      % or \end{mplibgroup}
%
%   \usemplibgroup{mygrx}
%
%   \mpfig
%     usemplibgroup "mygrx" scaled 1.5
%       withtransparency (1, 0.5) ;
%   \endmpfig
%\end{verbatim}
%
% Availabe options, much fewer than those for \cs{mppattern},
% are listed in Table~\ref{tab:mplibgroupoptions}.
% Again, the width/height/depth values of the mplibgroup will be written down into the log file.
% \begin{table}
% \centering
% \caption{options for \cs{mplibgroup}}\label{tab:mplibgroupoptions}
% \begin{tabular}{lll}
% \hline
%   Key         & Value Type    & Explanation\\
% \hline
%   |asgroup|   &\emph{string}  & |""|, |"isolated"|, |"knockout"|, or |"isolated,knockout"|\\
%   |bbox|      &\emph{table} or \emph{string} & |llx|, |lly|, |urx|, |ury| values\kern1pt*\\
%   |matrix|    &\emph{table} or \emph{string} & |xx|, |yx|, |xy|, |yy| values\kern1pt*
%     or MP transform code\\
%   |resources| &\emph{string}  & PDF resources if needed\\
% \hline
%               &               & \small *\,in string type, numbers are separated by spaces\\
% \end{tabular}
% \end{table}
%
% When |asgroup| option, including empty string, is not given,
% a normal form XObject will be generated rather than a transparency group.
% Thus the individual objects, not the XObject as a whole, will be affected
% by outer transparency command.
%
% As shown in the example, you can reuse the mplibgroup
% using the \TeX\ command \cs{usemplibgroup} or
% the \metapost command |usemplibgroup|.
% The behavior of these commands is the same as that described \hyperlink{usemplibgroup}{above},
% excepting that the mplibgroup made by \TeX\ code (not by \metapost code) respects original height and depth.
%
% \subsubsection{\texttt{... withtransparency ...}}
%   \hypertarget{luamplibtransparency}{}\relax
%   |withtransparency(|\emph{number}\,\textbar\,\emph{string}, \emph{number}|)|
%   is provided for \emph{plain} format as well.
%   The first argument accepts a number or a name of
%   alternative transparency methods (see |texdoc| |metafun| \S\,8.2 Figure~8.1).
%   The second argument accepts a number denoting opacity.
%\begin{verbatim}
%     fill fullcircle scaled 10
%         withcolor red
%         withtransparency (1, 0.5)           % or ("normal", 0.5)
%         ;
%\end{verbatim}
%
% \subsubsection{\texttt{... withshadingmethod ...}}
%   \hypertarget{luamplibshading}{}\relax
%   The syntax is exactly the same as \emph{metafun}'s new shading method (|texdoc metafun| \S\,8.3.3), except that
%   the `|shade|' contained in each and every macro name has changed to
%   `|shading|' in \pkg{luamplib}: for instance, while |withshademethod| is
%   a macro name which only works with \emph{metafun} format,
%   the equivalent provided by \pkg{luamplib}, |withshadingmethod|, works with \emph{plain} as well.
%   Other differences to the \emph{metafun}'s and some cautions are:
%   \begin{itemize}
%   \item \emph{textual pictures} (pictures made by |btex| |...| |etex|, |textext|,
%     |maketext|, |mplibgraphictext|, |TEX|, |infont|, etc)
%     as well as paths can have shading effect.
%\begin{verbatim}
%     draw btex \bfseries\TeX etex scaled 10
%         withshadingmethod "linear"
%         withshadingcolors (red,blue) ;
%\end{verbatim}
%   \item When you give shading effect to a picture made by `|infont|' operator,
%     the result of |withshadingvector| will be the same as that of |withshadingdirection|,
%     as \pkg{luamplib} considers only the bounding box of the picture.
%   \end{itemize}
% Macros provided by \pkg{luamplib} are:
% \begin{description}
%   \let\bfseries\relax
%   \item[\meta{path}\,\textbar\,\meta{textual picture} \texttt{withshadingmethod} \meta{string}]
%     where \meta{string} shall be |"linear"| or |"circular"|.
%     This is the only `must' item to get shading effect; all the macros below are optional.
%   \item[\texttt{withshadingvector} \meta{pair}]
%     Starting and ending points (as time value) on the path.
%   \item[\texttt{withshadingdirection} \meta{pair}]
%     Starting and ending points (as time value) on the bounding box.
%     Default value: |(0,2)|
%   \item[\texttt{withshadingorigin} \meta{pair}]
%     The center of starting and ending circles.
%     Default value: |center p|
%   \item[\texttt{withshadingradius} \meta{pair}]
%     Radii of starting and ending circles. This is no-op in linear mode.
%     Default value: |(0, abs(center p - urcorner p))|
%   \item[\texttt{withshadingfactor} \meta{number}]
%     Multiplier of the radii. This is no-op in linear mode.
%     Default value: |1.2|
%   \item[\texttt{withshadingcenter} \meta{pair}]
%     Values for shifting starting center.
%     For instance,
%     |(0,0)| means that the center of starting circle is |center p|;
%     |(1,1)| means |urcorner p|.
%   \item[\texttt{withshadingtransform} \meta{string}]
%     where \meta{string} shall be |"yes"| (respect transform) or |"no"| (ignore transform).
%     Default value: |"no"| for pictures made by |infont| operator; |"yes"| for all other cases.
%   \item[\texttt{withshadingdomain} \meta{pair}]
%     Limiting values of parametric variable
%     that varies on the axis of color gradient.
%     Default value: |(0,1)|
%   \item[\texttt{withshadingstep (...)}]
%     for combined shading of more than two colors.
%   \item[\texttt{withshadingfraction} \meta{number}]
%     Fractional number of each shading step.
%     Only meaningful with |withshadingstep|.
%   \item[\texttt{withshadingcolors (}\emph{color expr}, \emph{color expr}\texttt{)}]
%     Starting and ending colors.
%     Default value: |(white,black)|
% \end{description}
%
% \subsubsection{\texttt{mpliblength ...}, \texttt{mplibuclength ...}}
%   |mpliblength| \meta{string} returns the number of unicode characters in the string.
%   This is a unicode-aware version equivalent to the \metapost primitive |length|, but
%   accepts only a string-type argument.
%   For instance, |mpliblength| |"abçdéf"| returns |6|, not |8|.
%
%   On the other hand,
%   |mplibuclength| \meta{string} returns the number of unicode grapheme clusters in the string.
%   For instance, |mplibuclength| |"Äpfel"|,
%   where |Ä| is encoded using two codepoints (U+0041 and U+0308),
%   returns |5|, not |6| or |7|.
%   This operator requires \pkg{lua-uni-algos} package.
%
% \subsubsection{\texttt{mplibsubstring ... of ...}, \texttt{mplibucsubstring ... of ...}}
%   |mplibsubstring| \meta{pair} |of| \meta{string} is a unicode-aware version equivalent to the
%   \metapost's |substring ... of ...| primitive. The syntax is the same as the latter,
%   but the string is indexed by unicode characters.
%   For instance, |mplibsubstring| |(2,5)| |of| |"abçdéf"| returns |"çdé"|,
%   and |mplibsubstring| |(5,2)| |of| |"abçdéf"| returns |"édç"|.
%
%   On the other hand,
%   |mplibucsubstring| \meta{pair} |of| \meta{string} returns the part of the string indexed
%   by unicode grapheme clusters.
%   For instance, |mplibucsubstring| |(0,1)| |of| |"Äpfel"|,
%   where |Ä| is encoded using two codepoints (U+0041 and U+0308),
%   returns |"Ä"|, not |"A"|.
%   This operator requires \pkg{lua-uni-algos} package.
%
% \subsection{Lua}
%
% \subsubsection{\texttt{runscript ...}}
%   Using the primitive |runscript| \meta{string}, you can run a Lua code chunk from \metapost side
%   and get some \metapost code returned by Lua if you want.
%   As the functionality is provided by the \mplib library itself,
%   \pkg{luamplib} does not have much to say about it.
%
%   One thing is worth mentioning, however:
%   if you return a Lua \emph{table} to the \metapost process,
%   it is automatically converted to a relevant \metapost value type
%   such as pair, color, cmykcolor or transform.
%   So users can save some extra toil of converting a table to a string, though it's not a big deal.
%   For instance, |runscript|\,|"return {1,0,0}"| will give you the \metapost color expression |(1,0,0)|
%   automatically.
%
% \subsubsection{Lua table \texttt{luamplib.instances}}
%   Users can access the Lua table containing \mplib instances, |luamplib.instances|,
%   through which \metapost variables are also easily accessible from Lua side,
%   as documented in Lua\TeX{} manual \S\,11.2.8.4 (|texdoc luatex|).
%   The following will print |false|, |3.0|, |MetaPost| and
%   the knots and the cyclicity of the path |unitsquare|, consecutively.
%\begin{verbatim}
%   \begin{mplibcode}[instance1]
%     boolean b; b = 1 > 2;
%     numeric n; n = 3;
%     string s; s = "MetaPost";
%     path p; p = unitsquare;
%   \end{mplibcode}
%
%   \directlua{
%     local instance1 = luamplib.instances.instance1
%     print( instance1:get_boolean "b" )
%     print( instance1:get_number  "n" )
%     print( instance1:get_string  "s" )
%     local t = instance1:get_path "p"
%     for k,v in pairs(t) do
%       print(k, type(v)=='table' and table.concat(v,' ') or v)
%     end
%   }
%\end{verbatim}
%
% \subsubsection{Lua function \texttt{luamplib.process\_mplibcode}}
%   Users can execute a \metapost code chunk from Lua side by using this function:
%\begin{verbatim}
%     luamplib.process_mplibcode (<string> metapost code, <string> instance name)
%\end{verbatim}
%   The second argument cannot be absent, but can be an empty string (|""|) which
%   means that it has no instance name.
%
%   Some other elements in the |luamplib| namespace, listed in Table~\ref{tab:elementsinluamplib},
%   can have effects on the process of |process_mplibcode|.
%   \begin{table}
%     \centering
%     \caption{elements in \texttt{luamplib} table (partial)}\label{tab:elementsinluamplib}
%   \begin{tabular}{lll}\hline
%     Key & Type & Related \TeX\ macro \\\hline
%     |codeinherit|	& \emph{boolean} & \cs{mplibcodeinherit}\\
%     |everyendmplib|	& \emph{table} & \cs{everyendmplib}\\
%     |everymplib| & \emph{table} & \cs{everymplib}\\
%     |getcachedir|	& \emph{function} (\meta{string}) & \cs{mplibcachedir}\\
%     |globaltextext|	& \emph{boolean} & \cs{mplibglobaltextext}\\
%     |legacyverbatimtex| & \emph{boolean} & \cs{mpliblegacybehavior}\\
%     |noneedtoreplace|	& \emph{table} & \cs{mplibmakenocache}\\
%     |numbersystem|	& \emph{string} & \cs{mplibnumbersystem}\\
%     |setformat|	& \emph{function} (\meta{string}) & \cs{mplibsetformat}\\
%     |showlog|	& \emph{boolean} & \cs{mplibshowlog}\\
%     |textextlabel| & \emph{boolean} & \cs{mplibtextextlabel}\\
%     |verbatiminput| & \emph{boolean} & \cs{mplibverbatim}\\\hline
%   \end{tabular}
%   \end{table}
%
%
%    \section{Implementation}
%
%    \subsection{Lua module}
%
% \iffalse
%<*lua>
% \fi
%
%    \begin{macrocode}

luatexbase.provides_module {
  name          = "luamplib",
  version       = "2.37.2",
  date          = "2025/03/20",
  description   = "Lua package to typeset Metapost with LuaTeX's MPLib.",
}

%    \end{macrocode}
%
%    Use the |luamplib| namespace, since |mplib| is for the \metapost library
%    itself. \ConTeXt{} uses |metapost|.
%    \begin{macrocode}
luamplib          = luamplib or { }
local luamplib    = luamplib

local format, abs = string.format, math.abs

%    \end{macrocode}
%
%    Use our own function for warn/info/err.
%    \begin{macrocode}
local function termorlog (target, text, kind)
  if text then
    local mod, write, append = "luamplib", texio.write_nl, texio.write
    kind = kind
        or target == "term" and "Warning (more info in the log)"
        or target == "log" and "Info"
        or target == "term and log" and "Warning"
        or "Error"
    target = kind == "Error" and "term and log" or target
    local t = text:explode"\n+"
    write(target, format("Module %s %s:", mod, kind))
    if #t == 1 then
      append(target, format(" %s", t[1]))
    else
      for _,line in ipairs(t) do
        write(target, line)
      end
      write(target, format("(%s)     ", mod))
    end
    append(target, format(" on input line %s", tex.inputlineno))
    write(target, "")
    if kind == "Error" then error() end
  end
end
local function warn (...) -- beware '%' symbol
  termorlog("term and log", select("#",...) > 1 and format(...) or ...)
end
local function info (...)
  termorlog("log", select("#",...) > 1 and format(...) or ...)
end
local function err (...)
  termorlog("error", select("#",...) > 1 and format(...) or ...)
end

luamplib.showlog  = luamplib.showlog or false

%    \end{macrocode}
%
%    This module is a stripped down version of libraries that are used by
%    \ConTeXt. Provide a few ``shortcuts'' expected by the code.
%    \begin{macrocode}
local tableconcat = table.concat
local tableinsert = table.insert
local tableunpack = table.unpack
local texsprint   = tex.sprint
local texgettoks  = tex.gettoks
local texgetbox   = tex.getbox
local texruntoks  = tex.runtoks
if not texruntoks then
  err("Your LuaTeX version is too old. Please upgrade it to the latest")
end
local is_defined  = token.is_defined
local get_macro   = token.get_macro
local mplib = require ('mplib')
local kpse  = require ('kpse')
local lfs   = require ('lfs')
local lfsattributes = lfs.attributes
local lfsisdir      = lfs.isdir
local lfsmkdir      = lfs.mkdir
local lfstouch      = lfs.touch
local ioopen        = io.open

%    \end{macrocode}
%
%    Some helper functions, prepared for the case when |l-file| etc
%    is not loaded.
%    \begin{macrocode}
local file = file or { }
local replacesuffix = file.replacesuffix or function(filename, suffix)
  return (filename:gsub("%.[%a%d]+$","")) .. "." .. suffix
end
local is_writable = file.is_writable or function(name)
  if lfsisdir(name) then
    name = name .. "/_luam_plib_temp_file_"
    local fh = ioopen(name,"w")
    if fh then
      fh:close(); os.remove(name)
      return true
    end
  end
end
local mk_full_path = lfs.mkdirp or lfs.mkdirs or function(path)
  local full = ""
  for sub in path:gmatch("(/*[^\\/]+)") do
    full = full .. sub
    lfsmkdir(full)
  end
end

%    \end{macrocode}
%
%    |btex ... etex| in input |.mp| files will be replaced in finder.
%    Because of the limitation of \mplib regarding |make_text|,
%    we might have to make cache files modified from input files.
%    \begin{macrocode}
local luamplibtime = lfsattributes(kpse.find_file"luamplib.lua", "modification")
local currenttime = os.time()
local outputdir, cachedir
if lfstouch then
  for i,v in ipairs{'TEXMFVAR','TEXMF_OUTPUT_DIRECTORY','.','TEXMFOUTPUT'} do
    local var = i == 3 and v or kpse.var_value(v)
    if var and var ~= "" then
      for _,vv in next, var:explode(os.type == "unix" and ":" or ";") do
        local dir = format("%s/%s",vv,"luamplib_cache")
        if not lfsisdir(dir) then
          mk_full_path(dir)
        end
        if is_writable(dir) then
          outputdir = dir
          break
        end
      end
      if outputdir then break end
    end
  end
end
outputdir = outputdir or '.'
function luamplib.getcachedir(dir)
  dir = dir:gsub("##","#")
  dir = dir:gsub("^~",
    os.type == "windows" and os.getenv("UserProfile") or os.getenv("HOME"))
  if lfstouch and dir then
    if lfsisdir(dir) then
      if is_writable(dir) then
        cachedir = dir
      else
        warn("Directory '%s' is not writable!", dir)
      end
    else
      warn("Directory '%s' does not exist!", dir)
    end
  end
end
%    \end{macrocode}
%
%    Some basic \metapost files not necessary to make cache files.
%    \begin{macrocode}
local noneedtoreplace = {
  ["boxes.mp"] = true, --  ["format.mp"] = true,
  ["graph.mp"] = true, ["marith.mp"] = true, ["mfplain.mp"] = true,
  ["mpost.mp"] = true, ["plain.mp"] = true, ["rboxes.mp"] = true,
  ["sarith.mp"] = true, ["string.mp"] = true, -- ["TEX.mp"] = true,
  ["metafun.mp"] = true, ["metafun.mpiv"] = true, ["mp-abck.mpiv"] = true,
  ["mp-apos.mpiv"] = true, ["mp-asnc.mpiv"] = true, ["mp-bare.mpiv"] = true,
  ["mp-base.mpiv"] = true, ["mp-blob.mpiv"] = true, ["mp-butt.mpiv"] = true,
  ["mp-char.mpiv"] = true, ["mp-chem.mpiv"] = true, ["mp-core.mpiv"] = true,
  ["mp-crop.mpiv"] = true, ["mp-figs.mpiv"] = true, ["mp-form.mpiv"] = true,
  ["mp-func.mpiv"] = true, ["mp-grap.mpiv"] = true, ["mp-grid.mpiv"] = true,
  ["mp-grph.mpiv"] = true, ["mp-idea.mpiv"] = true, ["mp-luas.mpiv"] = true,
  ["mp-mlib.mpiv"] = true, ["mp-node.mpiv"] = true, ["mp-page.mpiv"] = true,
  ["mp-shap.mpiv"] = true, ["mp-step.mpiv"] = true, ["mp-text.mpiv"] = true,
  ["mp-tool.mpiv"] = true, ["mp-cont.mpiv"] = true,
}
luamplib.noneedtoreplace = noneedtoreplace
%    \end{macrocode}
%
%    |format.mp| is much complicated, so specially treated.
%    \begin{macrocode}
local function replaceformatmp(file,newfile,ofmodify)
  local fh = ioopen(file,"r")
  if not fh then return file end
  local data = fh:read("*all"); fh:close()
  fh = ioopen(newfile,"w")
  if not fh then return file end
  fh:write(
    "let normalinfont = infont;\n",
    "primarydef str infont name = rawtextext(str) enddef;\n",
    data,
    "vardef Fmant_(expr x) = rawtextext(decimal abs x) enddef;\n",
    "vardef Fexp_(expr x) = rawtextext(\"$^{\"&decimal x&\"}$\") enddef;\n",
    "let infont = normalinfont;\n"
  ); fh:close()
  lfstouch(newfile,currenttime,ofmodify)
  return newfile
end
%    \end{macrocode}
%
%    Replace |btex ... etex| and |verbatimtex ... etex| in input files,
%    if needed.
%    \begin{macrocode}
local name_b = "%f[%a_]"
local name_e = "%f[^%a_]"
local btex_etex = name_b.."btex"..name_e.."%s*(.-)%s*"..name_b.."etex"..name_e
local verbatimtex_etex = name_b.."verbatimtex"..name_e.."%s*(.-)%s*"..name_b.."etex"..name_e
local function replaceinputmpfile (name,file)
  local ofmodify = lfsattributes(file,"modification")
  if not ofmodify then return file end
  local newfile = name:gsub("%W","_")
  newfile = format("%s/luamplib_input_%s", cachedir or outputdir, newfile)
  if newfile and luamplibtime then
    local nf = lfsattributes(newfile)
    if nf and nf.mode == "file" and
      ofmodify == nf.modification and luamplibtime < nf.access then
      return nf.size == 0 and file or newfile
    end
  end
  if name == "format.mp" then return replaceformatmp(file,newfile,ofmodify) end
  local fh = ioopen(file,"r")
  if not fh then return file end
  local data = fh:read("*all"); fh:close()
%    \end{macrocode}
%    ``|etex|'' must be preceded by a space and followed by a space or semicolon as specified in
%    \LuaTeX\ manual, which is not the case of standalone \metapost though.
%    \begin{macrocode}
  local count,cnt = 0,0
  data, cnt = data:gsub(btex_etex, "btex %1 etex ") -- space
  count = count + cnt
  data, cnt = data:gsub(verbatimtex_etex, "verbatimtex %1 etex;") -- semicolon
  count = count + cnt
  if count == 0 then
    noneedtoreplace[name] = true
    fh = ioopen(newfile,"w");
    if fh then
      fh:close()
      lfstouch(newfile,currenttime,ofmodify)
    end
    return file
  end
  fh = ioopen(newfile,"w")
  if not fh then return file end
  fh:write(data); fh:close()
  lfstouch(newfile,currenttime,ofmodify)
  return newfile
end

%    \end{macrocode}
%
%    As the finder function for \mplib, use the |kpse| library and
%    make it behave like as if \metapost was used. And replace |.mp| files with
%    cache files if needed.
%    See also \#74, \#97.
%    \begin{macrocode}
local mpkpse
do
  local exe = 0
  while arg[exe-1] do
    exe = exe-1
  end
  mpkpse = kpse.new(arg[exe], "mpost")
end
local special_ftype = {
  pfb = "type1 fonts",
  enc = "enc files",
}
function luamplib.finder (name, mode, ftype)
  if mode == "w" then
    if name and name ~= "mpout.log" then
      kpse.record_output_file(name) -- recorder
    end
    return name
  else
    ftype = special_ftype[ftype] or ftype
    local file = mpkpse:find_file(name,ftype)
    if file then
      if lfstouch and ftype == "mp" and not noneedtoreplace[name] then
        file = replaceinputmpfile(name,file)
      end
    else
      file = mpkpse:find_file(name, name:match("%a+$"))
    end
    if file then
      kpse.record_input_file(file) -- recorder
    end
    return file
  end
end

%    \end{macrocode}
%
%    Create and load \mplib instances.
%    We do not support ancient version of \mplib any more.
%    (Don't know which version of \mplib started to support
%    |make_text| and |run_script|; let the users find it.)
%    \begin{macrocode}
local preamble = [[
  boolean mplib ; mplib := true ;
  let dump = endinput ;
  let normalfontsize = fontsize;
  input %s ;
]]
%    \end{macrocode}
%
%    \emph{plain} or \emph{metafun},
%    though we cannot support \emph{metafun} format fully.
%    \begin{macrocode}
local currentformat = "plain"
function luamplib.setformat (name)
  currentformat = name
end
%    \end{macrocode}
%
%    v2.9 has introduced the concept of ``code inherit''
%    \begin{macrocode}
luamplib.codeinherit = false
local mplibinstances = {}
luamplib.instances = mplibinstances
local has_instancename = false
local function reporterror (result, prevlog)
  if not result then
    err("no result object returned")
  else
    local t, e, l = result.term, result.error, result.log
%    \end{macrocode}
%    log has more information than term, so log first (2021/08/02)
%    \begin{macrocode}
    local log = l or t or "no-term"
    log = log:gsub("%(Please type a command or say `end'%)",""):gsub("\n+","\n")
    if result.status > 0 then
      local first = log:match"(.-\n! .-)\n! "
      if first then
        termorlog("term", first)
        termorlog("log", log, "Warning")
      else
        warn(log)
      end
      if result.status > 1 then
        err(e or "see above messages")
      end
    elseif prevlog then
      log = prevlog..log
%    \end{macrocode}
%    v2.6.1: now luamplib does not disregard |show| command,
%    even when |luamplib.showlog| is false.  Incidentally,
%    it does not raise error nor prints an info,
%    even if output has no figure.
%    \begin{macrocode}
      local show = log:match"\n>>? .+"
      if show then
        termorlog("term", show, "Info (more info in the log)")
        info(log)
      elseif luamplib.showlog and log:find"%g" then
        info(log)
      end
    end
    return log
  end
end
%    \end{macrocode}
%
%     |lualibs-os.lua| installs a randomseed. When this file is not loaded,
%     we should explicitly seed a unique integer to get random randomseed for each run.
%    \begin{macrocode}
if not math.initialseed then math.randomseed(currenttime) end
local function luamplibload (name)
  local mpx = mplib.new {
    ini_version = true,
    find_file   = luamplib.finder,
%    \end{macrocode}
%    Make use of |make_text| and |run_script|, which will co-operate
%    with \LuaTeX's |tex.runtoks| or other Lua functions. And we
%    provide |numbersystem| option since v2.4.
%    See \url{https://github.com/lualatex/luamplib/issues/21}.
%    \begin{macrocode}
    make_text   = luamplib.maketext,
    run_script  = luamplib.runscript,
    math_mode   = luamplib.numbersystem,
    job_name    = tex.jobname,
    random_seed = math.random(4095),
    extensions  = 1,
  }
%    \end{macrocode}
%    Append our own \metapost preamble to the preamble above.
%    \begin{macrocode}
  local preamble = tableconcat{
    format(preamble, replacesuffix(name,"mp")),
    luamplib.preambles.mplibcode,
    luamplib.legacyverbatimtex and luamplib.preambles.legacyverbatimtex or "",
    luamplib.textextlabel and luamplib.preambles.textextlabel or "",
  }
  local result, log
  if not mpx then
    result = { status = 99, error = "out of memory"}
  else
    result = mpx:execute(preamble)
  end
  log = reporterror(result)
  return mpx, result, log
end
%    \end{macrocode}
%
%    Here, excute each |mplibcode| data,
%    ie |\begin{mplibcode} ... \end{mplibcode}|.
%    \begin{macrocode}
local function process (data, instancename)
  local currfmt
  if instancename and instancename ~= "" then
    currfmt = instancename
    has_instancename = true
  else
    currfmt = tableconcat{
      currentformat,
      luamplib.numbersystem or "scaled",
      tostring(luamplib.textextlabel),
      tostring(luamplib.legacyverbatimtex),
    }
    has_instancename = false
  end
  local mpx = mplibinstances[currfmt]
  local standalone = not (has_instancename or luamplib.codeinherit)
  if mpx and standalone then
    mpx:finish()
  end
  local log = ""
  if standalone or not mpx then
    mpx, _, log = luamplibload(currentformat)
    mplibinstances[currfmt] = mpx
  end
  local converted, result = false, {}
  if mpx and data then
    result = mpx:execute(data)
    local log = reporterror(result, log)
    if log then
      if result.fig then
        converted = luamplib.convert(result)
      end
    end
  else
    err"Mem file unloadable. Maybe generated with a different version of mplib?"
  end
  return converted, result
end

%    \end{macrocode}
%
%    |dvipdfmx| is supported, though nobody seems to use it.
%    \begin{macrocode}
local pdfmode = tex.outputmode > 0

%    \end{macrocode}
%
%    |make_text| and some |run_script| uses \LuaTeX's |tex.runtoks|.
%    \begin{macrocode}
local catlatex = luatexbase.registernumber("catcodetable@latex")
local catat11  = luatexbase.registernumber("catcodetable@atletter")
%    \end{macrocode}
%
%    |tex.scantoks| sometimes fail to read catcode properly, especially
%    |\#|, |\&|, or |\%|. After some experiment, we dropped using it.
%    Instead, a function containing |tex.sprint| seems to work nicely.
%    \begin{macrocode}
local function run_tex_code (str, cat)
  texruntoks(function() texsprint(cat or catlatex, str) end)
end
%    \end{macrocode}
%
%    Prepare textext box number containers, locals and globals.
%    |localid| can be any number. They are local anyway.
%    The number will be reset at the start of a new code chunk.
%    Global boxes will use |\newbox| command in |tex.runtoks| process.
%    This is the same when |codeinherit| is true.
%    Boxes in instances with name will also be global, so that
%    their tex boxes can be shared among instances of the same name.
%    \begin{macrocode}
local texboxes = { globalid = 0, localid = 4096 }
%    \end{macrocode}
%    For conversion of |sp| to |bp|.
%    \begin{macrocode}
local factor = 65536*(7227/7200)
local textext_fmt = 'image(addto currentpicture doublepath unitsquare \z
  xscaled %f yscaled %f shifted (0,-%f) \z
  withprescript "mplibtexboxid=%i:%f:%f")'
local function process_tex_text (str, maketext)
  if str then
    if not maketext then str = str:gsub("\r.-$","") end
    local global = (has_instancename or luamplib.globaltextext or luamplib.codeinherit)
                   and "\\global" or ""
    local tex_box_id
    if global == "" then
      tex_box_id = texboxes.localid + 1
      texboxes.localid = tex_box_id
    else
      local boxid = texboxes.globalid + 1
      texboxes.globalid = boxid
      run_tex_code(format([[\expandafter\newbox\csname luamplib.box.%s\endcsname]], boxid))
      tex_box_id = tex.getcount'allocationnumber'
    end
    run_tex_code(format("\\luamplibtagtextbegin{%i}%s\\setbox%i\\hbox{%s}\\luamplibtagtextend", tex_box_id, global, tex_box_id, str))
    local box = texgetbox(tex_box_id)
    local wd  = box.width  / factor
    local ht  = box.height / factor
    local dp  = box.depth  / factor
    return textext_fmt:format(wd, ht+dp, dp, tex_box_id, wd, ht+dp)
  end
  return ""
end

%    \end{macrocode}
%
%    Make |color| or |xcolor|'s color expressions usable,
%    with \cs{mpcolor} or |mplibcolor|. These commands should be used
%    with graphical objects.
%    Attempt to support l3color as well.
%    \begin{macrocode}
local mplibcolorfmt = {
  xcolor = tableconcat{
    [[\begingroup\let\XC@mcolor\relax]],
    [[\def\set@color{\global\mplibtmptoks\expandafter{\current@color}}]],
    [[\color%s\endgroup]],
  },
  l3color = tableconcat{
    [[\begingroup\def\__color_select:N#1{\expandafter\__color_select:nn#1}]],
    [[\def\__color_backend_select:nn#1#2{\global\mplibtmptoks{#1 #2}}]],
    [[\def\__kernel_backend_literal:e#1{\global\mplibtmptoks\expandafter{\expanded{#1}}}]],
    [[\color_select:n%s\endgroup]],
  },
}
local colfmt = is_defined'color_select:n' and "l3color" or "xcolor"
if colfmt == "l3color" then
  run_tex_code{
    "\\newcatcodetable\\luamplibcctabexplat",
    "\\begingroup",
    "\\catcode`@=11 ",
    "\\catcode`_=11 ",
    "\\catcode`:=11 ",
    "\\savecatcodetable\\luamplibcctabexplat",
    "\\endgroup",
  }
end
local ccexplat = luatexbase.registernumber"luamplibcctabexplat"
local function process_color (str)
  if str then
    if not str:find("%b{}") then
      str = format("{%s}",str)
    end
    local myfmt = mplibcolorfmt[colfmt]
    if colfmt == "l3color" and is_defined"color" then
      if str:find("%b[]") then
        myfmt = mplibcolorfmt.xcolor
      else
        for _,v in ipairs(str:match"{(.+)}":explode"!") do
          if not v:find("^%s*%d+%s*$") then
            local pp = get_macro(format("l__color_named_%s_prop",v))
            if not pp or pp == "" then
              myfmt = mplibcolorfmt.xcolor
              break
            end
          end
        end
      end
    end
    run_tex_code(myfmt:format(str), ccexplat or catat11)
    local t = texgettoks"mplibtmptoks"
    if not pdfmode and not t:find"^pdf" then
      t = t:gsub("%a+ (.+)","pdf:bc [%1]")
    end
    return format('1 withprescript "mpliboverridecolor=%s"', t)
  end
  return ""
end

%    \end{macrocode}
%
%    for \cs{mpdim} or |mplibdimen|
%    \begin{macrocode}
local function process_dimen (str)
  if str then
    str = str:gsub("{(.+)}","%1")
    run_tex_code(format([[\mplibtmptoks\expandafter{\the\dimexpr %s\relax}]], str))
    return format("begingroup %s endgroup", texgettoks"mplibtmptoks")
  end
  return ""
end

%    \end{macrocode}
%
%    Newly introduced method of processing |verbatimtex ... etex|.
%    This function is used when |\mpliblegacybehavior{false}| is declared.
%    \begin{macrocode}
local function process_verbatimtex_text (str)
  if str then
    run_tex_code(str)
  end
  return ""
end

%    \end{macrocode}
%
%    For legacy verbatimtex process.
%    |verbatimtex ... etex| before |beginfig()| is not ignored,
%    but the \TeX\ code is inserted just before the \mplib box. And
%    \TeX\ code inside |beginfig() ... endfig| is inserted after the \mplib box.
%    \begin{macrocode}
local tex_code_pre_mplib = {}
luamplib.figid = 1
luamplib.in_the_fig = false
local function process_verbatimtex_prefig (str)
  if str then
    tex_code_pre_mplib[luamplib.figid] = str
  end
  return ""
end
local function process_verbatimtex_infig (str)
  if str then
    return format('special "postmplibverbtex=%s";', str)
  end
  return ""
end

local runscript_funcs = {
  luamplibtext    = process_tex_text,
  luamplibcolor   = process_color,
  luamplibdimen   = process_dimen,
  luamplibprefig  = process_verbatimtex_prefig,
  luamplibinfig   = process_verbatimtex_infig,
  luamplibverbtex = process_verbatimtex_text,
}

%    \end{macrocode}
%
%    For \emph{metafun} format. see issue \#79.
%    \begin{macrocode}
mp = mp or {}
local mp = mp
mp.mf_path_reset = mp.mf_path_reset or function() end
mp.mf_finish_saving_data = mp.mf_finish_saving_data or function() end
mp.report = mp.report or info
%    \end{macrocode}
%
%    \emph{metafun} 2021-03-09 changes crashes luamplib.
%    \begin{macrocode}
catcodes = catcodes or {}
local catcodes = catcodes
catcodes.numbers = catcodes.numbers or {}
catcodes.numbers.ctxcatcodes = catcodes.numbers.ctxcatcodes or catlatex
catcodes.numbers.texcatcodes = catcodes.numbers.texcatcodes or catlatex
catcodes.numbers.luacatcodes = catcodes.numbers.luacatcodes or catlatex
catcodes.numbers.notcatcodes = catcodes.numbers.notcatcodes or catlatex
catcodes.numbers.vrbcatcodes = catcodes.numbers.vrbcatcodes or catlatex
catcodes.numbers.prtcatcodes = catcodes.numbers.prtcatcodes or catlatex
catcodes.numbers.txtcatcodes = catcodes.numbers.txtcatcodes or catlatex

%    \end{macrocode}
%
%    A function from \ConTeXt\ general.
%    \begin{macrocode}
local function mpprint(buffer,...)
  for i=1,select("#",...) do
    local value = select(i,...)
    if value ~= nil then
      local t = type(value)
      if t == "number" then
        buffer[#buffer+1] = format("%.16f",value)
      elseif t == "string" then
        buffer[#buffer+1] = value
      elseif t == "table" then
        buffer[#buffer+1] = "(" .. tableconcat(value,",") .. ")"
      else -- boolean or whatever
        buffer[#buffer+1] = tostring(value)
      end
    end
  end
end
function luamplib.runscript (code)
  local id, str = code:match("(.-){(.*)}")
  if id and str then
    local f = runscript_funcs[id]
    if f then
      local t = f(str)
      if t then return t end
    end
  end
  local f = loadstring(code)
  if type(f) == "function" then
    local buffer = {}
    function mp.print(...)
      mpprint(buffer,...)
    end
    local res = {f()}
    buffer = tableconcat(buffer)
    if buffer and buffer ~= "" then
      return buffer
    end
    buffer = {}
    mpprint(buffer, tableunpack(res))
    return tableconcat(buffer)
  end
  return ""
end

%    \end{macrocode}
%
%    |make_text| must be one liner, so comment sign is not allowed.
%    \begin{macrocode}
local function protecttexcontents (str)
  return str:gsub("\\%%", "\0PerCent\0")
            :gsub("%%.-\n", "")
            :gsub("%%.-$",  "")
            :gsub("%zPerCent%z", "\\%%")
            :gsub("\r.-$",  "")
            :gsub("%s+", " ")
end
luamplib.legacyverbatimtex = true
function luamplib.maketext (str, what)
  if str and str ~= "" then
    str = protecttexcontents(str)
    if what == 1 then
      if not str:find("\\documentclass"..name_e) and
         not str:find("\\begin%s*{document}") and
         not str:find("\\documentstyle"..name_e) and
         not str:find("\\usepackage"..name_e) then
        if luamplib.legacyverbatimtex then
          if luamplib.in_the_fig then
            return process_verbatimtex_infig(str)
          else
            return process_verbatimtex_prefig(str)
          end
        else
          return process_verbatimtex_text(str)
        end
      end
    else
      return process_tex_text(str, true) -- bool is for 'char13'
    end
  end
  return ""
end

%    \end{macrocode}
%
%    luamplib's \metapost color operators
%    \begin{macrocode}
local function colorsplit (res)
  local t, tt = { }, res:gsub("[%[%]]","",2):explode()
  local be = tt[1]:find"^%d" and 1 or 2
  for i=be, #tt do
    if not tonumber(tt[i]) then break end
    t[#t+1] = tt[i]
  end
  return t
end

luamplib.gettexcolor = function (str, rgb)
  local res = process_color(str):match'"mpliboverridecolor=(.+)"'
  if res:find" cs " or res:find"@pdf.obj" then
    if not rgb then
      warn("%s is a spot color. Forced to CMYK", str)
    end
    run_tex_code({
      "\\color_export:nnN{",
      str,
      "}{",
      rgb and "space-sep-rgb" or "space-sep-cmyk",
      "}\\mplib_@tempa",
    },ccexplat)
    return get_macro"mplib_@tempa":explode()
  end
  local t = colorsplit(res)
  if #t == 3 or not rgb then return t end
  if #t == 4 then
    return { 1 - math.min(1,t[1]+t[4]), 1 - math.min(1,t[2]+t[4]), 1 - math.min(1,t[3]+t[4]) }
  end
  return { t[1], t[1], t[1] }
end

luamplib.shadecolor = function (str)
  local res = process_color(str):match'"mpliboverridecolor=(.+)"'
  if res:find" cs " or res:find"@pdf.obj" then -- spot color shade: l3 only
%    \end{macrocode}
%   An example of spot color shading:
%\begin{verbatim}
%     \DocumentMetadata{ }
%     \documentclass{article}
%     \usepackage{luamplib}
%     \ExplSyntaxOn
%     \color_model_new:nnn { pantone3005 }
%       { Separation }
%       {
%         name = PANTONE~3005~U ,
%         alternative-model = cmyk ,
%         alternative-values = {1, 0.56, 0, 0}
%       }
%       \color_set:nnn{spotA}{pantone3005}{1}
%       \color_set:nnn{spotB}{pantone3005}{0.6}
%     \color_model_new:nnn { pantone1215 }
%       { Separation }
%       {
%         name = PANTONE~1215~U ,
%         alternative-model = cmyk ,
%         alternative-values = {0, 0.15, 0.51, 0}
%       }
%       \color_set:nnn{spotC}{pantone1215}{1}
%     \color_model_new:nnn { pantone2040 }
%       { Separation }
%       {
%         name = PANTONE~2040~U ,
%         alternative-model = cmyk ,
%         alternative-values = {0, 0.28, 0.21, 0.04}
%       }
%       \color_set:nnn{spotD}{pantone2040}{1}
%     \ExplSyntaxOff
%     \begin{document}
%     \begin{mplibcode}
%     beginfig(1)
%       fill unitsquare xscaled \mpdim\textwidth yscaled 1cm
%            withshadingmethod "linear"
%            withshadingvector (0,1)
%            withshadingstep (
%                withshadingfraction .5
%                withshadingcolors ("spotB","spotC")
%            )
%            withshadingstep (
%                withshadingfraction 1
%                withshadingcolors ("spotC","spotD")
%            )
%       ;
%     endfig;
%     \end{mplibcode}
%     \end{document}
%\end{verbatim}
%   another one: user-defined DeviceN colorspace
%\begin{verbatim}
%    \DocumentMetadata{ }
%    \documentclass{article}
%    \usepackage{luamplib}
%    \ExplSyntaxOn
%    \color_model_new:nnn { pantone1215 }
%      { Separation }
%      {
%        name = PANTONE~1215~U ,
%        alternative-model = cmyk ,
%        alternative-values = {0, 0.15, 0.51, 0}
%      }
%    \color_model_new:nnn { pantone+black }
%      { DeviceN }
%      { names = {pantone1215,black} }
%    \color_set:nnn{purepantone}{pantone+black}{1,0}
%    \color_set:nnn{pureblack}  {pantone+black}{0,1}
%    \ExplSyntaxOff
%    \begin{document}
%    \mpfig
%    fill unitsquare xscaled \mpdim{\textwidth} yscaled 30
%        withshadingmethod "linear"
%        withshadingcolors ("purepantone","pureblack")
%        ;
%    \endmpfig
%    \end{document}
%\end{verbatim}
%    \begin{macrocode}
    run_tex_code({
      [[\color_export:nnN{]], str, [[}{backend}\mplib_@tempa]],
    },ccexplat)
    local name, value = get_macro'mplib_@tempa':match'{(.-)}{(.-)}'
    local t, obj = res:explode()
    if pdfmode then
      obj = format("%s 0 R", ltx.pdf.object_id( t[1]:sub(2,-1) ))
    else
      obj = t[2]
    end
    return format('(1) withprescript"mplib_spotcolor=%s:%s:%s"', value,obj,name)
  end
  return colorsplit(res)
end

%    \end{macrocode}
%
% Remove trailing zeros for smaller PDF
%    \begin{macrocode}
local decimals = "%.%d+"
local function rmzeros(str) return str:gsub("%.?0+$","") end

%    \end{macrocode}
%
%    luamplib's mplibgraphictext operator
%    \begin{macrocode}
local emboldenfonts = { }
local function getemboldenwidth (curr, fakebold)
  local width = emboldenfonts.width
  if not width then
    local f
    local function getglyph(n)
      while n do
        if n.head then
          getglyph(n.head)
        elseif n.font and n.font > 0 then
          f = n.font; break
        end
        n = node.getnext(n)
      end
    end
    getglyph(curr)
    width = font.getcopy(f or font.current()).size * fakebold / factor * 10
    emboldenfonts.width = width
  end
  return width
end
local function getrulewhatsit (line, wd, ht, dp)
  line, wd, ht, dp = line/1000, wd/factor, ht/factor, dp/factor
  local pl
  local fmt = "%f w %f %f %f %f re %s"
  if pdfmode then
    pl = node.new("whatsit","pdf_literal")
    pl.mode = 0
  else
    fmt = "pdf:content "..fmt
    pl = node.new("whatsit","special")
  end
  pl.data = fmt:format(line, 0, -dp, wd, ht+dp, "B") :gsub(decimals,rmzeros)
  local ss = node.new"glue"
  node.setglue(ss, 0, 65536, 65536, 2, 2)
  pl.next = ss
  return pl
end
local function getrulemetric (box, curr, bp)
  local running = -1073741824
  local wd,ht,dp = curr.width, curr.height, curr.depth
  wd = wd == running and box.width  or wd
  ht = ht == running and box.height or ht
  dp = dp == running and box.depth  or dp
  if bp then
    return wd/factor, ht/factor, dp/factor
  end
  return wd, ht, dp
end
local function embolden (box, curr, fakebold)
  local head = curr
  while curr do
    if curr.head then
      curr.head = embolden(curr, curr.head, fakebold)
    elseif curr.replace then
      curr.replace = embolden(box, curr.replace, fakebold)
    elseif curr.leader then
      if curr.leader.head then
        curr.leader.head = embolden(curr.leader, curr.leader.head, fakebold)
      elseif curr.leader.id == node.id"rule" then
        local glue = node.effective_glue(curr, box)
        local line = getemboldenwidth(curr, fakebold)
        local wd,ht,dp = getrulemetric(box, curr.leader)
        if box.id == node.id"hlist" then
          wd = glue
        else
          ht, dp = 0, glue
        end
        local pl = getrulewhatsit(line, wd, ht, dp)
        local pack = box.id == node.id"hlist" and node.hpack or node.vpack
        local list = pack(pl, glue, "exactly")
        head = node.insert_after(head, curr, list)
        head, curr = node.remove(head, curr)
      end
    elseif curr.id == node.id"rule" and curr.subtype == 0 then
      local line = getemboldenwidth(curr, fakebold)
      local wd,ht,dp = getrulemetric(box, curr)
      if box.id == node.id"vlist" then
        ht, dp = 0, ht+dp
      end
      local pl = getrulewhatsit(line, wd, ht, dp)
      local list
      if box.id == node.id"hlist" then
        list = node.hpack(pl, wd, "exactly")
      else
        list = node.vpack(pl, ht+dp, "exactly")
      end
      head = node.insert_after(head, curr, list)
      head, curr = node.remove(head, curr)
    elseif curr.id == node.id"glyph" and curr.font > 0 then
      local f = curr.font
      local key = format("%s:%s",f,fakebold)
      local i = emboldenfonts[key]
      if not i then
        local ft = font.getfont(f) or font.getcopy(f)
        if pdfmode then
          width = ft.size * fakebold / factor * 10
          emboldenfonts.width = width
          ft.mode, ft.width = 2, width
          i = font.define(ft)
        else
          if ft.format ~= "opentype" and ft.format ~= "truetype" then
            goto skip_type1
          end
          local name = ft.name:gsub('"',''):gsub(';$','')
          name = format('%s;embolden=%s;',name,fakebold)
          _, i = fonts.constructors.readanddefine(name,ft.size)
        end
        emboldenfonts[key] = i
      end
      curr.font = i
    end
    ::skip_type1::
    curr = node.getnext(curr)
  end
  return head
end
local function graphictextcolor (col, filldraw)
  if col:find"^[%d%.:]+$" then
    col = col:explode":"
    for i=1,#col do
      col[i] = format("%.3f", col[i])
    end
    if pdfmode then
      local op = #col == 4 and "k" or #col == 3 and "rg" or "g"
      col[#col+1] = filldraw == "fill" and op or op:upper()
      return tableconcat(col," ")
    end
    return format("[%s]", tableconcat(col," "))
  end
  col = process_color(col):match'"mpliboverridecolor=(.+)"'
  if pdfmode then
    local t, tt = col:explode(), { }
    local b = filldraw == "fill" and 1 or #t/2+1
    local e = b == 1 and #t/2 or #t
    for i=b,e do
      tt[#tt+1] = t[i]
    end
    return tableconcat(tt," ")
  end
  return col:gsub("^.- ","")
end
luamplib.graphictext = function (text, fakebold, fc, dc)
  local fmt = process_tex_text(text):sub(1,-2)
  local id = tonumber(fmt:match"mplibtexboxid=(%d+):")
  emboldenfonts.width = nil
  local box = texgetbox(id)
  box.head = embolden(box, box.head, fakebold)
  local fill = graphictextcolor(fc,"fill")
  local draw = graphictextcolor(dc,"draw")
  local bc = pdfmode and "" or "pdf:bc "
  return format('%s withprescript "mpliboverridecolor=%s%s %s")', fmt, bc, fill, draw)
end

%    \end{macrocode}
%
%    luamplib's mplibglyph operator
%    \begin{macrocode}
local function mperr (str)
  return format("hide(errmessage %q)", str)
end
local function getangle (a,b,c)
  local r = math.deg(math.atan(c.y-b.y, c.x-b.x) - math.atan(b.y-a.y, b.x-a.x))
  if r > 180 then
    r = r - 360
  elseif r < -180 then
    r = r + 360
  end
  return r
end
local function turning (t)
  local r, n = 0, #t
  for i=1,2 do
    tableinsert(t, t[i])
  end
  for i=1,n do
    r = r + getangle(t[i], t[i+1], t[i+2])
  end
  return r/360
end
local function glyphimage(t, fmt)
  local q,p,r = {{},{}}
  for i,v in ipairs(t) do
    local cmd = v[#v]
    if cmd == "m" then
      p = {format('(%s,%s)',v[1],v[2])}
      r = {{x=v[1],y=v[2]}}
    else
      local nt = t[i+1]
      local last = not nt or nt[#nt] == "m"
      if cmd == "l" then
        local pt = t[i-1]
        local seco = pt[#pt] == "m"
        if (last or seco) and r[1].x == v[1] and r[1].y == v[2] then
        else
          tableinsert(p, format('--(%s,%s)',v[1],v[2]))
          tableinsert(r, {x=v[1],y=v[2]})
        end
        if last then
          tableinsert(p, '--cycle')
        end
      elseif cmd == "c" then
        tableinsert(p, format('..controls(%s,%s)and(%s,%s)',v[1],v[2],v[3],v[4]))
        if last and r[1].x == v[5] and r[1].y == v[6] then
          tableinsert(p, '..cycle')
        else
          tableinsert(p, format('..(%s,%s)',v[5],v[6]))
          if last then
            tableinsert(p, '--cycle')
          end
          tableinsert(r, {x=v[5],y=v[6]})
        end
      else
        return mperr"unknown operator"
      end
      if last then
        tableinsert(q[ turning(r) > 0 and 1 or 2 ], tableconcat(p))
      end
    end
  end
  r = { }
  if fmt == "opentype" then
    for _,v in ipairs(q[1]) do
      tableinsert(r, format('addto currentpicture contour %s;',v))
    end
    for _,v in ipairs(q[2]) do
      tableinsert(r, format('addto currentpicture contour %s withcolor background;',v))
    end
  else
    for _,v in ipairs(q[2]) do
      tableinsert(r, format('addto currentpicture contour %s;',v))
    end
    for _,v in ipairs(q[1]) do
      tableinsert(r, format('addto currentpicture contour %s withcolor background;',v))
    end
  end
  return format('image(%s)', tableconcat(r))
end
if not table.tofile then require"lualibs-lpeg"; require"lualibs-table"; end
function luamplib.glyph (f, c)
  local filename, subfont, instance, kind, shapedata
  local fid = tonumber(f) or font.id(f)
  if fid > 0 then
    local fontdata = font.getfont(fid) or font.getcopy(fid)
    filename, subfont, kind = fontdata.filename, fontdata.subfont, fontdata.format
    instance = fontdata.specification and fontdata.specification.instance
    filename = filename and filename:gsub("^harfloaded:","")
  else
    local name
    f = f:match"^%s*(.+)%s*$"
    name, subfont, instance = f:match"(.+)%((%d+)%)%[(.-)%]$"
    if not name then
      name, instance = f:match"(.+)%[(.-)%]$" -- SourceHanSansK-VF.otf[Heavy]
    end
    if not name then
      name, subfont = f:match"(.+)%((%d+)%)$" -- Times.ttc(2)
    end
    name = name or f
    subfont = (subfont or 0)+1
    instance = instance and instance:lower()
    for _,ftype in ipairs{"opentype", "truetype"} do
      filename = kpse.find_file(name, ftype.." fonts")
      if filename then
        kind = ftype; break
      end
    end
  end
  if kind ~= "opentype" and kind ~= "truetype" then
    f = fid and fid > 0 and tex.fontname(fid) or f
    if kpse.find_file(f, "tfm") then
      return format("glyph %s of %q", tonumber(c) or format("%q",c), f)
    else
      return mperr"font not found"
    end
  end
  local time = lfsattributes(filename,"modification")
  local k = format("shapes_%s(%s)[%s]", filename, subfont or "", instance or "")
  local h = format(string.rep('%02x', 256/8), string.byte(sha2.digest256(k), 1, -1))
  local newname = format("%s/%s.lua", cachedir or outputdir, h)
  local newtime = lfsattributes(newname,"modification") or 0
  if time == newtime then
    shapedata = require(newname)
  end
  if not shapedata then
    shapedata = fonts and fonts.handlers.otf.readers.loadshapes(filename,subfont,instance)
    if not shapedata then return mperr"loadshapes() failed. luaotfload not loaded?" end
    table.tofile(newname, shapedata, "return")
    lfstouch(newname, time, time)
  end
  local gid = tonumber(c)
  if not gid then
    local uni = utf8.codepoint(c)
    for i,v in pairs(shapedata.glyphs) do
      if c == v.name or uni == v.unicode then
        gid = i; break
      end
    end
  end
  if not gid then return mperr"cannot get GID (glyph id)" end
  local fac = 1000 / (shapedata.units or 1000)
  local t = shapedata.glyphs[gid].segments
  if not t then return "image()" end
  for i,v in ipairs(t) do
    if type(v) == "table" then
      for ii,vv in ipairs(v) do
        if type(vv) == "number" then
          t[i][ii] = format("%.0f", vv * fac)
        end
      end
    end
  end
  kind = shapedata.format or kind
  return glyphimage(t, kind)
end

%    \end{macrocode}
%
%    mpliboutlinetext : based on mkiv's font-mps.lua
%    \begin{macrocode}
local rulefmt = "mpliboutlinepic[%i]:=image(addto currentpicture contour \z
  unitsquare shifted - center unitsquare;) xscaled %f yscaled %f shifted (%f,%f);"
local outline_horz, outline_vert
function outline_vert (res, box, curr, xshift, yshift)
  local b2u = box.dir == "LTL"
  local dy = (b2u and -box.depth or box.height)/factor
  local ody = dy
  while curr do
    if curr.id == node.id"rule" then
      local wd, ht, dp = getrulemetric(box, curr, true)
      local hd = ht + dp
      if hd ~= 0 then
        dy = dy + (b2u and dp or -ht)
        if wd ~= 0 and curr.subtype == 0 then
          res[#res+1] = rulefmt:format(#res+1, wd, hd, xshift+wd/2, yshift+dy+(ht-dp)/2)
        end
        dy = dy + (b2u and ht or -dp)
      end
    elseif curr.id == node.id"glue" then
      local vwidth = node.effective_glue(curr,box)/factor
      if curr.leader then
        local curr, kind = curr.leader, curr.subtype
        if curr.id == node.id"rule" then
          local wd = getrulemetric(box, curr, true)
          if wd ~= 0 then
            local hd = vwidth
            local dy = dy + (b2u and 0 or -hd)
            if hd ~= 0 and curr.subtype == 0 then
              res[#res+1] = rulefmt:format(#res+1, wd, hd, xshift+wd/2, yshift+dy+hd/2)
            end
          end
        elseif curr.head then
          local hd = (curr.height + curr.depth)/factor
          if hd <= vwidth then
            local dy, n, iy = dy, 0, 0
            if kind == 100 or kind == 103 then -- todo: gleaders
              local ady = abs(ody - dy)
              local ndy = math.ceil(ady / hd) * hd
              local diff = ndy - ady
              n = math.floor((vwidth-diff) / hd)
              dy = dy + (b2u and diff or -diff)
            else
              n = math.floor(vwidth / hd)
              if kind == 101 then
                local side = vwidth % hd / 2
                dy = dy + (b2u and side or -side)
              elseif kind == 102 then
                iy = vwidth % hd / (n+1)
                dy = dy + (b2u and iy or -iy)
              end
            end
            dy = dy + (b2u and curr.depth or -curr.height)/factor
            hd = b2u and hd or -hd
            iy = b2u and iy or -iy
            local func = curr.id == node.id"hlist" and outline_horz or outline_vert
            for i=1,n do
              res = func(res, curr, curr.head, xshift+curr.shift/factor, yshift+dy)
              dy = dy + hd + iy
            end
          end
        end
      end
      dy = dy + (b2u and vwidth or -vwidth)
    elseif curr.id == node.id"kern" then
      dy = dy + curr.kern/factor * (b2u and 1 or -1)
    elseif curr.id == node.id"vlist" then
      dy = dy + (b2u and curr.depth or -curr.height)/factor
      res = outline_vert(res, curr, curr.head, xshift+curr.shift/factor, yshift+dy)
      dy = dy + (b2u and curr.height or -curr.depth)/factor
    elseif curr.id == node.id"hlist" then
      dy = dy + (b2u and curr.depth or -curr.height)/factor
      res = outline_horz(res, curr, curr.head, xshift+curr.shift/factor, yshift+dy)
      dy = dy + (b2u and curr.height or -curr.depth)/factor
    end
    curr = node.getnext(curr)
  end
  return res
end
function outline_horz (res, box, curr, xshift, yshift, discwd)
  local r2l = box.dir == "TRT"
  local dx = r2l and (discwd or box.width/factor) or 0
  local dirs = { { dir = r2l, dx = dx } }
  while curr do
    if curr.id == node.id"dir" then
      local sign, dir = curr.dir:match"(.)(...)"
      local level, newdir = curr.level, r2l
      if sign == "+" then
        newdir = dir == "TRT"
        if r2l ~= newdir then
          local n = node.getnext(curr)
          while n do
            if n.id == node.id"dir" and n.level+1 == level then break end
            n = node.getnext(n)
          end
          n = n or node.tail(curr)
          dx = dx + node.rangedimensions(box, curr, n)/factor * (newdir and 1 or -1)
        end
        dirs[level] = { dir = r2l, dx = dx }
      else
        local level = level + 1
        newdir = dirs[level].dir
        if r2l ~= newdir then
          dx = dirs[level].dx
        end
      end
      r2l = newdir
    elseif curr.char and curr.font and curr.font > 0 then
      local ft = font.getfont(curr.font) or font.getcopy(curr.font)
      local gid = ft.characters[curr.char].index or curr.char
      local scale = ft.size / factor / 1000
      local slant   = (ft.slant or 0)/1000
      local extend  = (ft.extend or 1000)/1000
      local squeeze = (ft.squeeze or 1000)/1000
      local expand  = 1 + (curr.expansion_factor or 0)/1000000
      local xscale = scale * extend * expand
      local yscale = scale * squeeze
      dx = dx - (r2l and curr.width/factor*expand or 0)
      local xpos = dx + xshift + (curr.xoffset or 0)/factor
      local ypos = yshift + (curr.yoffset or 0)/factor
      local vertical = ft.shared and ft.shared.features.vertical and "rotated 90" or ""
      if vertical ~= "" then -- luatexko
        for _,v in ipairs(ft.characters[curr.char].commands or { }) do
          if v[1] == "down" then
            ypos = ypos - v[2] / factor
          elseif v[1] == "right" then
            xpos = xpos + v[2] / factor
          else
            break
          end
        end
      end
      local image
      if ft.format == "opentype" or ft.format == "truetype" then
        image = luamplib.glyph(curr.font, gid)
      else
        local name, scale = ft.name, 1
        local vf = font.read_vf(name, ft.size)
        if vf and vf.characters[gid] then
          local cmds = vf.characters[gid].commands or {}
          for _,v in ipairs(cmds) do
            if v[1] == "char" then
              gid = v[2]
            elseif v[1] == "font" and vf.fonts[v[2]] then
              name  = vf.fonts[v[2]].name
              scale = vf.fonts[v[2]].size / ft.size
            end
          end
        end
        image = format("glyph %s of %q scaled %f", gid, name, scale)
      end
      res[#res+1] = format("mpliboutlinepic[%i]:=%s xscaled %f yscaled %f slanted %f %s shifted (%f,%f);",
                           #res+1, image, xscale, yscale, slant, vertical, xpos, ypos)
      dx = dx + (r2l and 0 or curr.width/factor*expand)
    elseif curr.replace then
      local width = node.dimensions(curr.replace)/factor
      dx = dx - (r2l and width or 0)
      res = outline_horz(res, box, curr.replace, xshift+dx, yshift, width)
      dx = dx + (r2l and 0 or width)
    elseif curr.id == node.id"rule" then
      local wd, ht, dp = getrulemetric(box, curr, true)
      if wd ~= 0 then
        local hd = ht + dp
        dx = dx - (r2l and wd or 0)
        if hd ~= 0 and curr.subtype == 0 then
          res[#res+1] = rulefmt:format(#res+1, wd, hd, xshift+dx+wd/2, yshift+(ht-dp)/2)
        end
        dx = dx + (r2l and 0 or wd)
      end
    elseif curr.id == node.id"glue" then
      local width = node.effective_glue(curr, box)/factor
      dx = dx - (r2l and width or 0)
      if curr.leader then
        local curr, kind = curr.leader, curr.subtype
        if curr.id == node.id"rule" then
          local wd, ht, dp = getrulemetric(box, curr, true)
          local hd = ht + dp
          if hd ~= 0 then
            wd = width
            if wd ~= 0 and curr.subtype == 0 then
              res[#res+1] = rulefmt:format(#res+1, wd, hd, xshift+dx+wd/2, yshift+(ht-dp)/2)
            end
          end
        elseif curr.head then
          local wd = curr.width/factor
          if wd <= width then
            local dx = r2l and dx+width or dx
            local n, ix = 0, 0
            if kind == 100 or kind == 103 then -- todo: gleaders
              local adx = abs(dx-dirs[1].dx)
              local ndx = math.ceil(adx / wd) * wd
              local diff = ndx - adx
              n = math.floor((width-diff) / wd)
              dx = dx + (r2l and -diff-wd or diff)
            else
              n = math.floor(width / wd)
              if kind == 101 then
                local side = width % wd /2
                dx = dx + (r2l and -side-wd or side)
              elseif kind == 102 then
                ix = width % wd / (n+1)
                dx = dx + (r2l and -ix-wd or ix)
              end
            end
            wd = r2l and -wd or wd
            ix = r2l and -ix or ix
            local func = curr.id == node.id"hlist" and outline_horz or outline_vert
            for i=1,n do
              res = func(res, curr, curr.head, xshift+dx, yshift-curr.shift/factor)
              dx = dx + wd + ix
            end
          end
        end
      end
      dx = dx + (r2l and 0 or width)
    elseif curr.id == node.id"kern" then
      dx = dx + curr.kern/factor * (r2l and -1 or 1)
    elseif curr.id == node.id"math" then
      dx = dx + curr.surround/factor * (r2l and -1 or 1)
    elseif curr.id == node.id"vlist" then
      dx = dx - (r2l and curr.width/factor or 0)
      res = outline_vert(res, curr, curr.head, xshift+dx, yshift-curr.shift/factor)
      dx = dx + (r2l and 0 or curr.width/factor)
    elseif curr.id == node.id"hlist" then
      dx = dx - (r2l and curr.width/factor or 0)
      res = outline_horz(res, curr, curr.head, xshift+dx, yshift-curr.shift/factor)
      dx = dx + (r2l and 0 or curr.width/factor)
    end
    curr = node.getnext(curr)
  end
  return res
end
function luamplib.outlinetext (text)
  local fmt = process_tex_text(text)
  local id  = tonumber(fmt:match"mplibtexboxid=(%d+):")
  local box = texgetbox(id)
  local res = outline_horz({ }, box, box.head, 0, 0)
  if #res == 0 then res = { "mpliboutlinepic[1]:=image();" } end
  return tableconcat(res) .. format("mpliboutlinenum:=%i;", #res)
end

%    \end{macrocode}
%
%    lua functions for |mplib(uc)substring ... of ...|
%    \begin{macrocode}
function luamplib.getunicodegraphemes (s)
  local t = { }
  local graphemes = require'lua-uni-graphemes'
  for _, _, c in graphemes.graphemes(s) do
    table.insert(t, c)
  end
  return t
end
function luamplib.unicodesubstring (s,b,e,grph)
  local tt, t, step = { }
  if grph then
    t = luamplib.getunicodegraphemes(s)
  else
    t = { }
    for _, c in utf8.codes(s) do
      table.insert(t, utf8.char(c))
    end
  end
  if b <= e then
    b, step = b+1, 1
  else
    e, step = e+1, -1
  end
  for i = b, e, step do
    table.insert(tt, t[i])
  end
  s = table.concat(tt):gsub('"','"&ditto&"')
  return string.format('"%s"', s)
end

%    \end{macrocode}
%
%    Our \metapost preambles
%    \begin{macrocode}
luamplib.preambles = {
  mplibcode = [[
texscriptmode := 2;
def rawtextext primary t = runscript("luamplibtext{"&t&"}") enddef;
def mplibcolor primary t = runscript("luamplibcolor{"&t&"}") enddef;
def mplibdimen primary t = runscript("luamplibdimen{"&t&"}") enddef;
def VerbatimTeX primary t = runscript("luamplibverbtex{"&t&"}") enddef;
if known context_mlib:
  defaultfont := "cmtt10";
  let infont = normalinfont;
  let fontsize = normalfontsize;
  vardef thelabel@#(expr p,z) =
    if string p :
      thelabel@#(p infont defaultfont scaled defaultscale,z)
    else :
      p shifted (z + labeloffset*mfun_laboff@# -
        (mfun_labxf@#*lrcorner p + mfun_labyf@#*ulcorner p +
        (1-mfun_labxf@#-mfun_labyf@#)*llcorner p))
    fi
  enddef;
else:
  vardef textext@# primary t = rawtextext (t) enddef;
  def message expr t =
    if string t: runscript("mp.report[=["&t&"]=]") else: errmessage "Not a string" fi
  enddef;
  def withtransparency (expr a, t) =
    withprescript "tr_alternative=" & if numeric a: decimal fi a
    withprescript "tr_transparency=" & decimal t
  enddef;
  vardef ddecimal primary p =
    decimal xpart p & " " & decimal ypart p
  enddef;
  vardef boundingbox primary p =
    if (path p) or (picture p) :
      llcorner p -- lrcorner p -- urcorner p -- ulcorner p
    else :
      origin
    fi -- cycle
  enddef;
fi
def resolvedcolor(expr s) =
  runscript("return luamplib.shadecolor('"& s &"')")
enddef;
def colordecimals primary c =
  if cmykcolor c:
    decimal cyanpart c & ":" & decimal magentapart c & ":" &
    decimal yellowpart c & ":" & decimal blackpart c
  elseif rgbcolor c:
    decimal redpart c & ":" & decimal greenpart c & ":" & decimal bluepart c
  elseif string c:
    if known graphictextpic: c else: colordecimals resolvedcolor(c) fi
  else:
    decimal c
  fi
enddef;
def externalfigure primary filename =
  draw rawtextext("\includegraphics{"& filename &"}")
enddef;
def TEX = textext enddef;
def mplibtexcolor primary c =
  runscript("return luamplib.gettexcolor('"& c &"')")
enddef;
def mplibrgbtexcolor primary c =
  runscript("return luamplib.gettexcolor('"& c &"','rgb')")
enddef;
def mplibgraphictext primary t =
  begingroup;
  mplibgraphictext_ (t)
enddef;
def mplibgraphictext_ (expr t) text rest =
  save fakebold, scale, fillcolor, drawcolor, withfillcolor, withdrawcolor,
    fb, fc, dc, graphictextpic, alsoordoublepath;
  picture graphictextpic; graphictextpic := nullpicture;
  numeric fb; string fc, dc; fb:=2; fc:="white"; dc:="black";
  let scale = scaled;
  def fakebold  primary c = hide(fb:=c;) enddef;
  def fillcolor primary c = hide(fc:=colordecimals c;) enddef;
  def drawcolor primary c = hide(dc:=colordecimals c;) enddef;
  let withfillcolor = fillcolor; let withdrawcolor = drawcolor;
  def alsoordoublepath expr p = if picture p: also else: doublepath fi p enddef;
  addto graphictextpic alsoordoublepath (origin--cycle) rest; graphictextpic:=nullpicture;
  def fakebold  primary c = enddef;
  let fillcolor = fakebold; let drawcolor = fakebold;
  let withfillcolor = fillcolor; let withdrawcolor = drawcolor;
  image(draw runscript("return luamplib.graphictext([===["&t&"]===],"
    & decimal fb &",'"& fc &"','"& dc &"')") rest;)
  endgroup;
enddef;
def mplibglyph expr c of f =
  runscript (
    "return luamplib.glyph('"
    & if numeric f: decimal fi f
    & "','"
    & if numeric c: decimal fi c
    & "')"
  )
enddef;
def mplibdrawglyph expr g =
  draw image(
    save i; numeric i; i:=0;
    for item within g:
      i := i+1;
      fill pathpart item
      if i < length g: withpostscript "collect" fi;
    endfor
  )
enddef;
def mplib_do_outline_text_set_b (text f) (text d) text r =
  def mplib_do_outline_options_f = f enddef;
  def mplib_do_outline_options_d = d enddef;
  def mplib_do_outline_options_r = r enddef;
enddef;
def mplib_do_outline_text_set_f (text f) text r =
  def mplib_do_outline_options_f = f enddef;
  def mplib_do_outline_options_r = r enddef;
enddef;
def mplib_do_outline_text_set_u (text f) text r =
  def mplib_do_outline_options_f = f enddef;
enddef;
def mplib_do_outline_text_set_d (text d) text r =
  def mplib_do_outline_options_d = d enddef;
  def mplib_do_outline_options_r = r enddef;
enddef;
def mplib_do_outline_text_set_r (text d) (text f) text r =
  def mplib_do_outline_options_d = d enddef;
  def mplib_do_outline_options_f = f enddef;
  def mplib_do_outline_options_r = r enddef;
enddef;
def mplib_do_outline_text_set_n text r =
  def mplib_do_outline_options_r = r enddef;
enddef;
def mplib_do_outline_text_set_p = enddef;
def mplib_fill_outline_text =
  for n=1 upto mpliboutlinenum:
    i:=0;
    for item within mpliboutlinepic[n]:
      i:=i+1;
      fill pathpart item mplib_do_outline_options_f withpen pencircle scaled 0
      if (n<mpliboutlinenum) or (i<length mpliboutlinepic[n]): withpostscript "collect"; fi
    endfor
  endfor
enddef;
def mplib_draw_outline_text =
  for n=1 upto mpliboutlinenum:
    for item within mpliboutlinepic[n]:
      draw pathpart item mplib_do_outline_options_d;
    endfor
  endfor
enddef;
def mplib_filldraw_outline_text =
  for n=1 upto mpliboutlinenum:
    i:=0;
    for item within mpliboutlinepic[n]:
      i:=i+1;
      if (n<mpliboutlinenum) or (i<length mpliboutlinepic[n]):
        fill pathpart item mplib_do_outline_options_f withpostscript "collect";
      else:
        draw pathpart item mplib_do_outline_options_f withpostscript "both";
      fi
    endfor
  endfor
enddef;
vardef mpliboutlinetext@# (expr t) text rest =
  save kind; string kind; kind := str @#;
  save i; numeric i;
  picture mpliboutlinepic[]; numeric mpliboutlinenum;
  def mplib_do_outline_options_d = enddef;
  def mplib_do_outline_options_f = enddef;
  def mplib_do_outline_options_r = enddef;
  runscript("return luamplib.outlinetext[===["&t&"]===]");
  image ( addto currentpicture also image (
    if kind = "f":
      mplib_do_outline_text_set_f rest;
      mplib_fill_outline_text;
    elseif kind = "d":
      mplib_do_outline_text_set_d rest;
      mplib_draw_outline_text;
    elseif kind = "b":
      mplib_do_outline_text_set_b rest;
      mplib_fill_outline_text;
      mplib_draw_outline_text;
    elseif kind = "u":
      mplib_do_outline_text_set_u rest;
      mplib_filldraw_outline_text;
    elseif kind = "r":
      mplib_do_outline_text_set_r rest;
      mplib_draw_outline_text;
      mplib_fill_outline_text;
    elseif kind = "p":
      mplib_do_outline_text_set_p;
      mplib_draw_outline_text;
    else:
      mplib_do_outline_text_set_n rest;
      mplib_fill_outline_text;
    fi;
  ) mplib_do_outline_options_r; )
enddef ;
def withmppattern primary p =
  withprescript "mplibpattern=" & if numeric p: decimal fi p
enddef;
primarydef t withpattern p =
  image(
    if cycle t:
      fill
    else:
      draw
    fi
    t withprescript "mplibpattern=" & if numeric p: decimal fi p; )
enddef;
vardef mplibtransformmatrix (text e) =
  save t; transform t;
  t = identity e;
  runscript("luamplib.transformmatrix = {"
  & decimal xxpart t & ","
  & decimal yxpart t & ","
  & decimal xypart t & ","
  & decimal yypart t & ","
  & decimal xpart  t & ","
  & decimal ypart  t & ","
  & "}");
enddef;
primarydef p withfademethod s =
  if picture p:
    image(
      draw p;
      draw center p withprescript "mplibfadestate=stop";
    )
  else:
    p withprescript "mplibfadestate=stop"
  fi
    withprescript "mplibfadetype=" & s
    withprescript "mplibfadebbox=" &
      decimal (xpart llcorner p -1/4) & ":" &
      decimal (ypart llcorner p -1/4) & ":" &
      decimal (xpart urcorner p +1/4) & ":" &
      decimal (ypart urcorner p +1/4)
enddef;
def withfadeopacity (expr a,b) =
  withprescript "mplibfadeopacity=" &
    decimal a & ":" &
    decimal b
enddef;
def withfadevector (expr a,b) =
  withprescript "mplibfadevector=" &
    decimal xpart a & ":" &
    decimal ypart a & ":" &
    decimal xpart b & ":" &
    decimal ypart b
enddef;
let withfadecenter = withfadevector;
def withfaderadius (expr a,b) =
  withprescript "mplibfaderadius=" &
    decimal a & ":" &
    decimal b
enddef;
def withfadebbox (expr a,b) =
  withprescript "mplibfadebbox=" &
    decimal xpart a & ":" &
    decimal ypart a & ":" &
    decimal xpart b & ":" &
    decimal ypart b
enddef;
primarydef p asgroup s =
  image(
    draw center p
      withprescript "mplibgroupbbox=" &
        decimal (xpart llcorner p -1/4) & ":" &
        decimal (ypart llcorner p -1/4) & ":" &
        decimal (xpart urcorner p +1/4) & ":" &
        decimal (ypart urcorner p +1/4)
      withprescript "gr_state=start"
      withprescript "gr_type=" & s;
    draw p;
    draw center p withprescript "gr_state=stop";
  )
enddef;
def withgroupbbox (expr a,b) =
  withprescript "mplibgroupbbox=" &
    decimal xpart a & ":" &
    decimal ypart a & ":" &
    decimal xpart b & ":" &
    decimal ypart b
enddef;
def withgroupname expr s =
  withprescript "mplibgroupname=" & s
enddef;
def usemplibgroup primary s =
  draw maketext("\csname luamplib.group." & s & "\endcsname")
    shifted runscript("return luamplib.trgroupshifts['" & s & "']")
enddef;
path    mplib_shade_path ;
numeric mplib_shade_step ; mplib_shade_step := 0 ;
numeric mplib_shade_fx, mplib_shade_fy ;
numeric mplib_shade_lx, mplib_shade_ly ;
numeric mplib_shade_nx, mplib_shade_ny ;
numeric mplib_shade_dx, mplib_shade_dy ;
numeric mplib_shade_tx, mplib_shade_ty ;
primarydef p withshadingmethod m =
  p
  if picture p :
    withprescript "sh_operand_type=picture"
    if textual p:
      withprescript "sh_transform=no"
      mplib_with_shade_method (boundingbox p, m)
    else:
      withprescript "sh_transform=yes"
      mplib_with_shade_method (pathpart p, m)
    fi
  else :
    withprescript "sh_transform=yes"
    mplib_with_shade_method (p, m)
  fi
enddef;
def mplib_with_shade_method (expr p, m) =
  hide(mplib_with_shade_method_analyze(p))
  withprescript "sh_domain=0 1"
  withprescript "sh_color=into"
  withprescript "sh_color_a=" & colordecimals white
  withprescript "sh_color_b=" & colordecimals black
  withprescript "sh_first=" & ddecimal point 0 of p
  withprescript "sh_set_x=" & ddecimal (mplib_shade_nx,mplib_shade_lx)
  withprescript "sh_set_y=" & ddecimal (mplib_shade_ny,mplib_shade_ly)
  if m = "linear" :
    withprescript "sh_type=linear"
    withprescript "sh_factor=1"
    withprescript "sh_center_a=" & ddecimal llcorner p
    withprescript "sh_center_b=" & ddecimal urcorner p
  else :
    withprescript "sh_type=circular"
    withprescript "sh_factor=1.2"
    withprescript "sh_center_a=" & ddecimal center p
    withprescript "sh_center_b=" & ddecimal center p
    withprescript "sh_radius_a=" & decimal 0
    withprescript "sh_radius_b=" & decimal mplib_max_radius(p)
  fi
enddef;
def mplib_with_shade_method_analyze(expr p) =
  mplib_shade_path := p ;
  mplib_shade_step := 1 ;
  mplib_shade_fx   := xpart point 0 of p ;
  mplib_shade_fy   := ypart point 0 of p ;
  mplib_shade_lx   := mplib_shade_fx ;
  mplib_shade_ly   := mplib_shade_fy ;
  mplib_shade_nx   := 0 ;
  mplib_shade_ny   := 0 ;
  mplib_shade_dx   := abs(mplib_shade_fx - mplib_shade_lx) ;
  mplib_shade_dy   := abs(mplib_shade_fy - mplib_shade_ly) ;
  for i=1 upto length(p) :
    mplib_shade_tx := abs(mplib_shade_fx - xpart point i of p) ;
    mplib_shade_ty := abs(mplib_shade_fy - ypart point i of p) ;
    if mplib_shade_tx > mplib_shade_dx :
      mplib_shade_nx := i + 1 ;
      mplib_shade_lx := xpart point i of p ;
      mplib_shade_dx := mplib_shade_tx ;
    fi ;
    if mplib_shade_ty > mplib_shade_dy :
      mplib_shade_ny := i + 1 ;
      mplib_shade_ly := ypart point i of p ;
      mplib_shade_dy := mplib_shade_ty ;
    fi ;
  endfor ;
enddef;
vardef mplib_max_radius(expr p) =
  max (
    (xpart center   p - xpart llcorner p) ++ (ypart center   p - ypart llcorner p),
    (xpart center   p - xpart ulcorner p) ++ (ypart ulcorner p - ypart center   p),
    (xpart lrcorner p - xpart center   p) ++ (ypart center   p - ypart lrcorner p),
    (xpart urcorner p - xpart center   p) ++ (ypart urcorner p - ypart center   p)
  )
enddef;
def withshadingstep (text t) =
  hide(mplib_shade_step := mplib_shade_step + 1 ;)
  withprescript "sh_step=" & decimal mplib_shade_step
  t
enddef;
def withshadingradius expr a =
  withprescript "sh_radius_a=" & decimal (xpart a)
  withprescript "sh_radius_b=" & decimal (ypart a)
enddef;
def withshadingorigin expr a =
  withprescript "sh_center_a=" & ddecimal a
  withprescript "sh_center_b=" & ddecimal a
enddef;
def withshadingvector expr a =
  withprescript "sh_center_a=" & ddecimal (point xpart a of mplib_shade_path)
  withprescript "sh_center_b=" & ddecimal (point ypart a of mplib_shade_path)
enddef;
def withshadingdirection expr a =
  withprescript "sh_center_a=" & ddecimal (point xpart a of boundingbox(mplib_shade_path))
  withprescript "sh_center_b=" & ddecimal (point ypart a of boundingbox(mplib_shade_path))
enddef;
def withshadingtransform expr a =
  withprescript "sh_transform=" & a
enddef;
def withshadingcenter expr a =
  withprescript "sh_center_a=" & ddecimal (
    center mplib_shade_path shifted (
      xpart a * xpart (lrcorner mplib_shade_path - llcorner mplib_shade_path)/2,
      ypart a * ypart (urcorner mplib_shade_path - lrcorner mplib_shade_path)/2
    )
  )
enddef;
def withshadingdomain expr d =
  withprescript "sh_domain=" & ddecimal d
enddef;
def withshadingfactor expr f =
  withprescript "sh_factor=" & decimal f
enddef;
def withshadingfraction expr a =
  if mplib_shade_step > 0 :
    withprescript "sh_fraction_" & decimal mplib_shade_step & "=" & decimal a
  fi
enddef;
def withshadingcolors (expr a, b) =
  if mplib_shade_step > 0 :
    withprescript "sh_color=into"
    withprescript "sh_color_a_" & decimal mplib_shade_step & "=" & colordecimals a
    withprescript "sh_color_b_" & decimal mplib_shade_step & "=" & colordecimals b
  else :
    withprescript "sh_color=into"
    withprescript "sh_color_a=" & colordecimals a
    withprescript "sh_color_b=" & colordecimals b
  fi
enddef;
def mpliblength primary t =
  runscript("return utf8.len[===[" & t & "]===]")
enddef;
def mplibsubstring expr p of t =
  runscript("return luamplib.unicodesubstring([===[" & t & "]===],"
    & decimal xpart p & ","
    & decimal ypart p & ")")
enddef;
def mplibuclength primary t =
  runscript("return #luamplib.getunicodegraphemes[===[" & t & "]===]")
enddef;
def mplibucsubstring expr p of t =
  runscript("return luamplib.unicodesubstring([===[" & t & "]===],"
    & decimal xpart p & ","
    & decimal ypart p & ",true)")
enddef;
]],
  legacyverbatimtex = [[
def specialVerbatimTeX (text t) = runscript("luamplibprefig{"&t&"}") enddef;
def normalVerbatimTeX  (text t) = runscript("luamplibinfig{"&t&"}") enddef;
let VerbatimTeX = specialVerbatimTeX;
extra_beginfig := extra_beginfig & " let VerbatimTeX = normalVerbatimTeX;"&
  "runscript(" &ditto& "luamplib.in_the_fig=true" &ditto& ");";
extra_endfig := extra_endfig & " let VerbatimTeX = specialVerbatimTeX;"&
  "runscript(" &ditto&
  "if luamplib.in_the_fig then luamplib.figid=luamplib.figid+1 end "&
  "luamplib.in_the_fig=false" &ditto& ");";
]],
  textextlabel = [[
let luampliboriginalinfont = infont;
primarydef s infont f =
  if   (s < char 32)
    or (s = char 35) % #
    or (s = char 36) % $
    or (s = char 37) % %
    or (s = char 38) % &
    or (s = char 92) % \
    or (s = char 94) % ^
    or (s = char 95) % _
    or (s = char 123) % {
    or (s = char 125) % }
    or (s = char 126) % ~
    or (s = char 127) :
    s luampliboriginalinfont f
  else :
    rawtextext(s)
  fi
enddef;
def fontsize expr f =
  begingroup
  save size; numeric size;
  size := mplibdimen("1em");
  if size = 0: 10pt else: size fi
  endgroup
enddef;
]],
}

%    \end{macrocode}
%
%    When \cs{mplibverbatim} is enabled, do not expand |mplibcode| data.
%    \begin{macrocode}
luamplib.verbatiminput = false
%    \end{macrocode}
%
%    Do not expand |btex ... etex|, |verbatimtex ... etex|, and
%    string expressions.
%    \begin{macrocode}
local function protect_expansion (str)
  if str then
    str = str:gsub("\\","!!!Control!!!")
             :gsub("%%","!!!Comment!!!")
             :gsub("#", "!!!HashSign!!!")
             :gsub("{", "!!!LBrace!!!")
             :gsub("}", "!!!RBrace!!!")
    return format("\\unexpanded{%s}",str)
  end
end
local function unprotect_expansion (str)
  if str then
    return str:gsub("!!!Control!!!", "\\")
              :gsub("!!!Comment!!!", "%%")
              :gsub("!!!HashSign!!!","#")
              :gsub("!!!LBrace!!!",  "{")
              :gsub("!!!RBrace!!!",  "}")
  end
end
luamplib.everymplib    = setmetatable({ [""] = "" },{ __index = function(t) return t[""] end })
luamplib.everyendmplib = setmetatable({ [""] = "" },{ __index = function(t) return t[""] end })
function luamplib.process_mplibcode (data, instancename)
  texboxes.localid = 4096
%    \end{macrocode}
%    This is needed for legacy behavior
%    \begin{macrocode}
  if luamplib.legacyverbatimtex then
    luamplib.figid, tex_code_pre_mplib = 1, {}
  end
  local everymplib    = luamplib.everymplib[instancename]
  local everyendmplib = luamplib.everyendmplib[instancename]
  data = format("\n%s\n%s\n%s\n",everymplib, data, everyendmplib)
  :gsub("\r","\n")
%    \end{macrocode}
%    These five lines are needed for |mplibverbatim| mode.
%    \begin{macrocode}
  if luamplib.verbatiminput then
    data = data:gsub("\\mpcolor%s+(.-%b{})","mplibcolor(\"%1\")")
    :gsub("\\mpdim%s+(%b{})", "mplibdimen(\"%1\")")
    :gsub("\\mpdim%s+(\\%a+)","mplibdimen(\"%1\")")
    :gsub(btex_etex, "btex %1 etex ")
    :gsub(verbatimtex_etex, "verbatimtex %1 etex;")
%    \end{macrocode}
%    If not |mplibverbatim|, expand |mplibcode| data,
%    so that users can use \TeX\ codes in it.
%    It has turned out that no comment sign is allowed.
%    \begin{macrocode}
  else
    data = data:gsub(btex_etex, function(str)
      return format("btex %s etex ", protect_expansion(str)) -- space
    end)
    :gsub(verbatimtex_etex, function(str)
      return format("verbatimtex %s etex;", protect_expansion(str)) -- semicolon
    end)
    :gsub("\".-\"", protect_expansion)
    :gsub("\\%%", "\0PerCent\0")
    :gsub("%%.-\n","\n")
    :gsub("%zPerCent%z", "\\%%")
    run_tex_code(format("\\mplibtmptoks\\expandafter{\\expanded{%s}}",data))
    data = texgettoks"mplibtmptoks"
%    \end{macrocode}
%    Next line to address issue \#55
%    \begin{macrocode}
    :gsub("##", "#")
    :gsub("\".-\"", unprotect_expansion)
    :gsub(btex_etex, function(str)
      return format("btex %s etex", unprotect_expansion(str))
    end)
    :gsub(verbatimtex_etex, function(str)
      return format("verbatimtex %s etex", unprotect_expansion(str))
    end)
  end
  process(data, instancename)
end

%    \end{macrocode}
%
%    For parsing |prescript| materials.
%    \begin{macrocode}
local function script2table(s)
  local t = {}
  for _,i in ipairs(s:explode("\13+")) do
    local k,v = i:match("(.-)=(.*)") -- v may contain = or empty.
    if k and v and k ~= "" and not t[k] then
      t[k] = v
    end
  end
  return t
end

%    \end{macrocode}
%
%     pdfliterals will be stored in |figcontents| table, and written to pdf in one go
%     at the end of the flushing figure.
%     Subtable |post| is for the legacy behavior.
%    \begin{macrocode}
local figcontents = { post = { } }
local function put2output(a,...)
  figcontents[#figcontents+1] = type(a) == "string" and format(a,...) or a
end
local function pdf_startfigure(n,llx,lly,urx,ury)
  put2output("\\mplibstarttoPDF{%f}{%f}{%f}{%f}",llx,lly,urx,ury)
end
local function pdf_stopfigure()
  put2output("\\mplibstoptoPDF")
end
%    \end{macrocode}
%
%    |tex.sprint| with catcode regime -2, as sometimes |#| gets doubled
%    in the argument of pdfliteral.
%    \begin{macrocode}
local function pdf_literalcode (...)
  put2output{ -2, (format(...) :gsub(decimals,rmzeros)) }
end
local start_pdf_code = pdfmode
  and function() pdf_literalcode"q" end
  or  function() put2output"\\special{pdf:bcontent}" end
local stop_pdf_code = pdfmode
  and function() pdf_literalcode"Q" end
  or  function() put2output"\\special{pdf:econtent}" end

%    \end{macrocode}
%
%    Now we process hboxes created from |btex ... etex| or
%    |textext(...)| or |TEX(...)|, all being the same internally.
%    \begin{macrocode}
local function put_tex_boxes (object,prescript)
  local box = prescript.mplibtexboxid:explode":"
  local n,tw,th = box[1],tonumber(box[2]),tonumber(box[3])
  if n and tw and th then
    local op = object.path
    local first, second, fourth = op[1], op[2], op[4]
    local tx, ty = first.x_coord, first.y_coord
    local sx, rx, ry, sy = 1, 0, 0, 1
    if tw ~= 0 then
      sx = (second.x_coord - tx)/tw
      rx = (second.y_coord - ty)/tw
      if sx == 0 then sx = 0.00001 end
    end
    if th ~= 0 then
      sy = (fourth.y_coord - ty)/th
      ry = (fourth.x_coord - tx)/th
      if sy == 0 then sy = 0.00001 end
    end
    start_pdf_code()
    pdf_literalcode("%f %f %f %f %f %f cm",sx,rx,ry,sy,tx,ty)
    put2output("\\mplibputtextbox{%i}",n)
    stop_pdf_code()
  end
end

%    \end{macrocode}
%
%    Colors
%    \begin{macrocode}
local prev_override_color
local function do_preobj_CR(object,prescript)
  if object.postscript == "collect" then return end
  local override = prescript and prescript.mpliboverridecolor
  if override then
    if pdfmode then
      pdf_literalcode(override)
      override = nil
    else
      put2output("\\special{%s}",override)
      prev_override_color = override
    end
  else
    local cs = object.color
    if cs and #cs > 0 then
      pdf_literalcode(luamplib.colorconverter(cs))
      prev_override_color = nil
    elseif not pdfmode then
      override = prev_override_color
      if override then
        put2output("\\special{%s}",override)
      end
    end
  end
  return override
end

%    \end{macrocode}
%
%    For transparency and shading
%    \begin{macrocode}
local pdfmanagement = is_defined'pdfmanagement_add:nnn'
local pdfobjs, pdfetcs = {}, {}
pdfetcs.pgfextgs = "pgf@sys@addpdfresource@extgs@plain"
pdfetcs.pgfpattern = "pgf@sys@addpdfresource@patterns@plain"
pdfetcs.pgfcolorspace = "pgf@sys@addpdfresource@colorspaces@plain"
local function update_pdfobjs (os, stream)
  local key = os
  if stream then key = key..stream end
  local on = key and pdfobjs[key]
  if on then
    return on,false
  end
  if pdfmode then
    if stream then
      on = pdf.immediateobj("stream",stream,os)
    elseif os then
      on = pdf.immediateobj(os)
    else
      on = pdf.reserveobj()
    end
  else
    on = pdfetcs.cnt or 1
    if stream then
      texsprint(format("\\special{pdf:stream @mplibpdfobj%s (%s) <<%s>>}",on,stream,os))
    elseif os then
      texsprint(format("\\special{pdf:obj @mplibpdfobj%s %s}",on,os))
    else
      texsprint(format("\\special{pdf:obj @mplibpdfobj%s <<>>}",on))
    end
    pdfetcs.cnt = on + 1
  end
  if key then
    pdfobjs[key] = on
  end
  return on,true
end
pdfetcs.resfmt = pdfmode and "%s 0 R" or "@mplibpdfobj%s"
if pdfmode then
  pdfetcs.getpageres = pdf.getpageresources or function() return pdf.pageresources end
  local getpageres = pdfetcs.getpageres
  local setpageres = pdf.setpageresources or function(s) pdf.pageresources = s end
  local initialize_resources = function (name)
    local tabname = format("%s_res",name)
    pdfetcs[tabname] = { }
    if luatexbase.callbacktypes.finish_pdffile then -- ltluatex
      local obj = pdf.reserveobj()
      setpageres(format("%s/%s %i 0 R", getpageres() or "", name, obj))
      luatexbase.add_to_callback("finish_pdffile", function()
        pdf.immediateobj(obj, format("<<%s>>", tableconcat(pdfetcs[tabname])))
      end,
      format("luamplib.%s.finish_pdffile",name))
    end
  end
  pdfetcs.fallback_update_resources = function (name, res)
    local tabname = format("%s_res",name)
    if not pdfetcs[tabname] then
      initialize_resources(name)
    end
    if luatexbase.callbacktypes.finish_pdffile then
      local t = pdfetcs[tabname]
      t[#t+1] = res
    else
      local tpr, n = getpageres() or "", 0
      tpr, n = tpr:gsub(format("/%s<<",name), "%1"..res)
      if n == 0 then
        tpr = format("%s/%s<<%s>>", tpr, name, res)
      end
      setpageres(tpr)
    end
  end
else
  texsprint {
    "\\luamplibatfirstshipout{",
    "\\special{pdf:obj @MPlibTr<<>>}",
    "\\special{pdf:obj @MPlibSh<<>>}",
    "\\special{pdf:obj @MPlibCS<<>>}",
    "\\special{pdf:obj @MPlibPt<<>>}}",
  }
  pdfetcs.resadded = { }
  pdfetcs.fallback_update_resources = function (name,res,obj)
    texsprint{"\\special{pdf:put ", obj, " <<", res, ">>}"}
    if not pdfetcs.resadded[name] then
      texsprint{"\\luamplibateveryshipout{\\special{pdf:put @resources <</", name, " ", obj, ">>}}"}
      pdfetcs.resadded[name] = obj
    end
  end
end

%    \end{macrocode}
%
%    Transparency
%    \begin{macrocode}
local transparancy_modes = { [0] = "Normal",
  "Normal",       "Multiply",     "Screen",       "Overlay",
  "SoftLight",    "HardLight",    "ColorDodge",   "ColorBurn",
  "Darken",       "Lighten",      "Difference",   "Exclusion",
  "Hue",          "Saturation",   "Color",        "Luminosity",
  "Compatible",
  normal     = "Normal",     multiply   = "Multiply",   screen    = "Screen",
  overlay    = "Overlay",    softlight  = "SoftLight",  hardlight = "HardLight",
  colordodge = "ColorDodge", colorburn  = "ColorBurn",  darken    = "Darken",
  lighten    = "Lighten",    difference = "Difference", exclusion = "Exclusion",
  hue        = "Hue",        saturation = "Saturation", color     = "Color",
  luminosity = "Luminosity", compatible = "Compatible",
}
local function add_extgs_resources (on, new)
  local key = format("MPlibTr%s", on)
  if new then
    local val = format(pdfetcs.resfmt, on)
    if pdfmanagement then
      texsprint {
        "\\csname pdfmanagement_add:nnn\\endcsname{Page/Resources/ExtGState}{", key, "}{", val, "}"
      }
    else
      local tr = format("/%s %s", key, val)
      if is_defined(pdfetcs.pgfextgs) then
        texsprint { "\\csname ", pdfetcs.pgfextgs, "\\endcsname{", tr, "}" }
      elseif is_defined"TRP@list" then
        texsprint(catat11,{
          [[\if@filesw\immediate\write\@auxout{]],
          [[\string\g@addto@macro\string\TRP@list{]],
          tr,
          [[}}\fi]],
        })
        if not get_macro"TRP@list":find(tr) then
          texsprint(catat11,[[\global\TRP@reruntrue]])
        end
      else
        pdfetcs.fallback_update_resources("ExtGState",tr,"@MPlibTr")
      end
    end
  end
  return key
end
local function do_preobj_TR(object,prescript)
  if object.postscript == "collect" then return end
  local opaq = prescript and prescript.tr_transparency
  if opaq then
    local key, on, os, new
    local mode = prescript.tr_alternative or 1
    mode = transparancy_modes[tonumber(mode) or mode:lower()]
    if not mode then
      mode = prescript.tr_alternative
      warn("unsupported blend mode: '%s'", mode)
    end
    opaq = format("%.3f", opaq) :gsub(decimals,rmzeros)
    for i,v in ipairs{ {mode,opaq},{"Normal",1} } do
      os = format("<</BM/%s/ca %s/CA %s/AIS false>>",v[1],v[2],v[2])
      on, new = update_pdfobjs(os)
      key = add_extgs_resources(on,new)
      if i == 1 then
        pdf_literalcode("/%s gs",key)
      else
        return format("/%s gs",key)
      end
    end
  end
end

%    \end{macrocode}
%
%    Shading with \emph{metafun} format.
%    \begin{macrocode}
local function sh_pdfpageresources(shtype,domain,colorspace,ca,cb,coordinates,steps,fractions)
  for _,v in ipairs{ca,cb} do
    for i,vv in ipairs(v) do
      for ii,vvv in ipairs(vv) do
        v[i][ii] = tonumber(vvv) and format("%.3f",vvv) or vvv
      end
    end
  end
  local fun2fmt,os = "<</FunctionType 2/Domain[%s]/C0[%s]/C1[%s]/N 1>>"
  if steps > 1 then
    local list,bounds,encode = { },{ },{ }
    for i=1,steps do
      if i < steps then
        bounds[i] = format("%.3f", fractions[i] or 1)
      end
      encode[2*i-1] = 0
      encode[2*i]   = 1
      os = fun2fmt:format(domain,tableconcat(ca[i],' '),tableconcat(cb[i],' '))
        :gsub(decimals,rmzeros)
      list[i] = format(pdfetcs.resfmt, update_pdfobjs(os))
    end
    os = tableconcat {
      "<</FunctionType 3",
      format("/Bounds[%s]",    tableconcat(bounds,' ')),
      format("/Encode[%s]",    tableconcat(encode,' ')),
      format("/Functions[%s]", tableconcat(list,  ' ')),
      format("/Domain[%s]>>",  domain),
    } :gsub(decimals,rmzeros)
  else
    os = fun2fmt:format(domain,tableconcat(ca[1],' '),tableconcat(cb[1],' '))
      :gsub(decimals,rmzeros)
  end
  local objref = format(pdfetcs.resfmt, update_pdfobjs(os))
  os = tableconcat {
    format("<</ShadingType %i", shtype),
    format("/ColorSpace %s",    colorspace),
    format("/Function %s",      objref),
    format("/Coords[%s]",       coordinates),
    "/Extend[true true]/AntiAlias true>>",
  } :gsub(decimals,rmzeros)
  local on, new = update_pdfobjs(os)
  if new then
    local key, val = format("MPlibSh%s", on), format(pdfetcs.resfmt, on)
    if pdfmanagement then
      texsprint {
        "\\csname pdfmanagement_add:nnn\\endcsname{Page/Resources/Shading}{", key, "}{", val, "}"
      }
    else
      local res = format("/%s %s", key, val)
      pdfetcs.fallback_update_resources("Shading",res,"@MPlibSh")
    end
  end
  return on
end
local function color_normalize(ca,cb)
  if #cb == 1 then
    if #ca == 4 then
      cb[1], cb[2], cb[3], cb[4] = 0, 0, 0, 1-cb[1]
    else -- #ca = 3
      cb[1], cb[2], cb[3] = cb[1], cb[1], cb[1]
    end
  elseif #cb == 3 then -- #ca == 4
    cb[1], cb[2], cb[3], cb[4] = 1-cb[1], 1-cb[2], 1-cb[3], 0
  end
end
pdfetcs.clrspcs = setmetatable({ }, { __index = function(t,names)
  run_tex_code({
    [[\color_model_new:nnn]],
    format("{mplibcolorspace_%s}", names:gsub(",","_")),
    format("{DeviceN}{names={%s}}", names),
    [[\edef\mplib_@tempa{\pdf_object_ref_last:}]],
  }, ccexplat)
  local colorspace = get_macro'mplib_@tempa'
  t[names] = colorspace
  return colorspace
end })
local function do_preobj_SH(object,prescript)
  local shade_no
  local sh_type = prescript and prescript.sh_type
  if not sh_type then
    return
  else
    local domain  = prescript.sh_domain or "0 1"
    local centera = (prescript.sh_center_a or "0 0"):explode()
    local centerb = (prescript.sh_center_b or "0 0"):explode()
    local transform = prescript.sh_transform == "yes"
    local sx,sy,sr,dx,dy = 1,1,1,0,0
    if transform then
      local first = (prescript.sh_first or "0 0"):explode()
      local setx  = (prescript.sh_set_x or "0 0"):explode()
      local sety  = (prescript.sh_set_y or "0 0"):explode()
      local x,y = tonumber(setx[1]) or 0, tonumber(sety[1]) or 0
      if x ~= 0 and y ~= 0 then
        local path = object.path
        local path1x = path[1].x_coord
        local path1y = path[1].y_coord
        local path2x = path[x].x_coord
        local path2y = path[y].y_coord
        local dxa = path2x - path1x
        local dya = path2y - path1y
        local dxb = setx[2] - first[1]
        local dyb = sety[2] - first[2]
        if dxa ~= 0 and dya ~= 0 and dxb ~= 0 and dyb ~= 0 then
          sx = dxa / dxb ; if sx < 0 then sx = - sx end
          sy = dya / dyb ; if sy < 0 then sy = - sy end
          sr = math.sqrt(sx^2 + sy^2)
          dx = path1x - sx*first[1]
          dy = path1y - sy*first[2]
        end
      end
    end
    local ca, cb, colorspace, steps, fractions
    ca = { (prescript.sh_color_a_1 or prescript.sh_color_a or "0"):explode":" }
    cb = { (prescript.sh_color_b_1 or prescript.sh_color_b or "1"):explode":" }
    steps = tonumber(prescript.sh_step) or 1
    if steps > 1 then
      fractions = { prescript.sh_fraction_1 or 0 }
      for i=2,steps do
        fractions[i] = prescript[format("sh_fraction_%i",i)] or (i/steps)
        ca[i] = (prescript[format("sh_color_a_%i",i)] or "0"):explode":"
        cb[i] = (prescript[format("sh_color_b_%i",i)] or "1"):explode":"
      end
    end
    if prescript.mplib_spotcolor then
      ca, cb = { }, { }
      local names, pos, objref = { }, -1, ""
      local script = object.prescript:explode"\13+"
      for i=#script,1,-1 do
        if script[i]:find"mplib_spotcolor" then
          local t, name, value = script[i]:explode"="[2]:explode":"
          value, objref, name = t[1], t[2], t[3]
          if not names[name] then
            pos = pos+1
            names[name] = pos
            names[#names+1] = name
          end
          t = { }
          for j=1,names[name] do t[#t+1] = 0 end
          t[#t+1] = value
          tableinsert(#ca == #cb and ca or cb, t)
        end
      end
      for _,t in ipairs{ca,cb} do
        for _,tt in ipairs(t) do
          for i=1,#names-#tt do tt[#tt+1] = 0 end
        end
      end
      if #names == 1 then
        colorspace = objref
      else
        colorspace = pdfetcs.clrspcs[ tableconcat(names,",") ]
      end
    else
      local model = 0
      for _,t in ipairs{ca,cb} do
        for _,tt in ipairs(t) do
          model = model > #tt and model or #tt
        end
      end
      for _,t in ipairs{ca,cb} do
        for _,tt in ipairs(t) do
          if #tt < model then
            color_normalize(model == 4 and {1,1,1,1} or {1,1,1},tt)
          end
        end
      end
      colorspace = model == 4 and "/DeviceCMYK"
                or model == 3 and "/DeviceRGB"
                or model == 1 and "/DeviceGray"
                or err"unknown color model"
    end
    if sh_type == "linear" then
      local coordinates = format("%f %f %f %f",
        dx + sx*centera[1], dy + sy*centera[2],
        dx + sx*centerb[1], dy + sy*centerb[2])
      shade_no = sh_pdfpageresources(2,domain,colorspace,ca,cb,coordinates,steps,fractions)
    elseif sh_type == "circular" then
      local factor = prescript.sh_factor or 1
      local radiusa = factor * prescript.sh_radius_a
      local radiusb = factor * prescript.sh_radius_b
      local coordinates = format("%f %f %f %f %f %f",
        dx + sx*centera[1], dy + sy*centera[2], sr*radiusa,
        dx + sx*centerb[1], dy + sy*centerb[2], sr*radiusb)
      shade_no = sh_pdfpageresources(3,domain,colorspace,ca,cb,coordinates,steps,fractions)
    else
      err"unknown shading type"
    end
  end
  return shade_no
end

%    \end{macrocode}
%
% Shading Patterns: much similar to the metafun's shade,
% but we can apply shading to textual pictures as well as paths.
%    \begin{macrocode}
if not pdfmode then
  pdfetcs.patternresources = {}
end
local function add_pattern_resources (key, val)
  if pdfmanagement then
    texsprint {
      "\\csname pdfmanagement_add:nnn\\endcsname{Page/Resources/Pattern}{", key, "}{", val, "}"
    }
  else
    local res = format("/%s %s", key, val)
    if is_defined(pdfetcs.pgfpattern) then
      texsprint { "\\csname ", pdfetcs.pgfpattern, "\\endcsname{", res, "}" }
    else
      pdfetcs.fallback_update_resources("Pattern",res,"@MPlibPt")
      if not pdfmode then
        tableinsert(pdfetcs.patternresources, res) -- for gather_resources()
      end
    end
  end
end
function luamplib.dolatelua (on, os)
  local h, v = pdf.getpos()
  h = format("%f", h/factor) :gsub(decimals,rmzeros)
  v = format("%f", v/factor) :gsub(decimals,rmzeros)
  if pdfmode then
    pdf.obj(on, format("<<%s/Matrix[1 0 0 1 %s %s]>>", os, h, v))
    pdf.refobj(on)
  else
    local shift = os:explode()
    if tonumber(h) ~= tonumber(shift[1]) or tonumber(v) ~= tonumber(shift[2]) then
      warn([[Add 'withprescript "sh_matrixshift=%s %s"' to the picture shading]], h, v)
    end
  end
end
local function do_preobj_shading (object, prescript)
  if not prescript or not prescript.sh_operand_type then return end
  local on = do_preobj_SH(object, prescript)
  local os = format("/PatternType 2/Shading %s", format(pdfetcs.resfmt, on))
  on = update_pdfobjs()
  if pdfmode then
    put2output(tableconcat{ "\\latelua{ luamplib.dolatelua(",on,",[[",os,"]]) }" })
  else
%    \end{macrocode}
% Why |@xpos| |@ypos| do not work properly???\\
% Anyway, this seems to be needed for proper functioning:
%\begin{verbatim}
%     \pagewidth=\paperwidth
%     \pageheight=\paperheight
%     \special{papersize=\the\paperwidth,\the\paperheight}
%\end{verbatim}
%    \begin{macrocode}
    if is_defined"RecordProperties" then
      put2output(tableconcat{
        "\\csname tex_savepos:D\\endcsname\\RecordProperties{luamplib/getpos/",on,"}{xpos,ypos}\z
        \\special{pdf:put @mplibpdfobj",on," <<",os,"/Matrix[1 0 0 1 \z
        \\csname dim_to_decimal_in_bp:n\\endcsname{\\RefProperty{luamplib/getpos/",on,"}{xpos}sp} \z
        \\csname dim_to_decimal_in_bp:n\\endcsname{\\RefProperty{luamplib/getpos/",on,"}{ypos}sp}\z
        ]>>}"
      })
    else
      local shift = prescript.sh_matrixshift or "0 0"
      texsprint{ "\\special{pdf:put @mplibpdfobj",on," <<",os,"/Matrix[1 0 0 1 ",shift,"]>>}" }
      put2output(tableconcat{ "\\latelua{ luamplib.dolatelua(",on,",[[",shift,"]]) }" })
    end
  end
  local key, val = format("MPlibPt%s", on), format(pdfetcs.resfmt, on)
  add_pattern_resources(key,val)
  pdf_literalcode("/Pattern cs/%s scn", key)
%    \end{macrocode}
% To avoid possible double execution, once by Pattern gs, once by Sh operator.
%    \begin{macrocode}
  prescript.sh_type = nil
end

%    \end{macrocode}
%
%    Tiling Patterns
%    \begin{macrocode}
pdfetcs.patterns = { }
local function gather_resources (optres)
  local t, do_pattern = { }, not optres
  local names = {"ExtGState","ColorSpace","Shading"}
  if do_pattern then
    names[#names+1] = "Pattern"
  end
  if pdfmode then
    if pdfmanagement then
      for _,v in ipairs(names) do
        if ltx.__pdf.Page.Resources[v] then
          t[#t+1] = format("/%s %s 0 R", v, ltx.pdf.object_id("__pdf/Page/Resources/"..v))
        end
      end
    else
      local res = pdfetcs.getpageres() or ""
      run_tex_code[[\mplibtmptoks\expandafter{\the\pdfvariable pageresources}]]
      res = res .. texgettoks'mplibtmptoks'
      if do_pattern then return res end
      res = res:explode"/+"
      for _,v in ipairs(res) do
        v = v:match"^%s*(.-)%s*$"
        if not v:find"Pattern" and not optres:find(v) then
          t[#t+1] = "/" .. v
        end
      end
    end
  else
    if pdfmanagement then
      for _,v in ipairs(names) do
        run_tex_code ({
          "\\mplibtmptoks\\expanded{{",
          "\\pdfdict_if_empty:nF{g__pdf_Core/Page/Resources/", v, "}",
          "{/", v, " \\pdf_object_ref:n{__pdf/Page/Resources/", v, "}}}}",
        },ccexplat)
        t[#t+1] = texgettoks'mplibtmptoks'
      end
    elseif is_defined(pdfetcs.pgfextgs) then
      run_tex_code ({
        "\\mplibtmptoks\\expanded{{",
        "\\ifpgf@sys@pdf@extgs@exists /ExtGState @pgfextgs\\fi",
        "\\ifpgf@sys@pdf@colorspaces@exists /ColorSpace @pgfcolorspaces\\fi",
        do_pattern and "\\ifpgf@sys@pdf@patterns@exists /Pattern @pgfpatterns \\fi" or "",
        "}}",
      }, catat11)
      t[#t+1] = texgettoks'mplibtmptoks'
      if pdfetcs.resadded.Shading then
        t[#t+1] = format("/Shading %s", pdfetcs.resadded.Shading)
      end
    else
      for _,v in ipairs(names) do
        local vv = pdfetcs.resadded[v]
        if vv then
          t[#t+1] = format("/%s %s", v, vv)
        end
      end
    end
  end
  if do_pattern then return tableconcat(t) end
  -- get pattern resources
  local mytoks
  if pdfmanagement then
    run_tex_code ({
      "\\mplibtmptoks\\expanded{{",
      "\\pdfdict_if_empty:nF{g__pdf_Core/Page/Resources/Pattern}",
      "{\\pdfdict_use:n{g__pdf_Core/Page/Resources/Pattern}}", "}}",
    },ccexplat)
    mytoks = texgettoks"mplibtmptoks"
    if not pdfmode then
      mytoks = mytoks:gsub("\\str_convert_pdfname:n%s*{(.-)}","%1") -- why not expanded?
    end
  elseif is_defined(pdfetcs.pgfextgs) then
    if pdfmode then
      mytoks = get_macro"pgf@sys@pgf@resource@list@patterns"
    else
      local tt, abc = {}, get_macro"pgfutil@abc" or ""
      for v in abc:gmatch"@pgfpatterns%s*<<(.-)>>" do
        tt[#tt+1] = v
      end
      mytoks = tableconcat(tt)
    end
  else
    local tt = pdfmode and pdfetcs.Pattern_res or pdfetcs.patternresources
    mytoks = tt and tableconcat(tt)
  end
  if mytoks and mytoks ~= "" then
    t[#t+1] = format("/Pattern<<%s>>",mytoks)
  end
  return tableconcat(t)
end
function luamplib.registerpattern ( boxid, name, opts )
  local box = texgetbox(boxid)
  local wd = format("%.3f",box.width/factor)
  local hd = format("%.3f",(box.height+box.depth)/factor)
  info("w/h/d of pattern '%s': %s 0", name, format("%s %s",wd, hd):gsub(decimals,rmzeros))
  if opts.xstep == 0 then opts.xstep = nil end
  if opts.ystep == 0 then opts.ystep = nil end
  if opts.colored == nil then
    opts.colored = opts.coloured
    if opts.colored == nil then
      opts.colored = true
    end
  end
  if type(opts.matrix) == "table" then opts.matrix = tableconcat(opts.matrix," ") end
  if type(opts.bbox) == "table" then opts.bbox = tableconcat(opts.bbox," ") end
  if opts.matrix and opts.matrix:find"%a" then
    local data = format("mplibtransformmatrix(%s);",opts.matrix)
    process(data,"@mplibtransformmatrix")
    local t = luamplib.transformmatrix
    opts.matrix = format("%f %f %f %f", t[1], t[2], t[3], t[4])
    opts.xshift = opts.xshift or format("%f",t[5])
    opts.yshift = opts.yshift or format("%f",t[6])
  end
  local attr = {
    "/Type/Pattern",
    "/PatternType 1",
    format("/PaintType %i", opts.colored and 1 or 2),
    "/TilingType 2",
    format("/XStep %s", opts.xstep or wd),
    format("/YStep %s", opts.ystep or hd),
    format("/Matrix[%s %s %s]", opts.matrix or "1 0 0 1", opts.xshift or 0, opts.yshift or 0),
  }
  local optres = opts.resources or ""
  optres = optres .. gather_resources(optres)
  local patterns = pdfetcs.patterns
  if pdfmode then
    if opts.bbox then
      attr[#attr+1] = format("/BBox[%s]", opts.bbox)
    end
    attr = tableconcat(attr) :gsub(decimals,rmzeros)
    local index = tex.saveboxresource(boxid, attr, optres, true, opts.bbox and 4 or 1)
    patterns[name] = { id = index, colored = opts.colored }
  else
    local cnt = #patterns + 1
    local objname = "@mplibpattern" .. cnt
    local metric = format("bbox %s", opts.bbox or format("0 0 %s %s",wd,hd))
    texsprint {
      "\\expandafter\\newbox\\csname luamplib.patternbox.", cnt, "\\endcsname",
      "\\global\\setbox\\csname luamplib.patternbox.", cnt, "\\endcsname",
      "\\hbox{\\unhbox ", boxid, "}\\luamplibatnextshipout{",
      "\\special{pdf:bcontent}",
      "\\special{pdf:bxobj ", objname, " ", metric, "}",
      "\\raise\\dp\\csname luamplib.patternbox.", cnt, "\\endcsname",
      "\\box\\csname luamplib.patternbox.", cnt, "\\endcsname",
      "\\special{pdf:put @resources <<", optres, ">>}",
      "\\special{pdf:exobj <<", tableconcat(attr), ">>}",
      "\\special{pdf:econtent}}",
    }
    patterns[cnt] = objname
    patterns[name] = { id = cnt, colored = opts.colored }
  end
end
local function pattern_colorspace (cs)
  local on, new = update_pdfobjs(format("[/Pattern %s]", cs))
  if new then
    local key, val = format("MPlibCS%i",on), format(pdfetcs.resfmt,on)
    if pdfmanagement then
      texsprint {
        "\\csname pdfmanagement_add:nnn\\endcsname{Page/Resources/ColorSpace}{", key, "}{", val, "}"
      }
    else
      local res = format("/%s %s", key, val)
      if is_defined(pdfetcs.pgfcolorspace) then
        texsprint { "\\csname ", pdfetcs.pgfcolorspace, "\\endcsname{", res, "}" }
      else
        pdfetcs.fallback_update_resources("ColorSpace",res,"@MPlibCS")
      end
    end
  end
  return on
end
local function do_preobj_PAT(object, prescript)
  local name = prescript and prescript.mplibpattern
  if not name then return end
  local patterns = pdfetcs.patterns
  local patt = patterns[name]
  local index = patt and patt.id or err("cannot get pattern object '%s'", name)
  local key = format("MPlibPt%s",index)
  if patt.colored then
    pdf_literalcode("/Pattern cs /%s scn", key)
  else
    local color = prescript.mpliboverridecolor
    if not color then
      local t = object.color
      color = t and #t>0 and luamplib.colorconverter(t)
    end
    if not color then return end
    local cs
    if color:find" cs " or color:find"@pdf.obj" then
      local t = color:explode()
      if pdfmode then
        cs = format("%s 0 R", ltx.pdf.object_id( t[1]:sub(2,-1) ))
        color = t[3]
      else
        cs = t[2]
        color = t[3]:match"%[(.+)%]"
      end
    else
      local t = colorsplit(color)
      cs = #t == 4 and "/DeviceCMYK" or #t == 3 and "/DeviceRGB" or "/DeviceGray"
      color = tableconcat(t," ")
    end
    pdf_literalcode("/MPlibCS%i cs %s /%s scn", pattern_colorspace(cs), color, key)
  end
  if not patt.done then
    local val = pdfmode and format("%s 0 R",index) or patterns[index]
    add_pattern_resources(key,val)
  end
  patt.done = true
end

%    \end{macrocode}
%
%    Fading
%    \begin{macrocode}
pdfetcs.fading = { }
local function do_preobj_FADE (object, prescript)
  local fd_type = prescript and prescript.mplibfadetype
  local fd_stop = prescript and prescript.mplibfadestate
  if not fd_type then
    return fd_stop -- returns "stop" (if picture) or nil
  end
  local bbox = prescript.mplibfadebbox:explode":"
  local dx, dy = -bbox[1], -bbox[2]
  local vec = prescript.mplibfadevector; vec = vec and vec:explode":"
  if not vec then
    if fd_type == "linear" then
      vec = {bbox[1], bbox[2], bbox[3], bbox[2]} -- left to right
    else
      local centerx, centery = (bbox[1]+bbox[3])/2, (bbox[2]+bbox[4])/2
      vec = {centerx, centery, centerx, centery} -- center for both circles
    end
  end
  local coords = { vec[1]+dx, vec[2]+dy, vec[3]+dx, vec[4]+dy }
  if fd_type == "linear" then
    coords = format("%f %f %f %f", tableunpack(coords))
  elseif fd_type == "circular" then
    local width, height = bbox[3]-bbox[1], bbox[4]-bbox[2]
    local radius = (prescript.mplibfaderadius or "0:"..math.sqrt(width^2+height^2)/2):explode":"
    tableinsert(coords, 3, radius[1])
    tableinsert(coords, radius[2])
    coords = format("%f %f %f %f %f %f", tableunpack(coords))
  else
    err("unknown fading method '%s'", fd_type)
  end
  fd_type = fd_type == "linear" and 2 or 3
  local opaq = (prescript.mplibfadeopacity or "1:0"):explode":"
  local on, os, new
  on = sh_pdfpageresources(fd_type, "0 1", "/DeviceGray", {{opaq[1]}}, {{opaq[2]}}, coords, 1)
  os = format("<</PatternType 2/Shading %s>>", format(pdfetcs.resfmt, on))
  on = update_pdfobjs(os)
  bbox = format("0 0 %f %f", bbox[3]+dx, bbox[4]+dy)
  local streamtext = format("q /Pattern cs/MPlibFd%s scn %s re f Q", on, bbox)
    :gsub(decimals,rmzeros)
  os = format("<</Pattern<</MPlibFd%s %s>>>>", on, format(pdfetcs.resfmt, on))
  on = update_pdfobjs(os)
  local resources = format(pdfetcs.resfmt, on)
  on = update_pdfobjs"<</S/Transparency/CS/DeviceGray>>"
  local attr = tableconcat{
    "/Subtype/Form",
    "/BBox[", bbox, "]",
    "/Matrix[1 0 0 1 ", format("%f %f", -dx,-dy), "]",
    "/Resources ", resources,
    "/Group ", format(pdfetcs.resfmt, on),
  } :gsub(decimals,rmzeros)
  on = update_pdfobjs(attr, streamtext)
  os = "<</SMask<</S/Luminosity/G " .. format(pdfetcs.resfmt, on) .. ">>>>"
  on, new = update_pdfobjs(os)
  local key = add_extgs_resources(on,new)
  start_pdf_code()
  pdf_literalcode("/%s gs", key)
  if fd_stop then return "standalone" end
  return "start"
end

%    \end{macrocode}
%
%    Transparency Group
%    \begin{macrocode}
pdfetcs.tr_group = { shifts = { } }
luamplib.trgroupshifts = pdfetcs.tr_group.shifts
local function do_preobj_GRP (object, prescript)
  local grstate = prescript and prescript.gr_state
  if not grstate then return end
  local trgroup = pdfetcs.tr_group
  if grstate == "start" then
    trgroup.name = prescript.mplibgroupname or "lastmplibgroup"
    trgroup.isolated, trgroup.knockout = false, false
    for _,v in ipairs(prescript.gr_type:explode",+") do
      trgroup[v] = true
    end
    trgroup.bbox = prescript.mplibgroupbbox:explode":"
    put2output[[\begingroup\setbox\mplibscratchbox\hbox\bgroup\luamplibtagasgroupinit]]
  elseif grstate == "stop" then
    local llx,lly,urx,ury = tableunpack(trgroup.bbox)
    put2output(tableconcat{
      "\\egroup",
      format("\\wd\\mplibscratchbox %fbp", urx-llx),
      format("\\ht\\mplibscratchbox %fbp", ury-lly),
      "\\dp\\mplibscratchbox 0pt",
    })
    local grattr = format("/Group<</S/Transparency/I %s/K %s>>",trgroup.isolated,trgroup.knockout)
    local res = gather_resources()
    local bbox = format("%f %f %f %f", llx,lly,urx,ury) :gsub(decimals,rmzeros)
    if pdfmode then
      put2output(tableconcat{
        "\\saveboxresource type 2 attr{/Type/XObject/Subtype/Form/FormType 1",
        "/BBox[", bbox, "]", grattr, "} resources{", res, "}\\mplibscratchbox",
        "\\luamplibtagasgroupbegin",
        [[\setbox\mplibscratchbox\hbox{\useboxresource\lastsavedboxresourceindex}]],
        [[\wd\mplibscratchbox 0pt\ht\mplibscratchbox 0pt\dp\mplibscratchbox 0pt]],
        [[\box\mplibscratchbox]],
        "\\luamplibtagasgroupend",
        "\\endgroup",
        "\\expandafter\\xdef\\csname luamplib.group.", trgroup.name, "\\endcsname{",
        "\\setbox\\mplibscratchbox\\hbox{\\hskip",-llx,"bp\\raise",-lly,"bp\\hbox{",
        "\\useboxresource \\the\\lastsavedboxresourceindex",
        "}}\\wd\\mplibscratchbox",urx-llx,"bp\\ht\\mplibscratchbox",ury-lly,"bp",
        "\\box\\mplibscratchbox}",
      })
    else
      trgroup.cnt = (trgroup.cnt or 0) + 1
      local objname = format("@mplibtrgr%s", trgroup.cnt)
      put2output(tableconcat{
        "\\special{pdf:bxobj ", objname, " bbox ", bbox, "}",
        "\\unhbox\\mplibscratchbox",
        "\\special{pdf:put @resources <<", res, ">>}",
        "\\special{pdf:exobj <<", grattr, ">>}",
        "\\special{pdf:uxobj ", objname, "}",
        "\\endgroup",
      })
      token.set_macro("luamplib.group."..trgroup.name, tableconcat{
        "\\setbox\\mplibscratchbox\\hbox{\\hskip",-llx,"bp\\raise",-lly,"bp\\hbox{",
        "\\special{pdf:uxobj ", objname, "}",
        "}}\\wd\\mplibscratchbox",urx-llx,"bp\\ht\\mplibscratchbox",ury-lly,"bp",
        "\\box\\mplibscratchbox",
      }, "global")
    end
    trgroup.shifts[trgroup.name] = { llx, lly }
  end
  return grstate
end
function luamplib.registergroup (boxid, name, opts)
  local box = texgetbox(boxid)
  local wd, ht, dp = node.getwhd(box)
  local res = (opts.resources or "") .. gather_resources()
  local attr = { "/Type/XObject/Subtype/Form/FormType 1" }
  if type(opts.matrix) == "table" then opts.matrix = tableconcat(opts.matrix," ") end
  if type(opts.bbox) == "table" then opts.bbox = tableconcat(opts.bbox," ") end
  if opts.matrix and opts.matrix:find"%a" then
    local data = format("mplibtransformmatrix(%s);",opts.matrix)
    process(data,"@mplibtransformmatrix")
    opts.matrix = format("%f %f %f %f %f %f",tableunpack(luamplib.transformmatrix))
  end
  local grtype = 3
  if opts.bbox then
    attr[#attr+1] = format("/BBox[%s]", opts.bbox)
    grtype = 2
  end
  if opts.matrix then
    attr[#attr+1] = format("/Matrix[%s]", opts.matrix)
    grtype = opts.bbox and 4 or 1
  end
  if opts.asgroup then
    local t = { isolated = false, knockout = false }
    for _,v in ipairs(opts.asgroup:explode",+") do t[v] = true end
    attr[#attr+1] = format("/Group<</S/Transparency/I %s/K %s>>", t.isolated, t.knockout)
  end
  local trgroup = pdfetcs.tr_group
  trgroup.shifts[name] = { get_macro'MPllx', get_macro'MPlly' }
  local whd
  if pdfmode then
    attr = tableconcat(attr) :gsub(decimals,rmzeros)
    local index = tex.saveboxresource(boxid, attr, res, true, grtype)
    token.set_macro("luamplib.group."..name, tableconcat{
      "\\useboxresource ", index,
    }, "global")
    whd = format("%.3f %.3f 0", wd/factor, (ht+dp)/factor) :gsub(decimals,rmzeros)
  else
    trgroup.cnt = (trgroup.cnt or 0) + 1
    local objname = format("@mplibtrgr%s", trgroup.cnt)
    texsprint {
      "\\expandafter\\newbox\\csname luamplib.groupbox.", trgroup.cnt, "\\endcsname",
      "\\global\\setbox\\csname luamplib.groupbox.", trgroup.cnt, "\\endcsname",
      "\\hbox{\\unhbox ", boxid, "}\\luamplibatnextshipout{",
      "\\special{pdf:bcontent}",
      "\\special{pdf:bxobj ", objname, " width ", wd, "sp height ", ht, "sp depth ", dp, "sp}",
      "\\unhbox\\csname luamplib.groupbox.", trgroup.cnt, "\\endcsname",
      "\\special{pdf:put @resources <<", res, ">>}",
      "\\special{pdf:exobj <<", tableconcat(attr), ">>}",
      "\\special{pdf:econtent}}",
    }
    token.set_macro("luamplib.group."..name, tableconcat{
      "\\setbox\\mplibscratchbox\\hbox{\\special{pdf:uxobj ", objname, "}}",
      "\\wd\\mplibscratchbox ", wd, "sp",
      "\\ht\\mplibscratchbox ", ht, "sp",
      "\\dp\\mplibscratchbox ", dp, "sp",
      "\\box\\mplibscratchbox",
    }, "global")
    whd = format("%.3f %.3f %.3f", wd/factor, ht/factor, dp/factor) :gsub(decimals,rmzeros)
  end
  info("w/h/d of group '%s': %s", name, whd)
end

local function stop_special_effects(fade,opaq,over)
  if fade then -- fading
    stop_pdf_code()
  end
  if opaq then -- opacity
    pdf_literalcode(opaq)
  end
  if over then -- color
    put2output"\\special{pdf:ec}"
  end
end

%    \end{macrocode}
%
%    Codes below for inserting PDF lieterals are mostly from ConTeXt general,
%    with small changes when needed.
%    \begin{macrocode}
local function getobjects(result,figure,f)
  return figure:objects()
end

function luamplib.convert (result, flusher)
  luamplib.flush(result, flusher)
  return true -- done
end

local function pdf_textfigure(font,size,text,width,height,depth)
  text = text:gsub(".",function(c)
    return format("\\hbox{\\char%i}",string.byte(c)) -- kerning happens in metapost : false
  end)
  put2output("\\mplibtextext{%s}{%f}{%s}{%s}{%s}",font,size,text,0,0)
end

local bend_tolerance = 131/65536

local rx, sx, sy, ry, tx, ty, divider = 1, 0, 0, 1, 0, 0, 1

local function pen_characteristics(object)
  local t = mplib.pen_info(object)
  rx, ry, sx, sy, tx, ty = t.rx, t.ry, t.sx, t.sy, t.tx, t.ty
  divider = sx*sy - rx*ry
  return not (sx==1 and rx==0 and ry==0 and sy==1 and tx==0 and ty==0), t.width
end

local function concat(px, py) -- no tx, ty here
  return (sy*px-ry*py)/divider,(sx*py-rx*px)/divider
end

local function curved(ith,pth)
  local d = pth.left_x - ith.right_x
  if abs(ith.right_x - ith.x_coord - d) <= bend_tolerance and abs(pth.x_coord - pth.left_x - d) <= bend_tolerance then
    d = pth.left_y - ith.right_y
    if abs(ith.right_y - ith.y_coord - d) <= bend_tolerance and abs(pth.y_coord - pth.left_y - d) <= bend_tolerance then
      return false
    end
  end
  return true
end

local function flushnormalpath(path,open)
  local pth, ith
  for i=1,#path do
    pth = path[i]
    if not ith then
      pdf_literalcode("%f %f m",pth.x_coord,pth.y_coord)
    elseif curved(ith,pth) then
      pdf_literalcode("%f %f %f %f %f %f c",ith.right_x,ith.right_y,pth.left_x,pth.left_y,pth.x_coord,pth.y_coord)
    else
      pdf_literalcode("%f %f l",pth.x_coord,pth.y_coord)
    end
    ith = pth
  end
  if not open then
    local one = path[1]
    if curved(pth,one) then
      pdf_literalcode("%f %f %f %f %f %f c",pth.right_x,pth.right_y,one.left_x,one.left_y,one.x_coord,one.y_coord )
    else
      pdf_literalcode("%f %f l",one.x_coord,one.y_coord)
    end
  elseif #path == 1 then -- special case .. draw point
    local one = path[1]
    pdf_literalcode("%f %f l",one.x_coord,one.y_coord)
  end
end

local function flushconcatpath(path,open)
  pdf_literalcode("%f %f %f %f %f %f cm", sx, rx, ry, sy, tx ,ty)
  local pth, ith
  for i=1,#path do
    pth = path[i]
    if not ith then
      pdf_literalcode("%f %f m",concat(pth.x_coord,pth.y_coord))
    elseif curved(ith,pth) then
      local a, b = concat(ith.right_x,ith.right_y)
      local c, d = concat(pth.left_x,pth.left_y)
      pdf_literalcode("%f %f %f %f %f %f c",a,b,c,d,concat(pth.x_coord, pth.y_coord))
    else
      pdf_literalcode("%f %f l",concat(pth.x_coord, pth.y_coord))
    end
    ith = pth
  end
  if not open then
    local one = path[1]
    if curved(pth,one) then
      local a, b = concat(pth.right_x,pth.right_y)
      local c, d = concat(one.left_x,one.left_y)
      pdf_literalcode("%f %f %f %f %f %f c",a,b,c,d,concat(one.x_coord, one.y_coord))
    else
      pdf_literalcode("%f %f l",concat(one.x_coord,one.y_coord))
    end
  elseif #path == 1 then -- special case .. draw point
    local one = path[1]
    pdf_literalcode("%f %f l",concat(one.x_coord,one.y_coord))
  end
end

%    \end{macrocode}
%
%    Finally, flush figures by inserting PDF literals.
%    \begin{macrocode}
function luamplib.flush (result,flusher)
  if result then
    local figures = result.fig
    if figures then
      for f=1, #figures do
        info("flushing figure %s",f)
        local figure = figures[f]
        local objects = getobjects(result,figure,f)
        local fignum = tonumber(figure:filename():match("([%d]+)$") or figure:charcode() or 0)
        local miterlimit, linecap, linejoin, dashed = -1, -1, -1, false
        local bbox = figure:boundingbox()
        local llx, lly, urx, ury = bbox[1], bbox[2], bbox[3], bbox[4] -- faster than unpack
        if urx < llx then
%    \end{macrocode}
%    luamplib silently ignores this invalid figure for those
%    that do not contain |beginfig ... endfig|. (issue \#70)
%    Original code of ConTeXt general was:
%\begin{verbatim}
%    -- invalid
%    pdf_startfigure(fignum,0,0,0,0)
%    pdf_stopfigure()
%\end{verbatim}
%    \begin{macrocode}
        else
%    \end{macrocode}
%    For legacy behavior, insert `pre-fig' \TeX\ code here.
%    \begin{macrocode}
          if tex_code_pre_mplib[f] then
            put2output(tex_code_pre_mplib[f])
          end
          pdf_startfigure(fignum,llx,lly,urx,ury)
          start_pdf_code()
          if objects then
            local savedpath = nil
            local savedhtap = nil
            for o=1,#objects do
              local object        = objects[o]
              local objecttype    = object.type
%    \end{macrocode}
%    The following 10 lines are part of |btex...etex| patch.
%    Again, colors are processed at this stage.
%    \begin{macrocode}
              local prescript     = object.prescript
              prescript = prescript and script2table(prescript) -- prescript is now a table
              local cr_over = do_preobj_CR(object,prescript) -- color
              local tr_opaq = do_preobj_TR(object,prescript) -- opacity
              local fading_ = do_preobj_FADE(object,prescript) -- fading
              local trgroup = do_preobj_GRP(object,prescript) -- transparency group
              local pattern_ = do_preobj_PAT(object,prescript) -- tiling pattern
              local shading_ = do_preobj_shading(object,prescript) -- shading pattern
              if prescript and prescript.mplibtexboxid then
                put_tex_boxes(object,prescript)
              elseif objecttype == "start_bounds" or objecttype == "stop_bounds" then --skip
              elseif objecttype == "start_clip" then
                local evenodd = not object.istext and object.postscript == "evenodd"
                start_pdf_code()
                flushnormalpath(object.path,false)
                pdf_literalcode(evenodd and "W* n" or "W n")
              elseif objecttype == "stop_clip" then
                stop_pdf_code()
                miterlimit, linecap, linejoin, dashed = -1, -1, -1, false
              elseif objecttype == "special" then
%    \end{macrocode}
%    Collect \TeX\ codes that will be executed after flushing.
%    Legacy behavior.
%    \begin{macrocode}
                if prescript and prescript.postmplibverbtex then
                  figcontents.post[#figcontents.post+1] = prescript.postmplibverbtex
                end
              elseif objecttype == "text" then
                local ot = object.transform -- 3,4,5,6,1,2
                start_pdf_code()
                pdf_literalcode("%f %f %f %f %f %f cm",ot[3],ot[4],ot[5],ot[6],ot[1],ot[2])
                pdf_textfigure(object.font,object.dsize,object.text,object.width,object.height,object.depth)
                stop_pdf_code()
              elseif not trgroup and fading_ ~= "stop" then
                local evenodd, collect, both = false, false, false
                local postscript = object.postscript
                if not object.istext then
                  if postscript == "evenodd" then
                    evenodd = true
                  elseif postscript == "collect" then
                    collect = true
                  elseif postscript == "both" then
                    both = true
                  elseif postscript == "eoboth" then
                    evenodd = true
                    both    = true
                  end
                end
                if collect then
                  if not savedpath then
                    savedpath = { object.path or false }
                    savedhtap = { object.htap or false }
                  else
                    savedpath[#savedpath+1] = object.path or false
                    savedhtap[#savedhtap+1] = object.htap or false
                  end
                else
%    \end{macrocode}
%    Removed from ConTeXt general: color stuff.
%    \begin{macrocode}
                  local ml = object.miterlimit
                  if ml and ml ~= miterlimit then
                    miterlimit = ml
                    pdf_literalcode("%f M",ml)
                  end
                  local lj = object.linejoin
                  if lj and lj ~= linejoin then
                    linejoin = lj
                    pdf_literalcode("%i j",lj)
                  end
                  local lc = object.linecap
                  if lc and lc ~= linecap then
                    linecap = lc
                    pdf_literalcode("%i J",lc)
                  end
                  local dl = object.dash
                  if dl then
                    local d = format("[%s] %f d",tableconcat(dl.dashes or {}," "),dl.offset)
                    if d ~= dashed then
                      dashed = d
                      pdf_literalcode(dashed)
                    end
                  elseif dashed then
                    pdf_literalcode("[] 0 d")
                    dashed = false
                  end
                  local path = object.path
                  local transformed, penwidth = false, 1
                  local open = path and path[1].left_type and path[#path].right_type
                  local pen = object.pen
                  if pen then
                    if pen.type == 'elliptical' then
                      transformed, penwidth = pen_characteristics(object) -- boolean, value
                      pdf_literalcode("%f w",penwidth)
                      if objecttype == 'fill' then
                        objecttype = 'both'
                      end
                    else -- calculated by mplib itself
                      objecttype = 'fill'
                    end
                  end
%    \end{macrocode}
%    Added : shading
%    \begin{macrocode}
                  local shade_no = do_preobj_SH(object,prescript) -- shading
                  if shade_no then
                    pdf_literalcode"q /Pattern cs"
                    objecttype = false
                  end
                  if transformed then
                    start_pdf_code()
                  end
                  if path then
                    if savedpath then
                      for i=1,#savedpath do
                        local path = savedpath[i]
                        if transformed then
                          flushconcatpath(path,open)
                        else
                          flushnormalpath(path,open)
                        end
                      end
                      savedpath = nil
                    end
                    if transformed then
                      flushconcatpath(path,open)
                    else
                      flushnormalpath(path,open)
                    end
                    if objecttype == "fill" then
                      pdf_literalcode(evenodd and "h f*" or "h f")
                    elseif objecttype == "outline" then
                      if both then
                        pdf_literalcode(evenodd and "h B*" or "h B")
                      else
                        pdf_literalcode(open and "S" or "h S")
                      end
                    elseif objecttype == "both" then
                      pdf_literalcode(evenodd and "h B*" or "h B")
                    end
                  end
                  if transformed then
                    stop_pdf_code()
                  end
                  local path = object.htap
%    \end{macrocode}
%    How can we generate an |htap| object? Please let us know if you have succeeded.
%    \begin{macrocode}
                  if path then
                    if transformed then
                      start_pdf_code()
                    end
                    if savedhtap then
                      for i=1,#savedhtap do
                        local path = savedhtap[i]
                        if transformed then
                          flushconcatpath(path,open)
                        else
                          flushnormalpath(path,open)
                        end
                      end
                      savedhtap = nil
                      evenodd   = true
                    end
                    if transformed then
                      flushconcatpath(path,open)
                    else
                      flushnormalpath(path,open)
                    end
                    if objecttype == "fill" then
                      pdf_literalcode(evenodd and "h f*" or "h f")
                    elseif objecttype == "outline" then
                      pdf_literalcode(open and "S" or "h S")
                    elseif objecttype == "both" then
                      pdf_literalcode(evenodd and "h B*" or "h B")
                    end
                    if transformed then
                      stop_pdf_code()
                    end
                  end
%    \end{macrocode}
%    Added to ConTeXt general: post-object colors and shading stuff.
%    We should beware the |q ... Q| scope.
%    \begin{macrocode}
                  if shade_no then -- shading
                    pdf_literalcode("W%s n /MPlibSh%s sh Q",evenodd and "*" or "",shade_no)
                  end
                end
              end
              if fading_ == "start" then
                pdfetcs.fading.specialeffects = {fading_, tr_opaq, cr_over}
              elseif trgroup == "start" then
                pdfetcs.tr_group.specialeffects = {fading_, tr_opaq, cr_over}
              elseif fading_ == "stop" then
                local se = pdfetcs.fading.specialeffects
                if se then stop_special_effects(se[1], se[2], se[3]) end
              elseif trgroup == "stop" then
                local se = pdfetcs.tr_group.specialeffects
                if se then stop_special_effects(se[1], se[2], se[3]) end
              else
                stop_special_effects(fading_, tr_opaq, cr_over)
              end
              if fading_ or trgroup then -- extgs resetted
                miterlimit, linecap, linejoin, dashed = -1, -1, -1, false
              end
            end
          end
          stop_pdf_code()
          pdf_stopfigure()
%    \end{macrocode}
%    output collected materials to PDF, plus legacy |verbatimtex| code.
%    \begin{macrocode}
          for _,v in ipairs(figcontents) do
            if type(v) == "table" then
              texsprint"\\mplibtoPDF{"; texsprint(v[1], v[2]); texsprint"}"
            else
              texsprint(v)
            end
          end
          if #figcontents.post > 0 then texsprint(figcontents.post) end
          figcontents = { post = { } }
        end
      end
    end
  end
end

function luamplib.colorconverter (cr)
  local n = #cr
  if n == 4 then
    local c, m, y, k = cr[1], cr[2], cr[3], cr[4]
    return format("%.3f %.3f %.3f %.3f k %.3f %.3f %.3f %.3f K",c,m,y,k,c,m,y,k), "0 g 0 G"
  elseif n == 3 then
    local r, g, b = cr[1], cr[2], cr[3]
    return format("%.3f %.3f %.3f rg %.3f %.3f %.3f RG",r,g,b,r,g,b), "0 g 0 G"
  else
    local s = cr[1]
    return format("%.3f g %.3f G",s,s), "0 g 0 G"
  end
end
%    \end{macrocode}
%
% \iffalse
%</lua>
% \fi
%
%    \subsection{\texorpdfstring{\TeX}{TeX} package}
%
%
% \iffalse
%<*package>
% \fi
%
%    First we need to load some packages.
%
%    \begin{macrocode}
\ifcsname ProvidesPackage\endcsname
%    \end{macrocode}
% We need \LaTeX\ 2024-06-01 as we use |ltx.pdf.object_id| when pdfmanagement is loaded.
% But as \pkg{fp} package does not accept an option, we do not append the date option.
%    \begin{macrocode}
  \NeedsTeXFormat{LaTeX2e}
  \ProvidesPackage{luamplib}
    [2025/03/20 v2.37.2 mplib package for LuaTeX]
\fi
\ifdefined\newluafunction\else
  \input ltluatex
\fi
%    \end{macrocode}
%
% In DVI mode, a new XObject (mppattern, mplibgroup) must be encapsulated in an \cs{hbox}.
% But this should not affect typesetting. So we use Hook mechanism provided by \LaTeX\ kernel.
% In Plain, \pkg{atbegshi.sty} is loaded.
%    \begin{macrocode}
\ifnum\outputmode=0
  \ifdefined\AddToHookNext
    \def\luamplibatnextshipout{\AddToHookNext{shipout/background}}
    \def\luamplibatfirstshipout{\AddToHook{shipout/firstpage}}
    \def\luamplibateveryshipout{\AddToHook{shipout/background}}
  \else
    \input atbegshi.sty
    \def\luamplibatnextshipout#1{\AtBeginShipoutNext{\AtBeginShipoutAddToBox{#1}}}
    \let\luamplibatfirstshipout\AtBeginShipoutFirst
    \def\luamplibateveryshipout#1{\AtBeginShipout{\AtBeginShipoutAddToBox{#1}}}
  \fi
\fi
%    \end{macrocode}
%
%    Loading of lua code.
%    \begin{macrocode}
\directlua{require("luamplib")}
%    \end{macrocode}
%
%    legacy commands. Seems we don't need it, but no harm.
%    \begin{macrocode}
\ifx\pdfoutput\undefined
  \let\pdfoutput\outputmode
\fi
\ifx\pdfliteral\undefined
  \protected\def\pdfliteral{\pdfextension literal}
\fi
%    \end{macrocode}
%
%    Set the format for \metapost.
%    \begin{macrocode}
\def\mplibsetformat#1{\directlua{luamplib.setformat("#1")}}
%    \end{macrocode}
%
%    luamplib works in both PDF and DVI mode,
%    but only DVIPDFMx is supported currently among a number of DVI tools.
%    So we output a info.
%    \begin{macrocode}
\ifnum\pdfoutput>0
  \let\mplibtoPDF\pdfliteral
\else
  \def\mplibtoPDF#1{\special{pdf:literal direct #1}}
  \ifcsname PackageInfo\endcsname
    \PackageInfo{luamplib}{only dvipdfmx is supported currently}
  \else
    \immediate\write-1{luamplib Info: only dvipdfmx is supported currently}
  \fi
\fi
%    \end{macrocode}
%
%    To make |mplibcode| typeset always in horizontal mode.
%    \begin{macrocode}
\def\mplibforcehmode{\let\prependtomplibbox\leavevmode}
\def\mplibnoforcehmode{\let\prependtomplibbox\relax}
\mplibnoforcehmode
%    \end{macrocode}
%
%    Catcode. We want to allow comment sign in |mplibcode|.
%    \begin{macrocode}
\def\mplibsetupcatcodes{%
  %catcode`\{=12 %catcode`\}=12
  \catcode`\#=12 \catcode`\^=12 \catcode`\~=12 \catcode`\_=12
  \catcode`\&=12 \catcode`\$=12 \catcode`\%=12 \catcode`\^^M=12
}
%    \end{macrocode}
%
%    Make |btex...etex| box zero-metric.
%    \begin{macrocode}
\def\mplibputtextbox#1{\vbox to 0pt{\vss\hbox to 0pt{\raise\dp#1\copy#1\hss}}}
%    \end{macrocode}
%
%    use Transparency Group
%    \begin{macrocode}
\protected\def\usemplibgroup#1#{\usemplibgroupmain}
\def\usemplibgroupmain#1{%
  \prependtomplibbox\hbox dir TLT\bgroup
  \csname luamplib.group.#1\endcsname
  \egroup
}
\protected\def\mplibgroup#1{%
  \begingroup
  \def\MPllx{0}\def\MPlly{0}%
  \def\mplibgroupname{#1}%
  \mplibgroupgetnexttok
}
\def\mplibgroupgetnexttok{\futurelet\nexttok\mplibgroupbranch}
\def\mplibgroupskipspace{\afterassignment\mplibgroupgetnexttok\let\nexttok= }
\def\mplibgroupbranch{%
  \ifx [\nexttok
    \expandafter\mplibgroupopts
  \else
    \ifx\mplibsptoken\nexttok
      \expandafter\expandafter\expandafter\mplibgroupskipspace
    \else
      \let\mplibgroupoptions\empty
      \expandafter\expandafter\expandafter\mplibgroupmain
    \fi
  \fi
}
\def\mplibgroupopts[#1]{\def\mplibgroupoptions{#1}\mplibgroupmain}
\def\mplibgroupmain{\setbox\mplibscratchbox\hbox\bgroup\ignorespaces}
\protected\def\endmplibgroup{\egroup
  \directlua{ luamplib.registergroup(
    \the\mplibscratchbox, '\mplibgroupname', {\mplibgroupoptions}
  )}%
  \endgroup
}
%    \end{macrocode}
%
%    Patterns
%    \begin{macrocode}
{\def\:{\global\let\mplibsptoken= } \: }
\protected\def\mppattern#1{%
  \begingroup
  \def\mplibpatternname{#1}%
  \mplibpatterngetnexttok
}
\def\mplibpatterngetnexttok{\futurelet\nexttok\mplibpatternbranch}
\def\mplibpatternskipspace{\afterassignment\mplibpatterngetnexttok\let\nexttok= }
\def\mplibpatternbranch{%
  \ifx [\nexttok
    \expandafter\mplibpatternopts
  \else
    \ifx\mplibsptoken\nexttok
      \expandafter\expandafter\expandafter\mplibpatternskipspace
    \else
      \let\mplibpatternoptions\empty
      \expandafter\expandafter\expandafter\mplibpatternmain
    \fi
  \fi
}
\def\mplibpatternopts[#1]{%
  \def\mplibpatternoptions{#1}%
  \mplibpatternmain
}
\def\mplibpatternmain{%
  \setbox\mplibscratchbox\hbox\bgroup\ignorespaces
}
\protected\def\endmppattern{%
  \egroup
  \directlua{ luamplib.registerpattern(
    \the\mplibscratchbox, '\mplibpatternname', {\mplibpatternoptions}
  )}%
  \endgroup
}
%    \end{macrocode}
%
%    simple way to use \mplib:
%    |\mpfig draw fullcircle scaled 10; \endmpfig|
%    \begin{macrocode}
\def\mpfiginstancename{@mpfig}
\protected\def\mpfig{%
  \begingroup
  \futurelet\nexttok\mplibmpfigbranch
}
\def\mplibmpfigbranch{%
  \ifx *\nexttok
    \expandafter\mplibprempfig
  \else
    \ifx [\nexttok
      \expandafter\expandafter\expandafter\mplibgobbleoptsmpfig
    \else
      \expandafter\expandafter\expandafter\mplibmainmpfig
    \fi
  \fi
}
\def\mplibgobbleoptsmpfig[#1]{\mplibmainmpfig}
\def\mplibmainmpfig{%
  \begingroup
  \mplibsetupcatcodes
  \mplibdomainmpfig
}
\long\def\mplibdomainmpfig#1\endmpfig{%
  \endgroup
  \directlua{
    local legacy = luamplib.legacyverbatimtex
    local everympfig = luamplib.everymplib["\mpfiginstancename"] or ""
    local everyendmpfig = luamplib.everyendmplib["\mpfiginstancename"] or ""
    luamplib.legacyverbatimtex = false
    luamplib.everymplib["\mpfiginstancename"] = ""
    luamplib.everyendmplib["\mpfiginstancename"] = ""
    luamplib.process_mplibcode(
    "beginfig(0) "..everympfig.." "..[===[\unexpanded{#1}]===].." "..everyendmpfig.." endfig;",
    "\mpfiginstancename")
    luamplib.legacyverbatimtex = legacy
    luamplib.everymplib["\mpfiginstancename"] = everympfig
    luamplib.everyendmplib["\mpfiginstancename"] = everyendmpfig
  }%
  \endgroup
}
\def\mplibprempfig#1{%
  \begingroup
  \mplibsetupcatcodes
  \mplibdoprempfig
}
\long\def\mplibdoprempfig#1\endmpfig{%
  \endgroup
  \directlua{
    local legacy = luamplib.legacyverbatimtex
    local everympfig = luamplib.everymplib["\mpfiginstancename"]
    local everyendmpfig = luamplib.everyendmplib["\mpfiginstancename"]
    luamplib.legacyverbatimtex = false
    luamplib.everymplib["\mpfiginstancename"] = ""
    luamplib.everyendmplib["\mpfiginstancename"] = ""
    luamplib.process_mplibcode([===[\unexpanded{#1}]===],"\mpfiginstancename")
    luamplib.legacyverbatimtex = legacy
    luamplib.everymplib["\mpfiginstancename"] = everympfig
    luamplib.everyendmplib["\mpfiginstancename"] = everyendmpfig
  }%
  \endgroup
}
\protected\def\endmpfig{endmpfig}
%    \end{macrocode}
%
%    The Plain-specific stuff.
%    \begin{macrocode}
\unless\ifcsname ver@luamplib.sty\endcsname
  \def\mplibcodegetinstancename[#1]{\xdef\currentmpinstancename{#1}\mplibcodeindeed}
  \protected\def\mplibcode{%
    \begingroup
    \futurelet\nexttok\mplibcodebranch
  }
  \def\mplibcodebranch{%
    \ifx [\nexttok
      \expandafter\mplibcodegetinstancename
    \else
      \global\let\currentmpinstancename\empty
      \expandafter\mplibcodeindeed
    \fi
  }
  \def\mplibcodeindeed{%
    \begingroup
    \mplibsetupcatcodes
    \mplibdocode
  }
  \long\def\mplibdocode#1\endmplibcode{%
    \endgroup
    \directlua{luamplib.process_mplibcode([===[\unexpanded{#1}]===],"\currentmpinstancename")}%
    \endgroup
  }
  \protected\def\endmplibcode{endmplibcode}
\else
%    \end{macrocode}
%
%    The \LaTeX-specific part: a new environment.
%    \begin{macrocode}
  \newenvironment{mplibcode}[1][]{%
    \xdef\currentmpinstancename{#1}%
    \mplibtmptoks{}\ltxdomplibcode
  }{}
  \def\ltxdomplibcode{%
    \begingroup
    \mplibsetupcatcodes
    \ltxdomplibcodeindeed
  }
  \def\mplib@mplibcode{mplibcode}
  \long\def\ltxdomplibcodeindeed#1\end#2{%
    \endgroup
    \mplibtmptoks\expandafter{\the\mplibtmptoks#1}%
    \def\mplibtemp@a{#2}%
    \ifx\mplib@mplibcode\mplibtemp@a
      \directlua{luamplib.process_mplibcode([===[\the\mplibtmptoks]===],"\currentmpinstancename")}%
      \end{mplibcode}%
    \else
      \mplibtmptoks\expandafter{\the\mplibtmptoks\end{#2}}%
      \expandafter\ltxdomplibcode
    \fi
  }
\fi
%    \end{macrocode}
%
%    User settings.
%    \begin{macrocode}
\def\mplibshowlog#1{\directlua{
    local s = string.lower("#1")
    if s == "enable" or s == "true" or s == "yes" then
      luamplib.showlog = true
    else
      luamplib.showlog = false
    end
}}
\def\mpliblegacybehavior#1{\directlua{
    local s = string.lower("#1")
    if s == "enable" or s == "true" or s == "yes" then
      luamplib.legacyverbatimtex = true
    else
      luamplib.legacyverbatimtex = false
    end
}}
\def\mplibverbatim#1{\directlua{
    local s = string.lower("#1")
    if s == "enable" or s == "true" or s == "yes" then
      luamplib.verbatiminput = true
    else
      luamplib.verbatiminput = false
    end
}}
\newtoks\mplibtmptoks
%    \end{macrocode}
%
%    \cs{everymplib} \& \cs{everyendmplib}: macros resetting
%    |luamplib.every(end)mplib| tables
%
%    \begin{macrocode}
\ifcsname ver@luamplib.sty\endcsname
  \protected\def\everymplib{%
    \begingroup
    \mplibsetupcatcodes
    \mplibdoeverymplib
  }
  \protected\def\everyendmplib{%
    \begingroup
    \mplibsetupcatcodes
    \mplibdoeveryendmplib
  }
  \newcommand\mplibdoeverymplib[2][]{%
    \endgroup
    \directlua{
      luamplib.everymplib["#1"] = [===[\unexpanded{#2}]===]
    }%
  }
  \newcommand\mplibdoeveryendmplib[2][]{%
    \endgroup
    \directlua{
      luamplib.everyendmplib["#1"] = [===[\unexpanded{#2}]===]
    }%
  }
\else
  \def\mplibgetinstancename[#1]{\def\currentmpinstancename{#1}}
  \protected\def\everymplib#1#{%
    \ifx\empty#1\empty \mplibgetinstancename[]\else \mplibgetinstancename#1\fi
    \begingroup
    \mplibsetupcatcodes
    \mplibdoeverymplib
  }
  \long\def\mplibdoeverymplib#1{%
    \endgroup
    \directlua{
      luamplib.everymplib["\currentmpinstancename"] = [===[\unexpanded{#1}]===]
    }%
  }
  \protected\def\everyendmplib#1#{%
    \ifx\empty#1\empty \mplibgetinstancename[]\else \mplibgetinstancename#1\fi
    \begingroup
    \mplibsetupcatcodes
    \mplibdoeveryendmplib
  }
  \long\def\mplibdoeveryendmplib#1{%
    \endgroup
    \directlua{
      luamplib.everyendmplib["\currentmpinstancename"] = [===[\unexpanded{#1}]===]
    }%
  }
\fi
%    \end{macrocode}
%
%    Allow \TeX\ dimen/color macros. Now |runscript| does the job,
%    so the following lines are not needed for most cases.
%    \begin{macrocode}
\def\mpdim#1{ runscript("luamplibdimen{#1}") }
\def\mpcolor#1#{\domplibcolor{#1}}
\def\domplibcolor#1#2{ runscript("luamplibcolor{#1{#2}}") }
%    \end{macrocode}
%
%    \mplib's number system. Now |binary| has gone away.
%    \begin{macrocode}
\def\mplibnumbersystem#1{\directlua{
  local t = "#1"
  if t == "binary" then t = "decimal" end
  luamplib.numbersystem = t
}}
%    \end{macrocode}
%
%    Settings for |.mp| cache files.
%    \begin{macrocode}
\def\mplibmakenocache#1{\mplibdomakenocache #1,*,}
\def\mplibdomakenocache#1,{%
  \ifx\empty#1\empty
    \expandafter\mplibdomakenocache
  \else
    \ifx*#1\else
      \directlua{luamplib.noneedtoreplace["#1.mp"]=true}%
      \expandafter\expandafter\expandafter\mplibdomakenocache
    \fi
  \fi
}
\def\mplibcancelnocache#1{\mplibdocancelnocache #1,*,}
\def\mplibdocancelnocache#1,{%
  \ifx\empty#1\empty
    \expandafter\mplibdocancelnocache
  \else
    \ifx*#1\else
      \directlua{luamplib.noneedtoreplace["#1.mp"]=false}%
      \expandafter\expandafter\expandafter\mplibdocancelnocache
    \fi
  \fi
}
\def\mplibcachedir#1{\directlua{luamplib.getcachedir("\unexpanded{#1}")}}
%    \end{macrocode}
%
%    More user settings.
%    \begin{macrocode}
\def\mplibtextextlabel#1{\directlua{
    local s = string.lower("#1")
    if s == "enable" or s == "true" or s == "yes" then
      luamplib.textextlabel = true
    else
      luamplib.textextlabel = false
    end
}}
\def\mplibcodeinherit#1{\directlua{
    local s = string.lower("#1")
    if s == "enable" or s == "true" or s == "yes" then
      luamplib.codeinherit = true
    else
      luamplib.codeinherit = false
    end
}}
\def\mplibglobaltextext#1{\directlua{
    local s = string.lower("#1")
    if s == "enable" or s == "true" or s == "yes" then
      luamplib.globaltextext = true
    else
      luamplib.globaltextext = false
    end
}}
%    \end{macrocode}
%
%    The followings are from ConTeXt general, mostly.
%
%    We use a dedicated scratchbox.
%    \begin{macrocode}
\ifx\mplibscratchbox\undefined \newbox\mplibscratchbox \fi
%    \end{macrocode}
%
%    We encapsulate the literals.
%    \begin{macrocode}
\def\mplibstarttoPDF#1#2#3#4{%
  \prependtomplibbox
  \hbox dir TLT\bgroup
  \xdef\MPllx{#1}\xdef\MPlly{#2}%
  \xdef\MPurx{#3}\xdef\MPury{#4}%
  \xdef\MPwidth{\the\dimexpr#3bp-#1bp\relax}%
  \xdef\MPheight{\the\dimexpr#4bp-#2bp\relax}%
  \parskip0pt%
  \leftskip0pt%
  \parindent0pt%
  \everypar{}%
  \setbox\mplibscratchbox\vbox\bgroup
  \noindent
}
\def\mplibstoptoPDF{%
  \par
  \egroup %
  \setbox\mplibscratchbox\hbox %
    {\hskip-\MPllx bp%
     \raise-\MPlly bp%
     \box\mplibscratchbox}%
  \setbox\mplibscratchbox\vbox to \MPheight
    {\vfill
     \hsize\MPwidth
     \wd\mplibscratchbox0pt%
     \ht\mplibscratchbox0pt%
     \dp\mplibscratchbox0pt%
     \box\mplibscratchbox}%
  \wd\mplibscratchbox\MPwidth
  \ht\mplibscratchbox\MPheight
  \box\mplibscratchbox
  \egroup
}
%    \end{macrocode}
%
%    Text items have a special handler.
%    \begin{macrocode}
\def\mplibtextext#1#2#3#4#5{%
  \begingroup
  \setbox\mplibscratchbox\hbox
    {\font\temp=#1 at #2bp%
     \temp
     #3}%
  \setbox\mplibscratchbox\hbox
    {\hskip#4 bp%
     \raise#5 bp%
     \box\mplibscratchbox}%
  \wd\mplibscratchbox0pt%
  \ht\mplibscratchbox0pt%
  \dp\mplibscratchbox0pt%
  \box\mplibscratchbox
  \endgroup
}
%    \end{macrocode}
%
%    Input |luamplib.cfg| when it exists.
%    \begin{macrocode}
\openin0=luamplib.cfg
\ifeof0 \else
  \closein0
  \input luamplib.cfg
\fi
%    \end{macrocode}
%
%    Code for \pkg{tagpdf}
%    \begin{macrocode}
\def\luamplibtagtextbegin#1{}
\let\luamplibtagtextend\relax
\let\luamplibtagasgroupinit\relax
\let\luamplibtagasgroupbegin\relax
\let\luamplibtagasgroupend\relax
\ifcsname SuspendTagging\endcsname\else\endinput\fi
\ifcsname ver@tagpdf.sty\endcsname \else
  \ExplSyntaxOn
  \keys_define:nn{luamplib/notagging}
    {
      ,alt          .code:n = { }
      ,actualtext   .code:n = { }
      ,artifact     .code:n = { }
      ,text         .code:n = { }
      ,off          .code:n = { }
      ,tag          .code:n = { }
      ,adjust-BBox  .code:n = { }
      ,tagging-setup .code:n = { }
      ,instance     .code:n = { \tl_gset:Nn \currentmpinstancename {#1} }
      ,instancename .meta:n = { instance = {#1} }
      ,unknown      .code:n = { \tl_gset:Ne \currentmpinstancename {\l_keys_key_str} }
    }
  \RenewDocumentCommand\mplibcode{O{}}
    {
      \tl_gset_eq:NN \currentmpinstancename \c_empty_tl
      \keys_set:ne{luamplib/notagging}{#1}
      \mplibtmptoks{}\ltxdomplibcode
    }
  \ExplSyntaxOff
  \let\mplibalttext \luamplibtagtextbegin
  \let\mplibactualtext \mplibalttext
  \endinput\fi
\ExplSyntaxOn
\tl_new:N \l__luamplib_tag_envname_tl
\tl_set:Nn\l__luamplib_tag_envname_tl {mplibcode}
\tl_new:N \l__luamplib_tag_alt_tl
\tl_new:N \l__luamplib_tag_alt_dflt_tl
\tl_set:Nn\l__luamplib_tag_alt_dflt_tl {metapost~figure}
\tl_new:N \l__luamplib_tag_actual_tl
\tl_new:N \l__luamplib_tag_struct_tl
\tl_set:Nn\l__luamplib_tag_struct_tl {Figure}
\bool_new:N \l__luamplib_tag_usetext_bool
\bool_new:N \l__luamplib_tag_bboxcorr_bool
\seq_new:N  \l__luamplib_tag_bboxcorr_seq
\tl_new:N \l__luamplib_tag_bbox_draw_tl
\tl_new:N \l__luamplib_BBox_llx_tl
\tl_new:N \l__luamplib_BBox_lly_tl
\tl_new:N \l__luamplib_BBox_urx_tl
\tl_new:N \l__luamplib_BBox_ury_tl
\msg_new:nnn {luamplib}{figure-text-reuse}
{
  tex-text~box~#1~probably~is~incorrectly~tagged.\\
  Reusing~a~box~in~text-mode~figures~is~strongly~discouraged.
}
\msg_new:nnn{luamplib}{alt-text-missing}
{
  Alternate~text~for~#1~is~missing.\\
  Using~the~default~value~'#2'~instead.
}
%    \end{macrocode}
% Sockets for textext
%    \begin{macrocode}
\socket_new:nn{tagsupport/luamplib/textext/begin}{1}
\socket_new:nn{tagsupport/luamplib/textext/end}{0}
\socket_new:nn{tagsupport/luamplib/textext/put}{2}
\socket_new_plug:nnn{tagsupport/luamplib/textext/begin}{default}
{
    \tag_mc_end_push:
    \tag_struct_begin:n{tag=NonStruct,stash}
    \cs_gset_nopar:cpe {luamplib.tagbox.#1} {\tag_get:n{struct_num}}
%    \end{macrocode}
% TODO: We force a structure. If not, |a| and |b| in |btex a $x$ b etex| are not tagged.
%    \begin{macrocode}
    \tag_mc_begin:n{tag=text}
}
\socket_new_plug:nnn{tagsupport/luamplib/textext/end}{default}
{
    \tag_mc_end:
    \tag_struct_end:
    \tag_mc_begin_pop:n{}
}
\socket_new_plug:nnn{tagsupport/luamplib/textext/put}{default}
{
  \bool_if:NTF \l__luamplib_tag_usetext_bool
  {
    \tag_resume:n{\mplibputtextbox}
    \tag_mc_end:
    \cs_if_exist:cTF {luamplib.tagbox.#1}
    {
%    \end{macrocode}
% TODO: To avoid nested |P|. Not sure this is a bug of tagpdf.
%    \begin{macrocode}
      \tag_struct_begin:n{tag=NonStruct}
      \tag_struct_use_num:n {\csname luamplib.tagbox.#1\endcsname}
      #2
      \tag_struct_end:
    }
    {
      \msg_warning:nnn{luamplib}{figure-text-reuse}{#1}
      \tag_mc_begin:n{}
      \int_set:Nn \l_tmpa_int {#1}
      \tag_mc_reset_box:N \l_tmpa_int
      #2
      \tag_mc_end:
    }
    \tag_mc_begin:n{artifact}
  }
  {
    \int_set:Nn \l_tmpa_int {#1}
    \tag_mc_reset_box:N \l_tmpa_int
    #2
  }
}
\socket_assign_plug:nn{tagsupport/luamplib/textext/begin}{default}
\socket_assign_plug:nn{tagsupport/luamplib/textext/end}{default}
\socket_assign_plug:nn{tagsupport/luamplib/textext/put}{default}
\cs_set_nopar:Npn \luamplibtagtextbegin #1
{
  \bool_if:NTF \l__luamplib_tag_usetext_bool
  {
    \tag_socket_use:nn{luamplib/textext/begin}{#1}
  }
  {
    \tag_if_active:TF
      { \bool_set_true:N \l_tmpa_bool }
      { \bool_set_false:N \l_tmpa_bool }
    \tag_suspend:n{\luamplibtagtextbegin}
  }
}
\cs_set_nopar:Npn \luamplibtagtextend
{
  \bool_if:NTF \l__luamplib_tag_usetext_bool
  {
    \tag_socket_use:n{luamplib/textext/end}
  }
  {
    \bool_if:NT \l_tmpa_bool
      { \tag_resume:n{\luamplibtagtextend} }
  }
}
\cs_set_nopar:Npn \mplibputtextbox #1
{
  \vbox to 0pt{\vss\hbox to 0pt{%
%    \end{macrocode}
% This socket should be always active.
%    \begin{macrocode}
    \socket_use:nnn{tagsupport/luamplib/textext/put}{#1}{\raise\dp#1\copy#1}
  \hss}}
}
%    \end{macrocode}
% TODO: Not sure whether asgroup will be tagged correctly.
%       At least, this will raise a warning;
%       but how can we tease out whether a tex-text box is for asgroup or not?
%    \begin{macrocode}
\cs_set_nopar:Npn \luamplibtagasgroupinit
{
  \bool_set_false:N \l__luamplib_tag_usetext_bool
}
\cs_set_nopar:Npn \luamplibtagasgroupbegin
{
  \bool_if:NT \l__luamplib_tag_usetext_bool
  {
    \tag_resume:n{\luamplibtagasgroupbegin}
    \tag_mc_end:
    \tag_mc_begin:n{}
  }
}
\cs_set_nopar:Npn \luamplibtagasgroupend
{
  \bool_if:NT \l__luamplib_tag_usetext_bool
  {
    \tag_mc_end:
    \tag_mc_begin:n{artifact}
    \tag_suspend:n{\luamplibtagasgroupend}
  }
}
%    \end{macrocode}
% Sockets for mplibgroup
%    \begin{macrocode}
\socket_new:nn{tagsupport/luamplib/mplibgroup/begin}{0}
\socket_new:nn{tagsupport/luamplib/mplibgroup/end}{0}
\socket_new_plug:nnn{tagsupport/luamplib/mplibgroup/begin}{default}
{
    \bool_if:NT \l__luamplib_tag_usetext_bool
    {
      \tag_mc_end:
      \tag_mc_begin:n{}
    }
}
\socket_new_plug:nnn{tagsupport/luamplib/mplibgroup/end}{default}
{
    \bool_if:NT \l__luamplib_tag_usetext_bool
    {
      \tag_mc_end:
      \tag_mc_begin:n{artifact}
    }
}
\socket_assign_plug:nn{tagsupport/luamplib/mplibgroup/begin}{default}
\socket_assign_plug:nn{tagsupport/luamplib/mplibgroup/end}{default}
%    \end{macrocode}
% Sockets for main process
%    \begin{macrocode}
\socket_new:nn{tagsupport/luamplib/figure/begin}{0}
\socket_new:nn{tagsupport/luamplib/figure/end}{0}
\socket_new_plug:nnn{tagsupport/luamplib/figure/begin}{alt}
{
    \tag_mc_end_push:
    \tl_if_empty:NT\l__luamplib_tag_alt_tl
    {
      \msg_warning:nnee{luamplib}{alt-text-missing}
                       {\l__luamplib_tag_envname_tl}{\l__luamplib_tag_alt_dflt_tl}
      \tl_set:Ne\l__luamplib_tag_alt_tl {\l__luamplib_tag_alt_dflt_tl}
    }
    \tag_struct_begin:n
    {
      tag=\l__luamplib_tag_struct_tl,
      alt=\l__luamplib_tag_alt_tl,
    }
    \tag_mc_begin:n{}
}
\socket_new_plug:nnn{tagsupport/luamplib/figure/end}{alt}
{
    \tl_use:N \l__luamplib_tag_bbox_draw_tl
    \tag_mc_end:
    \tag_struct_end:
    \tag_mc_begin_pop:n{}
}
\socket_new_plug:nnn{tagsupport/luamplib/figure/begin}{actualtext}
{
    \tag_mc_end_push:
    \tag_struct_begin:n
    {
      tag=Span,
      actualtext=\l__luamplib_tag_actual_tl,
    }
    \tag_mc_begin:n{}
}
\socket_new_plug:nnn{tagsupport/luamplib/figure/end}{actualtext}
{
    \tag_mc_end:
    \tag_struct_end:
    \tag_mc_begin_pop:n{}
}
\socket_new_plug:nnn{tagsupport/luamplib/figure/begin}{artifact}
{
    \tag_mc_end_push:
    \tag_mc_begin:n{artifact}
}
\socket_new_plug:nnn{tagsupport/luamplib/figure/end}{artifact}
{
    \tag_mc_end:
    \tag_mc_begin_pop:n{}
}
%    \end{macrocode}
% A socket for BBox attribute
%    \begin{macrocode}
\socket_new:nn{tagsupport/luamplib/figure/bbox}{1}
\socket_new_plug:nnn{tagsupport/luamplib/figure/bbox}{alt}
{
    \tl_set:Ne \l_tmpa_tl {luamplib.BBox.\tag_get:n{struct_num}}
    \tex_savepos:D
    \property_record:ee{\l_tmpa_tl}{xpos,ypos}
    \tl_set:Ne \l__luamplib_BBox_llx_tl
      {
        \dim_to_decimal_in_bp:n { \property_ref:een {\l_tmpa_tl}{xpos}{0}sp }
      }
    \tl_set:Ne \l__luamplib_BBox_lly_tl
      {
        \dim_to_decimal_in_bp:n { \property_ref:een {\l_tmpa_tl}{ypos}{0}sp - \dp#1 }
      }
    \tl_set:Ne \l__luamplib_BBox_urx_tl
      {
        \dim_to_decimal_in_bp:n { \l__luamplib_BBox_llx_tl bp + \wd#1 }
      }
    \tl_set:Ne \l__luamplib_BBox_ury_tl
      {
        \dim_to_decimal_in_bp:n { \l__luamplib_BBox_lly_tl bp + \ht#1 + \dp#1 }
      }
    \bool_if:NT \l__luamplib_tag_bboxcorr_bool
      {
        \tl_set:Ne \l__luamplib_BBox_llx_tl
          {
            \fp_eval:n
            {
              \l__luamplib_BBox_llx_tl
              +
              \dim_to_decimal_in_bp:n {\seq_item:Nn \l__luamplib_tag_bboxcorr_seq {1} }
            }
          }
        \tl_set:Ne \l__luamplib_BBox_lly_tl
          {
            \fp_eval:n
            {
              \l__luamplib_BBox_lly_tl
              +
              \dim_to_decimal_in_bp:n {\seq_item:Nn \l__luamplib_tag_bboxcorr_seq {2} }
            }
          }
        \tl_set:Ne \l__luamplib_BBox_urx_tl
          {
            \fp_eval:n
            {
              \l__luamplib_BBox_urx_tl
              +
              \dim_to_decimal_in_bp:n {\seq_item:Nn \l__luamplib_tag_bboxcorr_seq {3} }
            }
          }
        \tl_set:Ne \l__luamplib_BBox_ury_tl
          {
            \fp_eval:n
            {
              \l__luamplib_BBox_ury_tl
              +
              \dim_to_decimal_in_bp:n {\seq_item:Nn \l__luamplib_tag_bboxcorr_seq {4} }
            }
          }
      }
    \tag_struct_gput:ene {\tag_get:n{struct_num}} {attribute}
      {
        /O /Layout /BBox [
          \l__luamplib_BBox_llx_tl\c_space_tl
          \l__luamplib_BBox_lly_tl\c_space_tl
          \l__luamplib_BBox_urx_tl\c_space_tl
          \l__luamplib_BBox_ury_tl
        ]
      }
    \bool_if:NT \l__tag_graphic_debug_bool
      {
        \iow_log:e
          {
            luamplib/tagging~debug:~BBox~of~structure~\tag_get:n{struct_num}~is~
            \l__luamplib_BBox_llx_tl\c_space_tl
            \l__luamplib_BBox_lly_tl\c_space_tl
            \l__luamplib_BBox_urx_tl\c_space_tl
            \l__luamplib_BBox_ury_tl
          }
        \tl_set:Ne \l__luamplib_tag_bbox_draw_tl
        {
          \pdfextension save\relax
          \color_group_begin:
          \opacity_select:n{0.5} \color_select:n{red}
          \pdfextension literal~text
          {
            \l__luamplib_BBox_llx_tl\c_space_tl
            \l__luamplib_BBox_lly_tl\c_space_tl
            \fp_eval:n { \l__luamplib_BBox_urx_tl - \l__luamplib_BBox_llx_tl }~
            \fp_eval:n { \l__luamplib_BBox_ury_tl - \l__luamplib_BBox_lly_tl }~
            re~f
          }
          \color_group_end:
          \pdfextension restore\relax
        }
      }
}
%    \end{macrocode}
% horizontal mode upon |text| and |actualtext|
%    \begin{macrocode}
\cs_set_nopar:Npn \__luamplib_tag_pseudo_para_begin:
{
  \prependtomplibbox \mplibnoforcehmode
  \mode_if_vertical:T
  {
    \tag_socket_use:n{para/begin}
    \group_insert_after:N \__luamplib_tag_pseudo_para_end:
  }
}
\cs_set_nopar:Npn \__luamplib_tag_pseudo_para_end:
{
  \mode_if_vertical:T
  {
    \tag_socket_use:n{para/end}
  }
}
%    \end{macrocode}
% A socket for tagging init
%    \begin{macrocode}
\socket_new:nn{tagsupport/luamplib/figure/init}{0}
\socket_new_plug:nnn{tagsupport/luamplib/figure/init}{alt}
{
  \socket_assign_plug:nn{tagsupport/luamplib/figure/begin}{alt}
  \socket_assign_plug:nn{tagsupport/luamplib/figure/end}{alt}
  \socket_assign_plug:nn{tagsupport/luamplib/figure/bbox}{alt}
}
\socket_new_plug:nnn{tagsupport/luamplib/figure/init}{actualtext}
{
  \socket_assign_plug:nn{tagsupport/luamplib/figure/begin}{actualtext}
  \socket_assign_plug:nn{tagsupport/luamplib/figure/end}{actualtext}
  \socket_assign_plug:nn{tagsupport/luamplib/figure/bbox}{noop}
  \__luamplib_tag_pseudo_para_begin:
}
\socket_new_plug:nnn{tagsupport/luamplib/figure/init}{artifact}
{
  \socket_assign_plug:nn{tagsupport/luamplib/figure/begin}{artifact}
  \socket_assign_plug:nn{tagsupport/luamplib/figure/end}{artifact}
  \socket_assign_plug:nn{tagsupport/luamplib/figure/bbox}{noop}
}
\socket_new_plug:nnn{tagsupport/luamplib/figure/init}{text}
{
  \bool_set_true:N \l__luamplib_tag_usetext_bool
  \socket_assign_plug:nn{tagsupport/luamplib/figure/begin}{artifact}
  \socket_assign_plug:nn{tagsupport/luamplib/figure/end}{artifact}
  \socket_assign_plug:nn{tagsupport/luamplib/figure/bbox}{noop}
  \__luamplib_tag_pseudo_para_begin:
}
\socket_new_plug:nnn{tagsupport/luamplib/figure/init}{off}
{
  \socket_assign_plug:nn{tagsupport/luamplib/figure/begin}{noop}
  \socket_assign_plug:nn{tagsupport/luamplib/figure/end}{noop}
  \socket_assign_plug:nn{tagsupport/luamplib/figure/bbox}{noop}
}
\socket_assign_plug:nn{tagsupport/luamplib/figure/init}{alt}
%    \end{macrocode}
% Key-value options
%    \begin{macrocode}
\keys_define:nn{luamplib/tagging}
  {
    ,alt .code:n =
      {
        \tl_set:Ne\l__luamplib_tag_alt_tl{\text_purify:n{#1}}
        \socket_assign_plug:nn{tagsupport/luamplib/figure/init}{alt}
      }
    ,actualtext .code:n =
      {
        \tl_set:Ne\l__luamplib_tag_actual_tl{\text_purify:n{#1}}
        \socket_assign_plug:nn{tagsupport/luamplib/figure/init}{actualtext}
      }
    ,artifact .code:n = { \socket_assign_plug:nn{tagsupport/luamplib/figure/init}{artifact} }
    ,text     .code:n = { \socket_assign_plug:nn{tagsupport/luamplib/figure/init}{text} }
    ,off      .code:n = { \socket_assign_plug:nn{tagsupport/luamplib/figure/init}{off} }
    ,tag      .code:n =
      {
        \str_case:nnF {#1}
          {
            {false}    { \keys_set:nn {luamplib/tagging} {off} }
            {artifact} { \keys_set:nn {luamplib/tagging} {artifact} }
          }
          {
            \tl_set:Nn\l__luamplib_tag_struct_tl{#1}
            \socket_assign_plug:nn{tagsupport/luamplib/figure/init}{alt}
          }
      }
    ,adjust-BBox .code:n =
      {
        \bool_set_true:N \l__luamplib_tag_bboxcorr_bool
        \seq_set_split:Nnn \l__luamplib_tag_bboxcorr_seq{~}{#1~0pt~0pt~0pt~0pt}
      }
    ,tagging-setup .code:n = { \keys_set_known:nn {luamplib/tagging} {#1} }
  }
\keys_define:nn {luamplib/instance}
  {
    ,instance     .code:n = { \tl_gset:Nn \currentmpinstancename {#1} }
    ,instancename .meta:n = { instance = {#1} }
    ,unknown      .code:n = { \tl_gset:Ne \currentmpinstancename {\l_keys_key_str} }
  }
%    \end{macrocode}
% Redefine our macros
%    \begin{macrocode}
\cs_set_nopar:Npn \mplibstarttoPDF #1 #2 #3 #4
  {
    \prependtomplibbox
    \hbox dir~TLT\bgroup
    \tag_socket_use:n{luamplib/figure/begin}
    \xdef\MPllx{#1}\xdef\MPlly{#2}%
    \xdef\MPurx{#3}\xdef\MPury{#4}%
    \xdef\MPwidth{\the\dimexpr#3bp-#1bp\relax}%
    \xdef\MPheight{\the\dimexpr#4bp-#2bp\relax}%
    \parskip0pt
    \leftskip0pt
    \parindent0pt
    \everypar{}%
    \setbox\mplibscratchbox\vbox\bgroup
    \tag_suspend:n{\mplibstarttoPDF}
    \noindent
  }
\cs_set_nopar:Npn \mplibstoptoPDF
  {
    \par
    \egroup
    \setbox\mplibscratchbox\hbox
      {\hskip-\MPllx bp
       \raise-\MPlly bp
       \box\mplibscratchbox}%
    \setbox\mplibscratchbox\vbox to \MPheight
      {\vfill
       \hsize\MPwidth
       \wd\mplibscratchbox0pt
       \ht\mplibscratchbox0pt
       \dp\mplibscratchbox0pt
       \box\mplibscratchbox}%
    \wd\mplibscratchbox\MPwidth
    \ht\mplibscratchbox\MPheight
    \tag_socket_use:nn{luamplib/figure/bbox}\mplibscratchbox
    \box\mplibscratchbox
    \tag_socket_use:n{luamplib/figure/end}
    \egroup
  }
\RenewDocumentCommand\mplibcode{O{}}
  {
    \tl_set:Nn \l__luamplib_tag_envname_tl {mplibcode}
    \tl_gset_eq:NN \currentmpinstancename \c_empty_tl
    \keys_set_known:neN {luamplib/tagging} {#1} \l_tmpa_tl
    \keys_set:nV {luamplib/instance} \l_tmpa_tl
    \tl_if_empty:NF \currentmpinstancename
      { \tl_set:Nn\l__luamplib_tag_alt_dflt_tl {metapost~figure~\currentmpinstancename} }
    \tag_socket_use:n{luamplib/figure/init}
    \mplibtmptoks{}\ltxdomplibcode
  }
\RenewDocumentCommand\mpfig{s O{}}
  {
    \begingroup
    \tl_set:Nn \l__luamplib_tag_envname_tl {mpfig}
    \keys_set_known:neN {luamplib/tagging} {#2} \l_tmpa_tl
    \keys_set:nV {luamplib/instance} \l_tmpa_tl
    \tl_if_empty:NF \mpfiginstancename
      { \tl_set:Nn\l__luamplib_tag_alt_dflt_tl {metapost~figure~\mpfiginstancename} }
    \tag_socket_use:n{luamplib/figure/init}
    \IfBooleanTF{#1}
      { \mplibprempfig * }
      { \mplibmainmpfig }
  }
\RenewDocumentCommand\usemplibgroup{O{} m}
  {
    \begingroup
    \tl_set:Nn \l__luamplib_tag_envname_tl {usemplibgroup}
    \keys_set_known:neN {luamplib/tagging} {#1} \l_tmpa_tl
    \keys_set:nV {luamplib/instance} \l_tmpa_tl
    \tl_set:Nn\l__luamplib_tag_alt_dflt_tl {metapost~figure~#2}
    \tag_socket_use:n{luamplib/figure/init}
    \prependtomplibbox\hbox dir~TLT\bgroup
    \tag_socket_use:n{luamplib/figure/begin}
    \setbox\mplibscratchbox\hbox\bgroup
    \tag_socket_use:n{luamplib/mplibgroup/begin}
    \csname luamplib.group.#2\endcsname
    \tag_socket_use:n{luamplib/mplibgroup/end}
    \egroup
    \tag_socket_use:nn{luamplib/figure/bbox}\mplibscratchbox
    \unhbox\mplibscratchbox
    \tag_socket_use:n{luamplib/figure/end}
    \egroup
    \endgroup
  }
%    \end{macrocode}
% Allow setting alt/actual text within \metapost code
%    \begin{macrocode}
\cs_new_nopar:Npn \mplibalttext #1
{
  \tl_set:Ne \l__luamplib_tag_alt_tl {\text_purify:n{#1}}
}
\cs_new_nopar:Npn \mplibactualtext #1
{
  \tl_set:Ne \l__luamplib_tag_actual_tl {\text_purify:n{#1}}
}
\ExplSyntaxOff
%    \end{macrocode}
%
%    That's all folks!
%
% \iffalse
%</package>
% \fi
%
% \clearpage
% \section{The GNU GPL License v2}
%
% The GPL requires the complete license text to be distributed along
% with the code. I recommend the canonical source, instead:
% \url{http://www.gnu.org/licenses/old-licenses/gpl-2.0.html}.
% But if you insist on an included copy, here it is.
% You might want to zoom in.
%
% \newsavebox{\gpl}
% \begin{lrbox}{\gpl}
% \begin{minipage}{3\textwidth}
% \columnsep=3\columnsep
% \begin{multicols}{3}
% \begin{center}
% {\Large GNU GENERAL PUBLIC LICENSE\par}
% \bigskip
% {Version 2, June 1991}
% \end{center}
%
% \begin{center}
% {\parindent 0in
%
% Copyright \textcopyright\ 1989, 1991 Free Software Foundation, Inc.
%
% \bigskip
%
% 51 Franklin Street, Fifth Floor, Boston, MA  02110-1301, USA
%
% \bigskip
%
% Everyone is permitted to copy and distribute verbatim copies
% of this license document, but changing it is not allowed.
% }
% \end{center}
%
% \begin{center}
% {\bf\large Preamble}
% \end{center}
%
%
% The licenses for most software are designed to take away your freedom to
% share and change it.  By contrast, the GNU General Public License is
% intended to guarantee your freedom to share and change free software---to
% make sure the software is free for all its users.  This General Public
% License applies to most of the Free Software Foundation's software and to
% any other program whose authors commit to using it.  (Some other Free
% Software Foundation software is covered by the GNU Library General Public
% License instead.)  You can apply it to your programs, too.
%
% When we speak of free software, we are referring to freedom, not price.
% Our General Public Licenses are designed to make sure that you have the
% freedom to distribute copies of free software (and charge for this service
% if you wish), that you receive source code or can get it if you want it,
% that you can change the software or use pieces of it in new free programs;
% and that you know you can do these things.
%
% To protect your rights, we need to make restrictions that forbid anyone to
% deny you these rights or to ask you to surrender the rights.  These
% restrictions translate to certain responsibilities for you if you
% distribute copies of the software, or if you modify it.
%
% For example, if you distribute copies of such a program, whether gratis or
% for a fee, you must give the recipients all the rights that you have.  You
% must make sure that they, too, receive or can get the source code.  And
% you must show them these terms so they know their rights.
%
% We protect your rights with two steps: (1) copyright the software, and (2)
% offer you this license which gives you legal permission to copy,
% distribute and/or modify the software.
%
% Also, for each author's protection and ours, we want to make certain that
% everyone understands that there is no warranty for this free software.  If
% the software is modified by someone else and passed on, we want its
% recipients to know that what they have is not the original, so that any
% problems introduced by others will not reflect on the original authors'
% reputations.
%
% Finally, any free program is threatened constantly by software patents.
% We wish to avoid the danger that redistributors of a free program will
% individually obtain patent licenses, in effect making the program
% proprietary.  To prevent this, we have made it clear that any patent must
% be licensed for everyone's free use or not licensed at all.
%
% The precise terms and conditions for copying, distribution and
% modification follow.
%
% \begin{center}
% {\Large \sc Terms and Conditions For Copying, Distribution and
%   Modification}
% \end{center}
%
% \begin{enumerate}
% \item
% This License applies to any program or other work which contains a notice
% placed by the copyright holder saying it may be distributed under the
% terms of this General Public License.  The ``Program'', below, refers to
% any such program or work, and a ``work based on the Program'' means either
% the Program or any derivative work under copyright law: that is to say, a
% work containing the Program or a portion of it, either verbatim or with
% modifications and/or translated into another language.  (Hereinafter,
% translation is included without limitation in the term ``modification''.)
% Each licensee is addressed as ``you''.
%
% Activities other than copying, distribution and modification are not
% covered by this License; they are outside its scope.  The act of
% running the Program is not restricted, and the output from the Program
% is covered only if its contents constitute a work based on the
% Program (independent of having been made by running the Program).
% Whether that is true depends on what the Program does.
%
% \item You may copy and distribute verbatim copies of the Program's source
%   code as you receive it, in any medium, provided that you conspicuously
%   and appropriately publish on each copy an appropriate copyright notice
%   and disclaimer of warranty; keep intact all the notices that refer to
%   this License and to the absence of any warranty; and give any other
%   recipients of the Program a copy of this License along with the Program.
%
% You may charge a fee for the physical act of transferring a copy, and you
% may at your option offer warranty protection in exchange for a fee.
%
% \item
% You may modify your copy or copies of the Program or any portion
% of it, thus forming a work based on the Program, and copy and
% distribute such modifications or work under the terms of Section 1
% above, provided that you also meet all of these conditions:
%
% \begin{enumerate}
%
% \item
% You must cause the modified files to carry prominent notices stating that
% you changed the files and the date of any change.
%
% \item
% You must cause any work that you distribute or publish, that in
% whole or in part contains or is derived from the Program or any
% part thereof, to be licensed as a whole at no charge to all third
% parties under the terms of this License.
%
% \item
% If the modified program normally reads commands interactively
% when run, you must cause it, when started running for such
% interactive use in the most ordinary way, to print or display an
% announcement including an appropriate copyright notice and a
% notice that there is no warranty (or else, saying that you provide
% a warranty) and that users may redistribute the program under
% these conditions, and telling the user how to view a copy of this
% License.  (Exception: if the Program itself is interactive but
% does not normally print such an announcement, your work based on
% the Program is not required to print an announcement.)
%
% \end{enumerate}
%
%
% These requirements apply to the modified work as a whole.  If
% identifiable sections of that work are not derived from the Program,
% and can be reasonably considered independent and separate works in
% themselves, then this License, and its terms, do not apply to those
% sections when you distribute them as separate works.  But when you
% distribute the same sections as part of a whole which is a work based
% on the Program, the distribution of the whole must be on the terms of
% this License, whose permissions for other licensees extend to the
% entire whole, and thus to each and every part regardless of who wrote it.
%
% Thus, it is not the intent of this section to claim rights or contest
% your rights to work written entirely by you; rather, the intent is to
% exercise the right to control the distribution of derivative or
% collective works based on the Program.
%
% In addition, mere aggregation of another work not based on the Program
% with the Program (or with a work based on the Program) on a volume of
% a storage or distribution medium does not bring the other work under
% the scope of this License.
%
% \item
% You may copy and distribute the Program (or a work based on it,
% under Section 2) in object code or executable form under the terms of
% Sections 1 and 2 above provided that you also do one of the following:
%
% \begin{enumerate}
%
% \item
%
% Accompany it with the complete corresponding machine-readable
% source code, which must be distributed under the terms of Sections
% 1 and 2 above on a medium customarily used for software interchange; or,
%
% \item
%
% Accompany it with a written offer, valid for at least three
% years, to give any third party, for a charge no more than your
% cost of physically performing source distribution, a complete
% machine-readable copy of the corresponding source code, to be
% distributed under the terms of Sections 1 and 2 above on a medium
% customarily used for software interchange; or,
%
% \item
%
% Accompany it with the information you received as to the offer
% to distribute corresponding source code.  (This alternative is
% allowed only for noncommercial distribution and only if you
% received the program in object code or executable form with such
% an offer, in accord with Subsection b above.)
%
% \end{enumerate}
%
%
% The source code for a work means the preferred form of the work for
% making modifications to it.  For an executable work, complete source
% code means all the source code for all modules it contains, plus any
% associated interface definition files, plus the scripts used to
% control compilation and installation of the executable.  However, as a
% special exception, the source code distributed need not include
% anything that is normally distributed (in either source or binary
% form) with the major components (compiler, kernel, and so on) of the
% operating system on which the executable runs, unless that component
% itself accompanies the executable.
%
% If distribution of executable or object code is made by offering
% access to copy from a designated place, then offering equivalent
% access to copy the source code from the same place counts as
% distribution of the source code, even though third parties are not
% compelled to copy the source along with the object code.
%
% \item
% You may not copy, modify, sublicense, or distribute the Program
% except as expressly provided under this License.  Any attempt
% otherwise to copy, modify, sublicense or distribute the Program is
% void, and will automatically terminate your rights under this License.
% However, parties who have received copies, or rights, from you under
% this License will not have their licenses terminated so long as such
% parties remain in full compliance.
%
% \item
% You are not required to accept this License, since you have not
% signed it.  However, nothing else grants you permission to modify or
% distribute the Program or its derivative works.  These actions are
% prohibited by law if you do not accept this License.  Therefore, by
% modifying or distributing the Program (or any work based on the
% Program), you indicate your acceptance of this License to do so, and
% all its terms and conditions for copying, distributing or modifying
% the Program or works based on it.
%
% \item
% Each time you redistribute the Program (or any work based on the
% Program), the recipient automatically receives a license from the
% original licensor to copy, distribute or modify the Program subject to
% these terms and conditions.  You may not impose any further
% restrictions on the recipients' exercise of the rights granted herein.
% You are not responsible for enforcing compliance by third parties to
% this License.
%
% \item
% If, as a consequence of a court judgment or allegation of patent
% infringement or for any other reason (not limited to patent issues),
% conditions are imposed on you (whether by court order, agreement or
% otherwise) that contradict the conditions of this License, they do not
% excuse you from the conditions of this License.  If you cannot
% distribute so as to satisfy simultaneously your obligations under this
% License and any other pertinent obligations, then as a consequence you
% may not distribute the Program at all.  For example, if a patent
% license would not permit royalty-free redistribution of the Program by
% all those who receive copies directly or indirectly through you, then
% the only way you could satisfy both it and this License would be to
% refrain entirely from distribution of the Program.
%
% If any portion of this section is held invalid or unenforceable under
% any particular circumstance, the balance of the section is intended to
% apply and the section as a whole is intended to apply in other
% circumstances.
%
% It is not the purpose of this section to induce you to infringe any
% patents or other property right claims or to contest validity of any
% such claims; this section has the sole purpose of protecting the
% integrity of the free software distribution system, which is
% implemented by public license practices.  Many people have made
% generous contributions to the wide range of software distributed
% through that system in reliance on consistent application of that
% system; it is up to the author/donor to decide if he or she is willing
% to distribute software through any other system and a licensee cannot
% impose that choice.
%
% This section is intended to make thoroughly clear what is believed to
% be a consequence of the rest of this License.
%
% \item
% If the distribution and/or use of the Program is restricted in
% certain countries either by patents or by copyrighted interfaces, the
% original copyright holder who places the Program under this License
% may add an explicit geographical distribution limitation excluding
% those countries, so that distribution is permitted only in or among
% countries not thus excluded.  In such case, this License incorporates
% the limitation as if written in the body of this License.
%
% \item
% The Free Software Foundation may publish revised and/or new versions
% of the General Public License from time to time.  Such new versions will
% be similar in spirit to the present version, but may differ in detail to
% address new problems or concerns.
%
% Each version is given a distinguishing version number.  If the Program
% specifies a version number of this License which applies to it and ``any
% later version'', you have the option of following the terms and conditions
% either of that version or of any later version published by the Free
% Software Foundation.  If the Program does not specify a version number of
% this License, you may choose any version ever published by the Free Software
% Foundation.
%
% \item
% If you wish to incorporate parts of the Program into other free
% programs whose distribution conditions are different, write to the author
% to ask for permission.  For software which is copyrighted by the Free
% Software Foundation, write to the Free Software Foundation; we sometimes
% make exceptions for this.  Our decision will be guided by the two goals
% of preserving the free status of all derivatives of our free software and
% of promoting the sharing and reuse of software generally.
%
% \begin{center}
% {\Large\sc
% No Warranty
% }
% \end{center}
%
% \item
% {\sc Because the program is licensed free of charge, there is no warranty
% for the program, to the extent permitted by applicable law.  Except when
% otherwise stated in writing the copyright holders and/or other parties
% provide the program ``as is'' without warranty of any kind, either expressed
% or implied, including, but not limited to, the implied warranties of
% merchantability and fitness for a particular purpose.  The entire risk as
% to the quality and performance of the program is with you.  Should the
% program prove defective, you assume the cost of all necessary servicing,
% repair or correction.}
%
% \item
% {\sc In no event unless required by applicable law or agreed to in writing
% will any copyright holder, or any other party who may modify and/or
% redistribute the program as permitted above, be liable to you for damages,
% including any general, special, incidental or consequential damages arising
% out of the use or inability to use the program (including but not limited
% to loss of data or data being rendered inaccurate or losses sustained by
% you or third parties or a failure of the program to operate with any other
% programs), even if such holder or other party has been advised of the
% possibility of such damages.}
%
% \end{enumerate}
%
%
% \begin{center}
% {\Large\sc End of Terms and Conditions}
% \end{center}
%
%
% \pagebreak[2]
%
% \section*{Appendix: How to Apply These Terms to Your New Programs}
%
% If you develop a new program, and you want it to be of the greatest
% possible use to the public, the best way to achieve this is to make it
% free software which everyone can redistribute and change under these
% terms.
%
%   To do so, attach the following notices to the program.  It is safest to
%   attach them to the start of each source file to most effectively convey
%   the exclusion of warranty; and each file should have at least the
%   ``copyright'' line and a pointer to where the full notice is found.
%
% \begin{quote}
% one line to give the program's name and a brief idea of what it does. \\
% Copyright (C) yyyy  name of author \\
%
% This program is free software; you can redistribute it and/or modify
% it under the terms of the GNU General Public License as published by
% the Free Software Foundation; either version 2 of the License, or
% (at your option) any later version.
%
% This program is distributed in the hope that it will be useful,
% but WITHOUT ANY WARRANTY; without even the implied warranty of
% MERCHANTABILITY or FITNESS FOR A PARTICULAR PURPOSE.  See the
% GNU General Public License for more details.
%
% You should have received a copy of the GNU General Public License
% along with this program; if not, write to the Free Software
% Foundation, Inc., 51 Franklin Street, Fifth Floor, Boston, MA  02110-1301, USA.
% \end{quote}
%
% Also add information on how to contact you by electronic and paper mail.
%
% If the program is interactive, make it output a short notice like this
% when it starts in an interactive mode:
%
% \begin{quote}
% Gnomovision version 69, Copyright (C) yyyy  name of author \\
% Gnomovision comes with ABSOLUTELY NO WARRANTY; for details type `show w'. \\
% This is free software, and you are welcome to redistribute it
% under certain conditions; type `show c' for details.
% \end{quote}
%
%
% The hypothetical commands {\tt show w} and {\tt show c} should show the
% appropriate parts of the General Public License.  Of course, the commands
% you use may be called something other than {\tt show w} and {\tt show c};
% they could even be mouse-clicks or menu items---whatever suits your
% program.
%
% You should also get your employer (if you work as a programmer) or your
% school, if any, to sign a ``copyright disclaimer'' for the program, if
% necessary.  Here is a sample; alter the names:
%
% \begin{quote}
% Yoyodyne, Inc., hereby disclaims all copyright interest in the program \\
% `Gnomovision' (which makes passes at compilers) written by James Hacker. \\
%
% signature of Ty Coon, 1 April 1989 \\
% Ty Coon, President of Vice
% \end{quote}
%
%
% This General Public License does not permit incorporating your program
% into proprietary programs.  If your program is a subroutine library, you
% may consider it more useful to permit linking proprietary applications
% with the library.  If this is what you want to do, use the GNU Library
% General Public License instead of this License.
%
% \end{multicols}
% \end{minipage}
% \end{lrbox}
%
% \begin{center}
% \scalebox{0.33}{\usebox{\gpl}}
% \end{center}
%
% \Finale
\endinput
